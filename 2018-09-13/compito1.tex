%
\documentclass[12pt]{article}
\usepackage{amssymb}
\usepackage{amsmath}
\usepackage{color}
\usepackage{graphicx}

\definecolor{grey}{rgb}{0.3,0.3,0.3}

\usepackage{listings, framed}
\lstset{
  language=Java,
  showstringspaces=false,
  columns=flexible,
  basicstyle={\small\ttfamily},
  frame=none,
  numbers=none,
  keywordstyle=\bfseries\color{grey},
  commentstyle=\itshape\color{red},
  identifierstyle=\color{black},
  stringstyle=\color{blue},
  numberstyle={\ttfamily},
%  breaklines=true,
  breakatwhitespace=true,
  tabsize=3,
  escapechar=|
}

\thispagestyle{empty}
\setlength{\textwidth}{18.5cm}
\setlength{\topmargin}{-2.5cm}
\setlength{\textheight}{24.5cm}
\setlength{\oddsidemargin}{-1cm}
\setlength{\evensidemargin}{-1cm}
\begin{document}

\begin{center}{\LARGE Compito di Programmazione I - BioInformatica}\\
\vspace*{-2ex}
\begin{center}
  \large 13 settembre 2018 (tempo disponibile: 2 ore)
\end{center}
\end{center}

\begin{center}{\Large Esercizio 1} ($6$ punti)
\end{center}
Che numero stampa l'esecuzione del seguente programma?
\begin{lstlisting}
#include <stdio.h>

int foo(int arr[], int length) {
  int counter = 0, i;
  for (i = 0; i < length - 1; i += 2)
    if (arr[i] > arr[i + 1])
      counter++;

  return counter;
}

int main(void) {
  int original[] = { -8, 6, 7, -5, 8, -1, 0, 4, 3 };
  printf("%d\n", foo(original, 9));  // cosa stampa?

  return 0;
}
\end{lstlisting}

\vspace*{1ex}
\begin{center}{\Large Esercizio 2} ($9$ punti)
\end{center}
Si scriva una funzione \texttt{divide} che riceve tre parametri: un array
\texttt{original} di \texttt{int}, un array \texttt{destination} di \texttt{double}
e la lunghezza dei due array (che si assume essere la stessa per entrambi).
La funzione deve modificare \texttt{destination} in modo che ogni suo elemento
in posizione $i$ diventi l'elemento di \texttt{original} in posizione $i$
diviso per $i + 1$ (risultato con la virgola). Per esempio, l'esecuzione
del seguente programma:

\begin{lstlisting}
#include <stdio.h>

// la funzione divide va aggiunta qui

int main(void) {
  int i, original[] = { 5, 4, 3, 2, 1, 0 };
  double destination[6];
  divide(original, destination, 6);
  for (i = 0; i < 6; i++)
    printf("%.2f ", destination[i]);
  return 0;
}
\end{lstlisting}
dovr\`a stampare \texttt{5.00 2.00 1.00 0.50 0.20 0.00}.
\newpage

\begin{center}{\Large Esercizio 3} ($11$ punti)\end{center}
%
Si scriva un file \texttt{registro.h} che definisce una struttura
\texttt{registro} che implementa un registro per una campagna raccolta punti.
I partecipanti alla campagna sono al massimo $10$ e sono identificati per cognome.
La struttura \texttt{registro} sar\`a fatta da un array di $10$ stringhe (i cognomi dei
partecipanti) e da un array di $10$ \texttt{double} (i punti di ciascun partecipante) e non
dovr\`a avere altre componenti.
Si scriva il file \texttt{registro.c} che implementa le seguenti funzioni, dichiarandole
in \texttt{registro.h}:
%
\begin{itemize}
\item \texttt{struct registro *construct\_registro()} che restituisce un nuovo registro vuoto;
\item \texttt{void destruct\_registro(struct registro *this)} che dealloca il registro \texttt{this};
\item \texttt{void aggiungi\_punti(struct registro *this, char *cognome, int punti)},
      che aumenta di \texttt{punti} i punti del partecipante \texttt{cognome}, se si
      trova in \texttt{this}. Se il partecipante non \`e
      in \texttt{this} e non si \`e ancora arrivati al massimo di $10$ partecipanti, questa funzione
      lo aggiunge come nuovo partecipante con punti
      iniziali pari a \texttt{punti}.  Se il partecipante non \`e
      in \texttt{this} e si \`e gi\`a arrivati a $10$ partecipanti, questa funzione non
      fa nulla;
\item \texttt{void bonus(struct registro *this, double percent)}, che aumenta i
      punti di ciascun partecipante del \texttt{percent} per cento.
\end{itemize}
%
%Se tutto \`e corretto, l'esecuzione del seguente programma:

%{\small
%\begin{verbatim}
%#include <stdlib.h>
%#include <stdio.h>
%#include "registro.h"
%
%int main(void) {
%  struct registro *r = construct_registro();
%  aggiungi_punti(r, "Rossi", 112); aggiungi_punti(r, "Bianchi", 1023);
%  aggiungi_punti(r, "Rossi", 13); aggiungi_punti(r, "Verdi", 11);
%  int pos;
%  for (pos = 0; pos < 10 && r->cognome[pos]; pos++)
%    printf("%s: %f\n", r->cognome[pos], r->saldo_punti[pos]);
%  printf("\n");
%  bonus(r, 2.50);
%  for (pos = 0; pos < 10 && r->cognome[pos]; pos++)
%    printf("%s: %f\n", r->cognome[pos], r->saldo_punti[pos]);
%  destruct_registro(r);
%  return 0;
%}
%\end{verbatim}}
%
%\noindent
%dovr\`a stampare:
%
%{\small
%\begin{verbatim}
%Rossi: 125.000000
%Bianchi: 1023.000000
%Verdi: 11.000000
%
%Rossi: 128.125000
%Bianchi: 1048.575000
%Verdi: 11.275000
%\end{verbatim}}

\vspace*{1ex}

\begin{center}
{\Large Esercizio 4} (6 punti)
\end{center}

\noindent Cosa stampa il seguente programma C?

\begin{lstlisting}
#include <stdio.h>

void f(int *p){
  *p = 5;
}

void g(int *p){
  int i = 12;
  p = &i;
}

int main(void){
  int n = 8;
  int *p = &n;
  printf("Il valore puntato da p e' uguale a %d.\n", *p);
  f(p);
  printf("Il valore puntato da p e' uguale a %d.\n", *p);
  g(p);
  printf("Il valore puntato da p e' uguale a %d.\n", *p);
}
\end{lstlisting}

\end{document}
