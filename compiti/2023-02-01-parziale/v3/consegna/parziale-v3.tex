\documentclass[12pt]{article}
\usepackage{amssymb}
\usepackage{amsmath}
\usepackage{color}
\usepackage{graphicx}
\usepackage{mdframed}
\usepackage{listings, xcolor}
\usepackage{textcomp}

\definecolor{verylightgray}{rgb}{.97,.97,.97}

\lstdefinelanguage{myC}{
        keywords=[1]{break, case, continue, default, do
, else, false, for, if, const, return, switch, true, while}, % generic keywords
        keywordstyle=[1]\color{blue}\bfseries,
        keywords=[2]{bool, int, long, float, double, byte, short, char, void, signed, unsigned}, % types
        keywordstyle=[2]\color{teal}\bfseries,
        keywordstyle=[2]\color{violet}\bfseries,
        keywords=[3]{NULL},
        keywordstyle=[3]\color{teal}\bfseries,
        identifierstyle=\color{black},
        sensitive=false,
        comment=[l]{//},
        morecomment=[s]{/*}{*/},
        commentstyle=\color{violet}\ttfamily,
        commentstyle=\small\ttfamily\color{myGreen},
        stringstyle=\color{red}\ttfamily,
        morestring=[b]',
        morestring=[b]"
}

\lstset{
        language=myC,
        backgroundcolor=\color{verylightgray},
        extendedchars=true,
        basicstyle=\small\ttfamily,
        showstringspaces=false,
        showspaces=false,
        numbers=none,
        numberstyle=\small,
        numbersep=9pt,
        tabsize=2,
        upquote=true,
        breaklines=true,
        showtabs=false,
        captionpos=b
        otherkeywords={define,include,\# }
}

\definecolor{myBlue}{rgb}{0.5,0.5,1}
\definecolor{myLightBlue}{rgb}{0.35,0.6,0.8}
\definecolor{myBlack}{rgb}{0,0,0}
\definecolor{myGreen}{rgb}{0.1,0.6,0.2}
\definecolor{myGray}{rgb}{0.5,0.5,0.5}
\definecolor{myLightgray}{rgb}{0.95,0.95,0.95}
\definecolor{myMauve}{rgb}{0.58,0,0.82}
\lstdefinelanguage{customc}{
    language=C,
    backgroundcolor = \color{myLightgray},
    basicstyle=\small\ttfamily\color{myBlack},
    keywordstyle=\color{myLightBlue},
    keywordstyle=[2]\color{red},
    commentstyle=\small\ttfamily\color{myGreen},
    morekeywords={RequirePackage,ProvidesPackage},
    %
    % The special highlighting works for '!if', '!endif' and '!else'
    % But it doesn't work for '#if', '#endif' and '#else'.
    alsoletter = {!},
    keywords=[2]{!if,!endif,!else},
    otherkeywords={define,include,\# }
}

\lstdefinestyle{myCustomc}{
    language = customc,
    % keywordstyle = \color{myMauve},
}

\lstset{escapechar=@,style=myCustomc}


\definecolor{grey}{rgb}{0.3,0.3,0.3}
\definecolor{lightgrey}{rgb}{0.9,0.9,0.9}

\thispagestyle{empty}
\setlength{\textwidth}{18.5cm}
\setlength{\topmargin}{-2.5cm}
\setlength{\textheight}{24.5cm}
\setlength{\oddsidemargin}{-1cm}
\setlength{\evensidemargin}{-1cm}

\begin{document}
\begin{center}{\LARGE Parziale di Programmazione I - Bioinformatica}\\
  \large 1 febbraio 2023, turno delle 15:00 (tempo disponibile: 2 ore)
\end{center}

\vspace*{1ex}
\begin{center}{\Large Esercizio 1} ($20$ punti)\\
  \textbf{(si consegni \texttt{pari\_dispari.c} e \texttt{pari\_dispari.h})}
\end{center}

Si completi il seguente file \texttt{pari\_dispari.c}:

\begin{center}
\begin{lstlisting}[language=myC]
// aggiungete #include se servono

// inizializza arr, lungo length, con numeri casuali non negativi divisibili
// per 5 in modo da non avere mai ne' due numeri pari in posizioni
// consecutive ne' due numeri dispari in posizioni consecutive; non
// randomizza la sequenza casuale con srand(time(NULL)): ci pensa il main
void init(int arr[], int length) { // completare
}

// stampa arr, lungo length, separando gli elementi
// con uno spazio, e va a capo
void print(int arr[], int length) { // completare
}

// determina se in arr, lungo length, la quantita' dei numeri dispari
// e' maggiore o uguale alla quantita' dei numeri pari; si assuma
// che arr non abbia mai ne' due numeri pari in posizioni consecutive
// ne' due numeri dispari in posizioni consecutive
int dispari_almeno_quanto_i_pari(int arr[], int length) { // completare
}
\end{lstlisting}
\end{center}
%
Si scriva quindi un file di header \texttt{pari\_dispari.h} che dichiara le precedenti funzioni.

Per esempio, il seguente file \texttt{main.c}
(gi\`a fornito e da non modificare):

\begin{center}
  \begin{lstlisting}[language=myC]
#include <stdio.h>
#include <stdlib.h>
#include <time.h>
#include "pari_dispari.h"

int main(void) {
  srand(time(NULL));
  int array1[20];
  init(array1, 20);
  printf("array1: ");
  print(array1, 20);
  printf("In array1 dispari >= pari ? %i\n", dispari_almeno_quanto_i_pari(array1, 20));
  int array2[15];
  init(array2, 15);
  printf("array2: ");
  print(array2, 15);
  printf("In array2 dispari >= pari ? %i\n", dispari_almeno_quanto_i_pari(array2, 15));
  int array3[17];
  init(array3, 17);
  printf("array3: ");
  print(array3, 17);
  printf("In array3 dispari >= pari ? %i\n", dispari_almeno_quanto_i_pari(array3, 17));
  return 0;
}
  \end{lstlisting}
\end{center}
stampa qualcosa del tipo:

\begin{mdframed}[backgroundcolor=lightgrey] 
{\small\begin{verbatim}
array1: 1390741700 1729062575 476161820 466897965 399090190 26084305 251372970 799201515
  1094977550 670565265 1743740570 186827875 1663476810 1017154705 904737510 597427485
  2044555980 678609605 212097830 807774915 
In array1 dispari >= pari ? 1
array2: 13150390 1810090715 1792295540 466060805 1560237680 283637865 247794490 855049935
  1862099380 1077922515 524959800 369117805 1744420990 630703485 418630810 
In array2 dispari >= pari ? 0
array3: 2007145705 1903055230 1870673145 2008279780 1306383695 60525760 1518577195
  1129157370 1796086945 38722370 2093968385 274308190 899586285 1623976460 1024382915
  434370140 530517665 
In array3 dispari >= pari ? 1
\end{verbatim}}
\end{mdframed}

\begin{center}
  {\Large Esercizio 2} ($11$ punti)\\
  \textbf{(si consegni \texttt{triangular.c})}
\end{center}
%
Si completi il seguente programma \texttt{triangular.c}.
\textbf{La funzione triangular deve essere ricorsiva}:

\begin{center}
  \begin{lstlisting}[language=myC]
#include <stdio.h>

// stampa i volte c, poi i-1 volte il carattere precedente a c
// poi i-2 volte il carattere precedente al carattere precedente a c, ecc.
void triangular(char c, int i) { // completare
}

int main(void) {
  triangular('j', 5); printf("\n");
  triangular('s', 8); printf("\n");
  return 0;
}
  \end{lstlisting}
\end{center}

Se tutto \`e corretto, tale programma dovrebbe stampare:

\begin{mdframed}[backgroundcolor=lightgrey] 
{\small\begin{verbatim}
jjjjjiiiihhhggf
ssssssssrrrrrrrqqqqqqpppppoooonnnmml
\end{verbatim}}
\end{mdframed}

\end{document}
