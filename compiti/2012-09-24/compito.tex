\documentclass{article}[10pt]
\usepackage[pdftex]{graphicx}
\usepackage{amsfonts}
\usepackage[italian]{babel}
%****************enlarge layout
\textheight     243.5mm
\topmargin      -20.0mm
\textwidth      480pt
\hoffset        -80pt
%*****************theorems and such
\newcounter{esnu}
\newenvironment{esercizio}{\medskip \noindent {\bf Esercizio\addtocounter{esnu}{1} \arabic{esnu}}}{}
\pagestyle{empty}
\newcommand{\liff}{\mathrel{\leftrightarrow}}   % Logical IFF Symbol
\newcommand{\metaiff}{\Longleftrightarrow}      %iff in metatheory

\begin{document}

\begin{tabular}{llclcr}
 \hspace{-35pt} &{\bf COGNOME:} & \hspace{100pt}        &{\bf NOME:}    & \hspace{100pt}        &{\bf MATRICOLA:}\hspace{35pt} \\
\hline
\end{tabular}
\begin{center} {\bf Esame di Programmazione I, 24 settembre 2012. 2 ore}\end{center}

\begin{esercizio}
\textbf{[14 punti]}
Si scriva una funzione

\begin{verbatim}
char *replace(char *s, char c, char *replacement)
\end{verbatim}

\noindent
che restituisce una nuova stringa ottenuta sostituendo con \texttt{replacement}
ogni occorrenza di \texttt{c} dentro \texttt{s}.

Se tutto \`e corretto, l'esecuzione del seguente programma:

{\small
\begin{verbatim}
int main(void) {
  char *s = "ciao amico";
  char *risultato;

  printf("%s\n", risultato = replace(s, 'a', "hello"));
  free(risultato);

  printf("%s\n", risultato = replace(s, 'u', "hello"));
  free(risultato);

  // si noti: il terminatore \0 non e' un carattere della stringa!
  printf("%s\n", risultato = replace(s, '\0', "hello"));
  free(risultato);

  return 0;
}
\end{verbatim}
}

\noindent
deve stampare:

{\small
\begin{verbatim}
cihelloo hellomico
ciao amico
ciao amico
\end{verbatim}
}

\end{esercizio}

\begin{esercizio}
\textbf{[9 punti]}
Si considerino le liste viste a lezione. Si implementi una funzione ricorsiva:
\begin{verbatim}
struct list *fib(int n);
\end{verbatim}
che restituisce la lista dei primi \texttt{n} numeri di Fibonacci, in ordine inverso.

Se tutto \`e corretto, l'esecuzione del programma:

{\small
\begin{verbatim}
#include <stdio.h>
#include <stdlib.h>
#include "list.h"

int main(void) {
  print_list(fib(10));
  printf("\n");

  print_list(fib(0));
  printf("\n");

  return 0;
}
\end{verbatim}
}
\noindent
dovr\`a stampare:

{\small
\begin{verbatim}
[55, 34, 21, 13, 8, 5, 3, 2, 1, 1]
[]
\end{verbatim}
}
%
\end{esercizio}

\pagebreak

\begin{esercizio}
\textbf{[9 punti]}
%
Si definisca una struttura \texttt{studente} che implementa uno studente.
Si scrivano i file \texttt{studente.h} e \texttt{studente.c} implementando le funzioni:
%
\begin{itemize}
\item \texttt{struct studente *construct\_studente(char *nome)} che restituisce un
      nuovo studente con il nome indicato;
\item \texttt{void destruct\_studente(struct studente *this)} che dealloca lo studente \texttt{this};
\item \texttt{void fa\_esame(struct studente *this, int voto)},
      che registra il voto indicato per lo studente \texttt{this}, se il voto \`e
      fra $18$ e $30$ inclusi, e non fa nulla altrimenti; non c'\`e limite al
      numero di esami che uno studente pu\`o fare;
\item \texttt{float media(struct studente *this)}, che restituisce la media dei voti
      degli esami sostenuti dallo studente \texttt{this}; se lo studente non ha ancora fatto esami,
      restituisce $0.0$.
\item \texttt{char *toString(struct studente *this)}, che restituisce una nuova stringa
      fatta dal nome dello studente \texttt{this} seguito dalla media degli esami sostenuti da
      \texttt{this}.
\end{itemize}
%
Se tutto \`e corretto, l'esecuzione del seguente programma:

{\small
\begin{verbatim}
#include <stdlib.h>
#include <stdio.h>
#include "studente.h"

int main(void) {
  struct studente *s1 = construct_studente("Giacomo");
  struct studente *s2 = construct_studente("Elisa");
  char *s;

  fa_esame(s1, 18);
  fa_esame(s1, 15);  // non viene registrato
  fa_esame(s2, 30);
  fa_esame(s1, 25);
  fa_esame(s2, 22);
  fa_esame(s2, 29);
  fa_esame(s2, 27);

  printf("%s\n", s = toString(s1));
  free(s);

  printf("%s\n", s = toString(s2));
  free(s);

  destruct_studente(s1);
  destruct_studente(s2);

  return 0;
}
\end{verbatim}}

\noindent
deve stampare:

{\small
\begin{verbatim}
Giacomo  21.50
Elisa  27.00
\end{verbatim}
}

\end{esercizio}
%
\end{document}
