\documentclass[12pt]{article}
\usepackage{amssymb}
\usepackage{amsmath}
\usepackage{color}
\usepackage{graphicx}

\definecolor{grey}{rgb}{0.3,0.3,0.3}

\usepackage{listings, framed}
\lstset{
  language=Java,
  showstringspaces=false,
  columns=flexible,
  basicstyle={\small\ttfamily},
  frame=none,
  numbers=none,
  keywordstyle=\bfseries\color{grey},
  commentstyle=\itshape\color{red},
  identifierstyle=\color{black},
  stringstyle=\color{blue},
  numberstyle={\ttfamily},
%  breaklines=true,
  breakatwhitespace=true,
  tabsize=3,
  escapechar=|
}

\thispagestyle{empty}
\setlength{\textwidth}{18.5cm}
\setlength{\topmargin}{-2.5cm}
\setlength{\textheight}{24.5cm}
\setlength{\oddsidemargin}{-1cm}
\setlength{\evensidemargin}{-1cm}
\begin{document}
\begin{center}{\LARGE Parziale di Programmazione I - BioInformatica}\\
\vspace*{-2ex}
\begin{center}
  \large 1 febbraio 2019 (tempo disponibile: 2 ore)
\end{center}
\end{center}

\vspace*{1ex}
\begin{center}{\Large Esercizio 1} ($9$ punti)
\end{center}
Si scriva un programma \texttt{sum.c} che implementa una funzione \texttt{int sum(int arr[], int length)}. Tale funzione deve ricevere un array \texttt{arr} di interi, lungo \texttt{length}, e deve restituire la somma degli elementi di \texttt{arr} che abbiano l'elemento precedente dispari. Per esempio, se \texttt{arr} fosse $\{2,3,4,1,5\}$, la funzione dovrebbe restituire $9$ (la somma di $4$ e $5$). Si scriva il file di header \texttt{sum.h} in cui si dichiara tale funzione.

\vspace*{1ex}
\begin{center}{\Large Esercizio 2} ($11$ punti)\end{center}
%
Si scriva un programma \texttt{main\_sum.c} che include la funzione dell'Esercizio~1 tramite
il file di header \texttt{sum.h}.
Il programma \texttt{main\_sum.c} deve contenere una funzione iniziale \texttt{main} che esegue
le seguenti operazioni:
\begin{enumerate}
\item legge da tastiera la lunghezza \texttt{length} di un array, richiedendola ad oltranza se fosse inserita negativa;
\item crea un array \texttt{elements} di \texttt{length} interi;
\item legge da tastiera gli elementi di tale array, uno alla volta;
\item chiama la funzione \texttt{sum} dell'Esercizio~1, passando \texttt{elements} e \texttt{length};
\item stampa sul video il risultato di tale chiamata.
\end{enumerate}

\vspace*{1ex}
\begin{center}{\Large Esercizio 3} ($12$ punti)\end{center}

Si scriva un programma \texttt{cross.c} con una funzione iniziale \texttt{main}
che esegue le seguenti operazioni:
\begin{enumerate}
\item legge da tastiera una dimensione intera \texttt{size}, dispari e positiva. Se non lo fosse,
  la richiede ad oltranza;
\item stampa su video una croce di asterischi, le cui aste hanno lunghezza \texttt{size}.
\end{enumerate}

Per esempio, se l'utente inserisse \texttt{5} come \texttt{size}, il programma dovrebbe stampare:
%
\begin{verbatim}
  *
  *
*****
  *
  *
\end{verbatim}

\end{document}
