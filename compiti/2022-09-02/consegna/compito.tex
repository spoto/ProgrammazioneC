\documentclass[12pt]{article}
\usepackage{amssymb}
\usepackage{amsmath}
\usepackage{color}
\usepackage{graphicx}
\usepackage{mdframed}
\usepackage{listings, xcolor}
\usepackage{textcomp}

\definecolor{verylightgray}{rgb}{.97,.97,.97}
\definecolor{lightred}{rgb}{.97,.50,.50}

\lstdefinelanguage{myC}{
        keywords=[1]{break, case, continue, default, do
, else, false, for, if, const, return, switch, true, while}, % generic keywords
        keywordstyle=[1]\color{blue}\bfseries,
        keywords=[2]{bool, int, long, float, double, byte, short, char, void, signed, unsigned}, % types
        keywordstyle=[2]\color{teal}\bfseries,
        keywordstyle=[2]\color{violet}\bfseries,
        keywords=[3]{NULL},
        keywordstyle=[3]\color{teal}\bfseries,
        identifierstyle=\color{black},
        sensitive=false,
        comment=[l]{//},
        morecomment=[s]{/*}{*/},
        commentstyle=\color{violet}\ttfamily,
        commentstyle=\small\ttfamily\color{myGreen},
        stringstyle=\color{red}\ttfamily,
        morestring=[b]',
        morestring=[b]"
}

\lstset{
        language=myC,
        backgroundcolor=\color{verylightgray},
        extendedchars=true,
        basicstyle=\small\ttfamily,
        showstringspaces=false,
        showspaces=false,
        numbers=none,
        numberstyle=\small,
        numbersep=9pt,
        tabsize=2,
        upquote=true,
        breaklines=true,
        showtabs=false,
        captionpos=b
        otherkeywords={define,include,\# }
}

\definecolor{myBlue}{rgb}{0.5,0.5,1}
\definecolor{myLightBlue}{rgb}{0.35,0.6,0.8}
\definecolor{myBlack}{rgb}{0,0,0}
\definecolor{myGreen}{rgb}{0.1,0.6,0.2}
\definecolor{myGray}{rgb}{0.5,0.5,0.5}
\definecolor{myLightgray}{rgb}{0.95,0.95,0.95}
\definecolor{myMauve}{rgb}{0.58,0,0.82}
\lstdefinelanguage{customc}{
    language=C,
    backgroundcolor = \color{myLightgray},
    basicstyle=\small\ttfamily\color{myBlack},
    keywordstyle=\color{myLightBlue},
    keywordstyle=[2]\color{red},
    commentstyle=\small\ttfamily\color{myGreen},
    morekeywords={RequirePackage,ProvidesPackage},
    %
    % The special highlighting works for '!if', '!endif' and '!else'
    % But it doesn't work for '#if', '#endif' and '#else'.
    alsoletter = {!},
    keywords=[2]{!if,!endif,!else},
    otherkeywords={define,include,\# }
}

\lstdefinestyle{myCustomc}{
    language = customc,
    % keywordstyle = \color{myMauve},
}

\lstset{escapechar=@,style=myCustomc}


\definecolor{grey}{rgb}{0.3,0.3,0.3}
\definecolor{lightgrey}{rgb}{0.9,0.9,0.9}

\thispagestyle{empty}
\setlength{\textwidth}{18.5cm}
\setlength{\topmargin}{-2.5cm}
\setlength{\textheight}{24.5cm}
\setlength{\oddsidemargin}{-1cm}
\setlength{\evensidemargin}{-1cm}

\begin{document}
\begin{center}{\LARGE Compito di Programmazione I - Bioinformatica}\\
\begin{center}
  \large 2 settembre 2022 (tempo disponibile: 2 ore)
\end{center}
\end{center}

\vspace*{1ex}
\begin{center}{\Large Esercizio 1} ($31$ punti)\\
  \textbf{(si consegni \texttt{vulcaniano.c})}
\end{center}

Un array di caratteri \`e detto \emph{minuscolo} se i suoi caratteri
sono tutti lettere minuscole dell'alfabeto inglese (eventualmente ripetute).
Per esempio, l'array \texttt{['j','p','b','j','i','e']} \`e minuscolo
mentre l'array \texttt{['j','(','b','j','A','e']} non \`e minuscolo.

Un array di caratteri \`e detto \emph{vulcaniano} se \`e minuscolo
e, inoltre, \`e composto da una prima parte che contiene
vocali, in ordine alfabetico, e da una seconda parte contiene
consonanti, sempre in ordine alfabetico.
Per esempio, l'array
\[
\mbox{\texttt{[}}\underbrace{\mbox{\texttt{'e','i'}}}_{\text{vocali in ordine alfabetico}}\mbox{\texttt{,}}\underbrace{\mbox{\texttt{'b','j','j','p'}}}_{\text{consonanti in ordine alfabetico}}\mbox{\texttt{]}}
\]
\`e vulcaniano.
Invece l'array \texttt{['e','i','B','j','j','p']} non \`e vulcaniano,
perch\'e non \`e minuscolo.
Neanche l'array \texttt{['e','b','i','j','j','p']} \`e vulcaniano, perch\'e la
vocale \texttt{i} segue la consonante \texttt{b}.
Neanche l'array \texttt{['e','i','j','j','b','p']} \`e vulcaniano, perch\'e
le consonanti non sono in ordine alfabetico.
E neanche l'array \texttt{['i','e','b','j','j','p']} \`e vulcaniano, perch\'e
le vocali non sono in ordine alfabetico.

Si completino le cinque funzioni del programma \texttt{vulcaniano.c}:

\begin{center}
  \begin{lstlisting}[language=myC]
#include "vulcaniano.h"
// COMPLETARE CON GLI #include NECESSARI

// inizializza l'array indicato, lungo length,
// in modo che diventi un array minuscolo casuale (5 punti)
void init_random(char arr[], int length) {
  // COMPLETARE
}

// stampa l'array indicato, su una riga, senza spazi fra i caratteri,
// andando a capo alla fine (2 punti)
void print(char arr[], int length) {
  // COMPLETARE
}

// riceve un array minuscolo lungo length e ne sposta i caratteri in modo
// che l'array diventi vulcaniano (13 punti)
void ordina_vulcaniano(char arr[], int length) {
  // COMPLETARE
}

// stampa una lista di caratteri, senza spazio fra i caratteri,
// andando a capo alla fine (3 punti)
void print_list(struct element_t *l) {
  // COMPLETARE
}

// riceve un array vulcaniano lungo length e restituisce una lista
// che contiene gli stessi caratteri dell'array, nello stesso ordine,
// ma senza ripetizioni: un carattere viene inserito nella lista solo
// la prima volta che compare, mentre la seconda volta non viene
// inserito nella lista (8 punti)
struct element_t *non_ripetuti(char arr[], int length) {
  // COMPLETARE
}
  \end{lstlisting}
\end{center}

\begin{mdframed}[backgroundcolor=lightred] 
  \textbf{I file \texttt{vulcaniano.h} e \texttt{main.c} sono gi\`a scritti e completi, non vanno modificati e non vanno consegnati. Se servisse, si possono aggiungere funzioni ausiliarie dentro \texttt{vulcaniano.c}.}
\end{mdframed}

\vspace*{5ex}
Se tutto \`e corretto,
un esempio di esecuzione del \texttt{main.c} potrebbe essere:

\begin{mdframed}[backgroundcolor=verylightgray] 
\begin{verbatim}
Inserisci la lunghezza dell'array, non negativa: 6
                            Array minuscolo casuale: jpbjie
                    Array trasformato in vulcaniano: eibjjp
Lista derivata dall'array eliminando le ripetizioni: eibjp
\end{verbatim}
\end{mdframed}

Un altro esempio di esecuzione del \texttt{main.c} potrebbe essere:

\begin{mdframed}[backgroundcolor=verylightgray] 
\begin{verbatim}
Inserisci la lunghezza dell'array, non negativa: 21
                            Array minuscolo casuale: elnmjkbpemuaytfawnfos
                    Array trasformato in vulcaniano: aaeeoubffjklmmnnpstwy
Lista derivata dall'array eliminando le ripetizioni: aeoubfjklmnpstwy
\end{verbatim}
\end{mdframed}

\end{document}
