\documentclass[12pt]{article}
\usepackage{amssymb}
\usepackage{amsmath}
\usepackage{color}
\usepackage{graphicx}
\usepackage{mdframed}
\usepackage{listings, xcolor}
\usepackage{textcomp}

\definecolor{verylightgray}{rgb}{.97,.97,.97}

\lstdefinelanguage{myC}{
        keywords=[1]{break, case, continue, default, do
, else, false, for, if, const, return, switch, true, while}, % generic keywords
        keywordstyle=[1]\color{blue}\bfseries,
        keywords=[2]{bool, int, long, float, double, byte, short, char, void, signed, unsigned}, % types
        keywordstyle=[2]\color{teal}\bfseries,
        keywordstyle=[2]\color{violet}\bfseries,
        keywords=[3]{NULL},
        keywordstyle=[3]\color{teal}\bfseries,
        identifierstyle=\color{black},
        sensitive=false,
        comment=[l]{//},
        morecomment=[s]{/*}{*/},
        commentstyle=\color{violet}\ttfamily,
        commentstyle=\small\ttfamily\color{myGreen},
        stringstyle=\color{red}\ttfamily,
        morestring=[b]',
        morestring=[b]"
}

\lstset{
        language=myC,
        backgroundcolor=\color{verylightgray},
        extendedchars=true,
        basicstyle=\small\ttfamily,
        showstringspaces=false,
        showspaces=false,
        numbers=none,
        numberstyle=\small,
        numbersep=9pt,
        tabsize=2,
        upquote=true,
        breaklines=true,
        showtabs=false,
        captionpos=b
        otherkeywords={define,include,\# }
}

\definecolor{myBlue}{rgb}{0.5,0.5,1}
\definecolor{myLightBlue}{rgb}{0.35,0.6,0.8}
\definecolor{myBlack}{rgb}{0,0,0}
\definecolor{myGreen}{rgb}{0.1,0.6,0.2}
\definecolor{myGray}{rgb}{0.5,0.5,0.5}
\definecolor{myLightgray}{rgb}{0.95,0.95,0.95}
\definecolor{myMauve}{rgb}{0.58,0,0.82}
\lstdefinelanguage{customc}{
    language=C,
    backgroundcolor = \color{myLightgray},
    basicstyle=\small\ttfamily\color{myBlack},
    keywordstyle=\color{myLightBlue},
    keywordstyle=[2]\color{red},
    commentstyle=\small\ttfamily\color{myGreen},
    morekeywords={RequirePackage,ProvidesPackage},
    %
    % The special highlighting works for '!if', '!endif' and '!else'
    % But it doesn't work for '#if', '#endif' and '#else'.
    alsoletter = {!},
    keywords=[2]{!if,!endif,!else},
    otherkeywords={define,include,\# }
}

\lstdefinestyle{myCustomc}{
    language = customc,
    % keywordstyle = \color{myMauve},
}

\lstset{escapechar=@,style=myCustomc}


\definecolor{grey}{rgb}{0.3,0.3,0.3}
\definecolor{lightgrey}{rgb}{0.9,0.9,0.9}

\thispagestyle{empty}
\setlength{\textwidth}{18.5cm}
\setlength{\topmargin}{-2.5cm}
\setlength{\textheight}{24.5cm}
\setlength{\oddsidemargin}{-1cm}
\setlength{\evensidemargin}{-1cm}

\begin{document}
\begin{center}{\LARGE Parziale di Programmazione I - Bioinformatica}\\
\begin{center}
  \large 31 gennaio 2022, turno delle 9:00 (tempo disponibile: 2 ore)
\end{center}
\end{center}

\vspace*{1ex}
\begin{center}{\Large Esercizio 1} ($18$ punti)\\
  \textbf{(si consegni \texttt{second.c} e \texttt{second.h})}
\end{center}

Si scriva un programma \texttt{second.c} che implementa le seguenti funzioni:

\begin{center}
\begin{lstlisting}[language=myC]
// inizializza l'array, lungo length, con numeri casuali tra 0 e 9 inclusi
void init(int array[], int length) { ... }

// stampa l'array, lungo length, su una riga e poi va a capo
void print(int array[], int length) { ... }

// restituisce l'elemento dell'array, lungo length, il cui valore
// e' il secondo piu' grande fra quelli contenuti nell'array;
// se non esistesse, restituisce -1.
// L'array non deve venire modificato
int second(int array[], int length) { ... }
\end{lstlisting}
\end{center}
%
Si scriva quindi un file di header \texttt{second.h} che dichiara le precedenti funzioni.

Per esempio, il seguente file \texttt{main.c}
(gi\`a fornito e da non modificare):

\begin{center}
  \begin{lstlisting}[language=myC]
#include <stdio.h>
#include "second.h"

int main(void) {
  int array1[8];
  init(array1, 8);
  printf("array1: ");
  print(array1, 8);
  printf("Secondo piu' grande elemento di array1: %i\n", second(array1, 8));
  int array2[] = { 8, 8, 4, 8 };
  printf("array2: ");
  print(array2, 4);
  printf("Secondo piu' grande elemento di array2: %i\n", second(array2, 4));
  int array3[] = { 8, 8, 8 };
  printf("array3: ");
  print(array3, 3);
  printf("Secondo piu' grande elemento di array3: %i\n", second(array3, 4));
  int array4[] = { };
  printf("array4: ");
  print(array4, 0);
  printf("Secondo piu' grande elemento di array4: %i\n", second(array4, 0));
  return 0;
}
  \end{lstlisting}
\end{center}
stampa qualcosa del tipo:

\begin{mdframed}[backgroundcolor=lightgrey] 
\begin{verbatim}
array1: 3 1 7 2 6 1 7 5 
Secondo piu' grande elemento di array1: 6
array2: 8 8 4 8 
Secondo piu' grande elemento di array2: 4
array3: 8 8 8 
Secondo piu' grande elemento di array3: -1
array4: 
Secondo piu' grande elemento di array4: -1
\end{verbatim}
\end{mdframed}

\mbox{}\\
\begin{center}{\Large Esercizio 2} ($13$ punti)\\
  \textbf{(si consegni \texttt{bar.c})}\end{center}
%
Si completi il seguente programma \texttt{bar.c} in modo che la funzione
\texttt{bar} stampi una riga di $n$ asterischi seguiti da
$2n$ chioccioline. \textbf{La funzione bar deve essere ricorsiva}:

\begin{center}
  \begin{lstlisting}[language=myC]
#include <stdio.h>

// funzione ricorsiva che stampa n asterischi seguiti da 2*n chioccioline
// sia assuma che n sia non negativo
void bar(int n) {
  // DA COMPLETARE, DEVE ESSERE RICORSIVA
}

int main(void) {
  // DA COMPLETARE:
  // 1) legge n>= 0 da tastiera, richiedendolo a oltranza se negativo
  // 2) chiama bar(n)
  // 3) va a capo
  return 0;
}
  \end{lstlisting}
\end{center}

Per esempio, un utilizzo del programma potrebbe essere il seguente:

\begin{mdframed}[backgroundcolor=lightgrey] 
\begin{verbatim}
Inserisci n >= 0: 5
*****@@@@@@@@@@
\end{verbatim}
\end{mdframed}

\end{document}
