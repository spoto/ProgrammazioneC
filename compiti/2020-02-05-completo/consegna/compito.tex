\documentclass[12pt]{article}
\usepackage{amssymb}
\usepackage{amsmath}
\usepackage{color}
\usepackage{graphicx}

\definecolor{grey}{rgb}{0.3,0.3,0.3}

\usepackage{listings, framed}
\lstset{
  language=Java,
  showstringspaces=false,
  columns=flexible,
  basicstyle={\small\ttfamily},
  frame=none,
  numbers=none,
  keywordstyle=\bfseries\color{grey},
  commentstyle=\itshape\color{red},
  identifierstyle=\color{black},
  stringstyle=\color{blue},
  numberstyle={\ttfamily},
%  breaklines=true,
  breakatwhitespace=true,
  tabsize=3,
  escapechar=|
}

\thispagestyle{empty}
\setlength{\textwidth}{18.5cm}
\setlength{\topmargin}{-2.5cm}
\setlength{\textheight}{24.5cm}
\setlength{\oddsidemargin}{-1cm}
\setlength{\evensidemargin}{-1cm}
\begin{document}
\begin{center}{\LARGE Esame Completo di Programmazione I - BioInformatica}\\
\vspace*{-2ex}
\begin{center}
  \large 5 febbraio 2020 (tempo disponibile: 2 ore)
\end{center}
\end{center}

\vspace*{1ex}
\begin{center}{\Large Esercizio 1} ($9$ punti)
\end{center}
Si scriva un programma \texttt{has\_local\_max.c} che implementa una funzione \texttt{int has\_local\_max(int arr[], int length, int how\_many)}. Tale funzione deve ricevere un array \texttt{arr} di interi, lungo \texttt{length}, e deve determinare se tale array ha almeno \texttt{how\_many} massimi locali. Un massimo locale \`e un elemento che \`e maggiore sia del precedente che del successivo (se esistono). Per esempio, se \texttt{arr} fosse $\{12,-1,10,7,5,6\}$, i suoi massimi locali sarebbero $12$, $10$ e $6$. Si scriva il file di header \texttt{has\_local\_max.h} in cui si dichiara tale funzione.

\vspace*{1ex}
\begin{center}{\Large Esercizio 2} ($7$ punti)\end{center}
%
Si scriva un programma \texttt{main\_local\_max.c} che include la funzione dell'Esercizio~1 tramite
il file di header \texttt{has\_local\_max.h}.
Il programma \texttt{main\_local\_max.c} deve contenere una funzione iniziale \texttt{main} che esegue
le seguenti operazioni:
\begin{enumerate}
\item legge da tastiera la lunghezza \texttt{length} di un array, richiedendola ad oltranza se fosse inserita negativa;
\item crea un array \texttt{elements} di \texttt{length} interi;
\item legge da tastiera gli elementi di tale array, uno alla volta;
\item chiama la funzione \texttt{has\_local\_max} dell'Esercizio~1, per sapere se l'array
  \texttt{elements} contiene almeno $3$ massimi locali;
\item sulla base del risultato di tale chiamata, stampa \texttt{"Ci sono almeno tre massimi locali"}
  oppure \texttt{"Ci sono meno di tre massimi locali"}.
\end{enumerate}

\vspace*{1ex}
\begin{center}{\Large Esercizio 3} ($16$ punti)\end{center}
Si scriva un programma \texttt{numeri.c} che implementa le seguenti funzioni:
\begin{itemize}
	\item \texttt{void scriviInFile (char nomeFile[])}. Tale funzione riceve come parametro il nome di un file di testo e lo crea. Inoltre la funzione richiede all'utente di inserire da tastiera una serie di numeri float che termina con l'inserimento del numero 0.0. La funzione salva i numeri inseriti nel file di testo creato, usando due cifre decimali.
	\item \texttt{void leggiDaFile (char nomeFile[])}. Tale funzione riceve come paramero il nome di un file di testo, lo apre in lettura, legge i numeri float presenti e li stampa a video nell'ordine in cui sono stati inseriti nel file.
	\item \texttt{void main()}. Tale funzione chiede all'utente l'inserimento di un nome di file di testo, richiama prima la funzione \texttt{scriviInFile} e poi la funzione \texttt{leggiDaFile} passando come parametro il nome inserito.
\end{itemize}.

\end{document}
