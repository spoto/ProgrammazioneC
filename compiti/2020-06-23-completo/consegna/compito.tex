\documentclass[12pt]{article}
\usepackage{amssymb}
\usepackage{amsmath}
\usepackage{color}
\usepackage{graphicx}

\definecolor{grey}{rgb}{0.3,0.3,0.3}

\usepackage{listings, framed}
\lstset{
  language=Java,
  showstringspaces=false,
  columns=flexible,
  basicstyle={\small\ttfamily},
  frame=none,
  numbers=none,
  keywordstyle=\bfseries\color{grey},
  commentstyle=\itshape\color{red},
  identifierstyle=\color{black},
  stringstyle=\color{blue},
  numberstyle={\ttfamily},
%  breaklines=true,
  breakatwhitespace=true,
  tabsize=3,
  escapechar=|
}

\thispagestyle{empty}
\setlength{\textwidth}{18.5cm}
\setlength{\topmargin}{-2.5cm}
\setlength{\textheight}{24.5cm}
\setlength{\oddsidemargin}{-1cm}
\setlength{\evensidemargin}{-1cm}
\begin{document}
\begin{center}{\LARGE Compito di Programmazione I - BioInformatica}\\
\vspace*{-2ex}
\begin{center}
  \large 23 giugno 2020 (tempo disponibile: 2 ore)
\end{center}
\end{center}

\vspace*{1ex}
\begin{center}{\Large Esercizio 1} ($13$ punti) \textbf{[Si consegni \texttt{spazi.c}]}\end{center}

Si completi la funzione
%
\begin{verbatim}
int spazi(char arr[], int length)
\end{verbatim}
%
dentro il file sorgente \texttt{spazi.c}. Tale funzione riceve un array di caratteri
e la sua lunghezza. La funzione deve modificare l'array in modo che i caratteri ripetuti
diventino spazi. Deve ritornare il numero totale dei caratteri ripetuti che ha trovato.
\textbf{Si faccia l'ipotesi semplificativa che l'array inizialmente non contenga spazi}.

Per esempio, l'esecuzione del \texttt{main} (gi\`a completo, non modificatelo):

\begin{verbatim}
int main(void) {
  char x[] = { 'f', 'y', 'F', 'Y', 'y', 'd', '@', 'y', 'o', '3', '@' };
  int quanti = spazi(x, 11);

  for (int pos = 0; pos < 11; pos++)
    printf("%c", x[pos]);

  printf("\nquanti = %d\n", quanti);
  return 0;
}
\end{verbatim}
%
deve stampare
%
\begin{verbatim}
f FY d  o3 
quanti = 5
\end{verbatim}

\end{document}
