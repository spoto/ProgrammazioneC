\documentclass[12pt]{article}
\usepackage{amssymb}
\usepackage{amsmath}
\usepackage{color}
\usepackage{graphicx}
\usepackage{mdframed}
\usepackage{listings, xcolor}
% \usepackage[dvipsnames]{xcolor} % Non va!
\usepackage{textcomp}

\lstdefinelanguage{myC}{
	keywords=[1]{break, case, continue, default, do
		, else, false, for, if, const, return, switch, true, while}, % generic keywords
	keywordstyle=[1]\color{blue}\bfseries,
	keywords=[2]{bool, int, long, float, double, byte, short, char, void, signed, unsigned}, % types
	keywordstyle=[2]\color{teal}\bfseries,
	keywordstyle=[2]\color{violet}\bfseries,
	keywords=[3]{NULL},
	keywordstyle=[3]\color{violet}\bfseries,
	identifierstyle=\color{black},
	sensitive=false,
	comment=[l]{//},
	morecomment=[s]{/*}{*/},
	commentstyle=\color{violet}\ttfamily,
	commentstyle=\normalsize\ttfamily\color{grey}, % scriptsize
	stringstyle=\color{forestgreen}\ttfamily,
	morestring=[b]',
	morestring=[b]"
}

\lstset{
	language=myC,
	backgroundcolor=\color{veryLightgray},
	extendedchars=true,
	basicstyle=\small\ttfamily,
	showstringspaces=false,
	showspaces=false,
	numbers=none,
	numberstyle=\normalsize,
	numbersep=9pt,
	tabsize=2,
	upquote=true,
	breaklines=true,
	showtabs=false,
	captionpos=b
}

\definecolor{myBlue}{rgb}{0.5,0.5,1}
\definecolor{myLightBlue}{rgb}{0.35,0.6,0.8}
\definecolor{myBlack}{rgb}{0,0,0}
\definecolor{myGreen}{rgb}{0.1,0.6,0.2}
\definecolor{myGray}{rgb}{0.5,0.5,0.5}
\definecolor{myLightgray}{rgb}{0.95,0.95,0.95}
\definecolor{verylightgray}{rgb}{.97,.97,.97}
\definecolor{myMauve}{rgb}{0.58,0,0.82}
\definecolor{forestgreen}{rgb}{0.13, 0.55, 0.13}

\lstdefinelanguage{customc}{
	language=C,
	backgroundcolor = \color{white},
	basicstyle=\small\ttfamily\color{myBlack},
	keywordstyle=\color{myLightBlue},
	keywordstyle=[2]\color{red},
	commentstyle=\small\ttfamily\color{forestgreen},
	morekeywords={RequirePackage,ProvidesPackage},
	%
	% The special highlighting works for '!if', '!endif' and '!else'
	% But it doesn't work for '#if', '#endif' and '#else'.
	alsoletter = {!},
	keywords=[2]{!if,!endif,!else},
}

\lstdefinestyle{myCustomc}{
	language = customc,
	% keywordstyle = \color{myMauve},
}

\lstset{escapechar=@,style=myCustomc}


\definecolor{grey}{rgb}{0.3,0.3,0.3}
\definecolor{lightgrey}{rgb}{0.9,0.9,0.9}

\thispagestyle{empty}
\setlength{\textwidth}{18.5cm}
\setlength{\topmargin}{-2.5cm}
\setlength{\textheight}{24.5cm}
\setlength{\oddsidemargin}{-1cm}
\setlength{\evensidemargin}{-1cm}

\newcommand\ddfrac[2]{\frac{\displaystyle #1}{\displaystyle #2}}

\begin{document}
	\begin{center}{\LARGE Secondo parziale di Programmazione I - Bioinformatica}\\
		%\vspace*{-1ex}
		\begin{center}
			\large 20 giugno 2023 (tempo disponibile: 2 ore)
		\end{center}
	\end{center}
	\noindent\makebox[\linewidth]{\rule{\linewidth}{0.4pt}}
\begin{center}
INDICAZIONI GENERALI
\end{center}
\begin{itemize}
\item{Utilizzare il comando \texttt{ulimit -v 500000} per limitare l'utilizzo delle risorse al terminale su cui viene eseguito il comando ed evitare spiacevoli inconvenienti dovuti ad eccessive allocazioni di memoria.}
\item{Scaricare il file di ogni esercizio e riconsegnarlo senza modificarne il nome.}
\item{I file non consegnati o consegnati con errori di compilazione non verrano presi in considerazione. }
\item{Si possono utilizzare funzioni aggiuntive non presenti nei file modello e aggiungere linee di commento alle funzioni gi\'a implementate nel modello.}
\item{I file possono essere consegnati pi\'u volte. Per ogni esercizio, solo l'ultimo file consegnato sar\'a considerato valido.}
\item{Compilare con l'opzione \texttt{-Wall} per avere tutti i warning.}
\item {È vietato utilizzare funzioni di libreria diverse da \texttt{scanf()}, \texttt{printf()}, \texttt{malloc()} e \texttt{sizeof()}.}
\end{itemize}
\noindent\makebox[\linewidth]{\rule{\linewidth}{0.4pt}}

	\newpage
	% ESERCIZIO 1
		\begin{center}
			{\Large Esercizio 1} ($16$ punti).
			\\
			\textbf{(si consegni {es1\_targhe.c})}
	\end{center}

	

Si vogliono trovare le targhe pari a partire da un insieme di targhe.
Una targa valida e' una stringa alfanumerica di 7 caratteri, in particolare due alfabetici 3 numerici e 2 alfabetici (es. \texttt{AS234NN}). 
Si completi l'implementazione del codice allegato \texttt{es1\_targhe.c}.

La funzione \texttt{leggiTarghe()} legge da stdin 10 targhe valide e le memorizza in una matrice di caratteri.

La funzione \texttt{trovaTarghePari()} riceve in input la matrice m di caratteri contenente le targhe e una matrice di caratteri vuota, e popolera' la matrice vuota con le targhe pari presenti in m, infine restituirà il numero di targhe pari trovate. 

	Completare il file \texttt{es1\_targhe.c} che ha la seguente struttura.

		\begin{lstlisting}[language=myC]
		#include <stdio.h>
		#include <string.h>
		#include <stdlib.h>
		#define M 10
		#define L 8//lunghezza standard di una targa
		
		
		/*legge le targhe da stdin e le memorizza in una matrice di caratteri*/
		void leggiTarghe(char [][L] );
		/*memorizza le targhe pari in una matrice di caratteri e resituisce il numero di targhe pari*/
		int trovaTarghePari(char [][L], char [][L] );
		int main(void){
			char targhe[M][L];
			leggiTarghe(targhe);
			
			char targhePari[M][L];	
			int k = trovaTarghePari(targhe,targhePari);
			printf("Le seguenti targhe sono pari:\n");
			for (int i=0;i<k;i++){
				printf("%s\n",targhePari[i]);
			}
			return -1;
		}
		void leggiTarghe(char targhe[M][L] ){
			 // da completare	
		}
		
		int trovaTarghePari(char targhe[][L], char targhePari [][L] ){
  		 // da completare	
		}	
		
		\end{lstlisting}

	
	
	
	
	
	
	
	\newpage
	% ESERCIZIO 2
	\vspace*{8ex}
	\begin{center}{\Large Esercizio 2} ($14$ punti)\\
		\textbf{(si consegni \texttt{es2\_elim.c})}
	\end{center}

	Completare il file \texttt{es2\_elim.c} che ha la seguente struttura
	
	\begin{lstlisting}[language=myC]
	#include <stdio.h>
	#include <stdlib.h>
	//definizione del tipo di dato per gli elementi di una lista
	typedef struct elem {
		// da completare 
	} Elem;
	
	Elem* crealista(int a[], int size);
	void printList(Elem* head);
	
	int main(void) {
		int arr[] = {2,4,4,4,4,6,6,8,8,8,10,10,10};
		int size = sizeof(arr) / sizeof(int);
		Elem* result = crealista(arr, size);
		printf("Lista senza duplicati:\n ");
		printList(result);
		return 0;
	}
	//funzione che dato un array e la sua lunghezza restituisce 
	//il puntatore al primo elemento della lista
	Elem* crealista(int a[], int size) {
		// da completare  
	}

	//funzione che dato il puntatore di testa ad una lista stampa 
	//tutti gli elementi della lista
	void printList(Elem* head) {
		// da completare  
	}
	\end{lstlisting}
	in modo che, dato un array di n  interi ordinati in modo non decrescente, si trasferiscano tutti gli elementi dell’array senza duplicazioni in una lista di interi.
	
	\bigskip
	\noindent
	Il \texttt{main()},  dato l'array\\
	\texttt{int arr[] = {2,2,2,2,2,2,2,2,4,4,4,4,6,6,8,8,8,10,10,10};}\\
	dovrà  stampare \\
	\texttt{2 4 6 8 10}
	
	
\end{document}