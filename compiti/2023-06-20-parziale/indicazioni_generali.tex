\noindent\makebox[\linewidth]{\rule{\linewidth}{0.4pt}}
\begin{center}
INDICAZIONI GENERALI
\end{center}
\begin{itemize}
\item{Utilizzare il comando \texttt{ulimit -v 500000} per limitare l'utilizzo delle risorse al terminale su cui viene eseguito il comando ed evitare spiacevoli inconvenienti dovuti ad eccessive allocazioni di memoria.}
\item{Scaricare il file di ogni esercizio e riconsegnarlo senza modificarne il nome.}
\item{I file non consegnati o consegnati con errori di compilazione non verrano presi in considerazione. }
\item{Si possono utilizzare funzioni aggiuntive non presenti nei file modello e aggiungere linee di commento alle funzioni gi\'a implementate nel modello.}
\item{I file possono essere consegnati pi\'u volte. Per ogni esercizio, solo l'ultimo file consegnato sar\'a considerato valido.}
\item{Compilare con l'opzione \texttt{-Wall} per avere tutti i warning.}
\item {È vietato utilizzare funzioni di libreria diverse da \texttt{scanf()}, \texttt{printf()}, \texttt{malloc()} e \texttt{sizeof()}.}
\end{itemize}
\noindent\makebox[\linewidth]{\rule{\linewidth}{0.4pt}}
