\documentclass[12pt]{article}
\usepackage{amssymb}
\usepackage{amsmath}
\usepackage{color}
\usepackage{graphicx}
\usepackage{mdframed}
\usepackage{listings, xcolor}
\usepackage{textcomp}

\definecolor{verylightgray}{rgb}{.97,.97,.97}

\lstdefinelanguage{myC}{
        keywords=[1]{break, case, continue, default, do
, else, false, for, if, const, return, switch, true, while}, % generic keywords
        keywordstyle=[1]\color{blue}\bfseries,
        keywords=[2]{bool, int, long, float, double, byte, short, char, void, signed, unsigned}, % types
        keywordstyle=[2]\color{teal}\bfseries,
        keywordstyle=[2]\color{violet}\bfseries,
        keywords=[3]{NULL},
        keywordstyle=[3]\color{teal}\bfseries,
        identifierstyle=\color{black},
        sensitive=false,
        comment=[l]{//},
        morecomment=[s]{/*}{*/},
        commentstyle=\color{violet}\ttfamily,
        commentstyle=\scriptsize\ttfamily\color{myGreen},
        stringstyle=\color{red}\ttfamily,
        morestring=[b]',
        morestring=[b]"
}

\lstset{
        language=myC,
        backgroundcolor=\color{verylightgray},
        extendedchars=true,
        basicstyle=\small\ttfamily,
        showstringspaces=false,
        showspaces=false,
        numbers=none,
        numberstyle=\small,
        numbersep=9pt,
        tabsize=2,
        upquote=true,
        breaklines=true,
        showtabs=false,
        captionpos=b
        otherkeywords={define,include,\# }
}

\definecolor{myBlue}{rgb}{0.5,0.5,1}
\definecolor{myLightBlue}{rgb}{0.35,0.6,0.8}
\definecolor{myBlack}{rgb}{0,0,0}
\definecolor{myGreen}{rgb}{0.1,0.6,0.2}
\definecolor{myGray}{rgb}{0.5,0.5,0.5}
\definecolor{myLightgray}{rgb}{0.95,0.95,0.95}
\definecolor{myMauve}{rgb}{0.58,0,0.82}

\lstdefinelanguage{customc}{
    language=C,
    backgroundcolor = \color{myLightgray},
    basicstyle=\scriptsize\ttfamily\color{myBlack},
    keywordstyle=\color{myLightBlue},
    keywordstyle=[2]\color{red},
    commentstyle=\scriptsize\ttfamily\color{myGreen},
    morekeywords={RequirePackage,ProvidesPackage},
    %
    % The special highlighting works for '!if', '!endif' and '!else'
    % But it doesn't work for '#if', '#endif' and '#else'.
    alsoletter = {!},
    keywords=[2]{!if,!endif,!else},
    otherkeywords={define,include,\# }
}

\lstdefinestyle{myCustomc}{
    language = customc,
    % keywordstyle = \color{myMauve},
}

\lstset{escapechar=@,style=myCustomc}


\definecolor{grey}{rgb}{0.3,0.3,0.3}
\definecolor{lightgrey}{rgb}{0.9,0.9,0.9}

\thispagestyle{empty}
\setlength{\textwidth}{18.5cm}
\setlength{\topmargin}{-2.5cm}
\setlength{\textheight}{24.5cm}
\setlength{\oddsidemargin}{-1cm}
\setlength{\evensidemargin}{-1cm}
\begin{document}
\begin{center}{\LARGE Esame Completo di Programmazione I - Bioinformatica}\\
%\vspace*{-1ex}
\begin{center}
  \large 16 settembre 2021 (tempo disponibile: 2 ore)
\end{center}
\end{center}

\vspace*{1ex}
\begin{center}{\Large Esercizio 1} ($15$ punti)\\
  \textbf{(si consegni \texttt{distinct.c})}
\end{center}
Si completi il seguente programma \texttt{distinct.c}:

\begin{center}
\begin{lstlisting}[language=myC]
// inizializza l'array lungo length con numeri casuali tra 1 e 10 inclusi
void init_random(int arr[], int length) {
  // DA COMPLETARE
}

// stampa su una riga l'array lungo length e poi va a capo
void print(int arr[], int length) {
  // DA COMPLETARE
}

// modifica l'array arr, lungo length, in modo da mettere i suoi elementi distinti
// al suo inizio; ritorna il numero di tali elementi distinti.
//
// Per esempio, se arr fosse
//
// 4 6 4 6 2 3 7 6 9 10 5 8 2 9 5 4 9 4 5 6
//
// allora dopo la chiamata a questa funzione i nove elementi distinti di arr
// finirebbero al suo inizio, in qualsiasi ordine, seguiti da qualsiasi valore.
// Per esempio l'array potrebbe diventare
//
// 3 7 10 8 2 9 4 5 6 .................
//
// (non importa cosa contengano gli undici elementi finali)
// e la funzione ritornera' 9 (numero di elementi distinti)
int only_distinct(int arr[], int length) {
  // DA COMPLETARE
}

int main(void) {
  int arr[20];
  init_random(arr, 20);
  print(arr, 20); // stampa di arr prima di chiamare only_distinct()
  int x = only_distinct(arr, 20);
  print(arr, x); // stampa degli elementi distinti di arr
  return 0;
}
\end{lstlisting}
\end{center}

\noindent
L'esecuzione del programma dovr\`a stampare qualcosa del tipo:

\begin{mdframed}[backgroundcolor=lightgrey] 
{\scriptsize\begin{verbatim}
4 6 4 6 2 3 7 6 9 10 5 8 2 9 5 4 9 4 5 6 
3 7 10 8 2 9 4 5 6
\end{verbatim}}
\end{mdframed}

\newpage
\mbox{}\\
\begin{center}{\Large Esercizio 2} ($16$ punti)\\
  \textbf{(si consegni \texttt{calcola.c})}\end{center}
Si completi il seguente programma \texttt{calcola.c}, la cui funzione \texttt{main()} calcola la somma di numeri presenti in un file il cui contenuto \`{e} una sequenza (di lunghezza ignota) di numeri interi positivi, uno per riga. I numeri sono scritti in lettere cifra per cifra, e sono terminati dalla parola ``stop''. Il file di esempio potrebbe essere
\begin{verbatim}
otto cinque nove stop
due due stop
sette zero sette stop
uno sei tre stop
\end{verbatim}
\begin{center}
\begin{lstlisting}[language=myC]
#include <stdio.h>
#include <stdlib.h>
#include <string.h>

int myerror(char *message);
int myfclose(FILE *f);
FILE * myfopen(char *name, char *mode);
int calcolaSomma(FILE* fp);
struct cifra{char lettera[8]; int valore;};
struct cifra cifre[10] = { {"zero",0} , {"uno",1} , {"due",2} , {"tre",3} ,
                            {"quattro",4} , {"cinque", 5} , {"sei",6} ,
                            {"sette",7} , {"otto",8} , {"nove",9}
};

int main()
{
    char filename[6] = "in.txt";
    FILE* fp;
    fp = myfopen(filename,"r");
    int risultato = calcolaSomma(fp);
    myfclose(fp);
    printf("La somma  di tutti i numeri risulta: %d \n",risultato);
  return 0;
}

int myerror(char *message)
{
    fputs( message, stderr );
    exit(1);
}

/**
 *  chiude lo stream su file controllando se ci sono errori,
 *  invoca myerror sia che f sia null sia che il risultato delle
 *  chiusura sia diverso da zero
 */
int myfclose(FILE *f)
{
     // DA COMPLETARE
}

/**
 *  apre lo stream su file controllando se ci sono errori,
 *  invoca myerror se f e' null
 */
FILE * myfopen(char *name, char *mode)
{
     // DA COMPLETARE
}

/**
 *  legge un file identificato da FILE * f, distingue le singole cifre
 *  e traduce, calcola il numero fino a "stop" che ne determina la fine,
 *  restituisce la somma di tutti i numeri
 */
int calcolaSomma(FILE* fp)
{
     // DA COMPLETARE
}
\end{lstlisting}
\end{center}
La sua esecuzione, con il file di esempio sopra specificato, dovr\`a stampare sul video:
\begin{mdframed}[backgroundcolor=lightgrey] 
\begin{verbatim}
La somma  di tutti i numeri risulta: 1751 
\end{verbatim}
\end{mdframed}

\mbox{}\\
Si osservi che:
\begin{itemize}
\item Si assume che i numeri siano di dimensione tale da essere rappresentabili tramite il tipo int.
\end{itemize}
\end{document}
