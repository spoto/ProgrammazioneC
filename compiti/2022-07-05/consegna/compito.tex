\documentclass[12pt]{article}
\usepackage{amssymb}
\usepackage{amsmath}
\usepackage{color}
\usepackage{graphicx}
\usepackage{mdframed}
\usepackage{listings, xcolor}
\usepackage{textcomp}

\definecolor{verylightgray}{rgb}{.97,.97,.97}

\lstdefinelanguage{myC}{
        keywords=[1]{break, case, continue, default, do
, else, false, for, if, const, return, switch, true, while}, % generic keywords
        keywordstyle=[1]\color{blue}\bfseries,
        keywords=[2]{bool, int, long, float, double, byte, short, char, void, signed, unsigned}, % types
        keywordstyle=[2]\color{teal}\bfseries,
        keywordstyle=[2]\color{violet}\bfseries,
        keywords=[3]{NULL},
        keywordstyle=[3]\color{teal}\bfseries,
        identifierstyle=\color{black},
        sensitive=false,
        comment=[l]{//},
        morecomment=[s]{/*}{*/},
        commentstyle=\color{violet}\ttfamily,
        commentstyle=\small\ttfamily\color{myGreen},
        stringstyle=\color{red}\ttfamily,
        morestring=[b]',
        morestring=[b]"
}

\lstset{
        language=myC,
        backgroundcolor=\color{verylightgray},
        extendedchars=true,
        basicstyle=\small\ttfamily,
        showstringspaces=false,
        showspaces=false,
        numbers=none,
        numberstyle=\small,
        numbersep=9pt,
        tabsize=2,
        upquote=true,
        breaklines=true,
        showtabs=false,
        captionpos=b
        otherkeywords={define,include,\# }
}

\definecolor{myBlue}{rgb}{0.5,0.5,1}
\definecolor{myLightBlue}{rgb}{0.35,0.6,0.8}
\definecolor{myBlack}{rgb}{0,0,0}
\definecolor{myGreen}{rgb}{0.1,0.6,0.2}
\definecolor{myGray}{rgb}{0.5,0.5,0.5}
\definecolor{myLightgray}{rgb}{0.95,0.95,0.95}
\definecolor{myMauve}{rgb}{0.58,0,0.82}
\lstdefinelanguage{customc}{
    language=C,
    backgroundcolor = \color{myLightgray},
    basicstyle=\small\ttfamily\color{myBlack},
    keywordstyle=\color{myLightBlue},
    keywordstyle=[2]\color{red},
    commentstyle=\small\ttfamily\color{myGreen},
    morekeywords={RequirePackage,ProvidesPackage},
    %
    % The special highlighting works for '!if', '!endif' and '!else'
    % But it doesn't work for '#if', '#endif' and '#else'.
    alsoletter = {!},
    keywords=[2]{!if,!endif,!else},
    otherkeywords={define,include,\# }
}

\lstdefinestyle{myCustomc}{
    language = customc,
    % keywordstyle = \color{myMauve},
}

\lstset{escapechar=@,style=myCustomc}


\definecolor{grey}{rgb}{0.3,0.3,0.3}
\definecolor{lightgrey}{rgb}{0.9,0.9,0.9}

\thispagestyle{empty}
\setlength{\textwidth}{18.5cm}
\setlength{\topmargin}{-2.5cm}
\setlength{\textheight}{24.5cm}
\setlength{\oddsidemargin}{-1cm}
\setlength{\evensidemargin}{-1cm}

\begin{document}
\begin{center}{\LARGE Compito di Programmazione I - Bioinformatica}\\
\begin{center}
  \large 5 luglio 2022 (tempo disponibile: 2 ore)
\end{center}
\end{center}

\vspace*{1ex}
\begin{center}{\Large Esercizio 1} ($15$ punti)\\
  \textbf{(si consegni \texttt{max\_subarray.c})}
\end{center}

Si completi il programma \texttt{max\_subarray.c} in modo che la funzione:

\begin{center}
  \begin{lstlisting}[language=myC]
    void max_subarray(int arr[], int length, int *start, int *end)
  \end{lstlisting}
\end{center}
%
identifichi un sottoarray di \texttt{arr} (cio\`e una porzione contigua di \texttt{arr})
i cui elementi abbiano somma massima. Per indicare dove si trova tale
sottoarray, la funzione scriver\`a gli estremi del sottoarray
nelle variabili puntate da \texttt{start} (incluso)
ed \texttt{end} (escluso). L'array \texttt{arr} \`e lungo \texttt{length}.
Per esempio, nel \texttt{main} gi\`a fornito con il compito:

\begin{center}
  \begin{lstlisting}[language=myC]
int main(void) {
  int arr[] = {-2, 1, -3, 4, -1, 2, 1, -5, 4};
  int start, end;
  max_subarray(arr, 9, &start, &end);
  printf("Un sottoarray di somma massima e': ");
  for (int pos = start; pos < end; pos++)
    printf("%i ", arr[pos]);
  printf("\n");
  return 0;
}
  \end{lstlisting}
\end{center}

\noindent
la funzione \texttt{sub\_array} dovr\`a identificare
che un sottoarray massimo \`e 4, -1, 2, 1 e quindi memorizzer\`a
3 dentro \texttt{start} (posizione di inizio, inclusa)
e 7 dentro \texttt{end} (posizione di fine, esclusa).
L'esecuzione di tale \texttt{main} stamper\`a quindi:

\begin{mdframed}[backgroundcolor=lightgrey] 
\begin{verbatim}
Un sottoarray di somma massima e': 4 -1 2 1
\end{verbatim}
\end{mdframed}

\mbox{}\\
\begin{center}{\Large Esercizio 2} ($16$ punti)\\
  \textbf{(si consegni \texttt{punti.c})}\end{center}
Si completi il seguente programma \texttt{punti.c}, la cui funzione \texttt{main()} :
\begin{center}
\begin{lstlisting}[language=myC]
int main() {
    struct polilinea_t *polilinea1;
    struct polilinea_t *polilinea2;

    polilinea1 = new_polilinea();
    polilinea2 = new_polilinea();

    add_punto(polilinea1, 0, 0);
    add_punto(polilinea1, 1, 0);
    add_punto(polilinea1, 1, 1);

    printf("Lunghezza polilinea 1: %.2lf\n", length(polilinea1));
    printf("Lunghezza polilinea 2: %.2lf\n", length(polilinea2));
    return 0;
}
\end{lstlisting}
\end{center}
produce il seguente ouput:
\begin{mdframed}[backgroundcolor=lightgrey] 
\begin{verbatim}
Lunghezza polilinea 1: 2.00
Lunghezza polilinea 2: 0.00
\end{verbatim}
\end{mdframed}

\mbox{}\\
Nota:
\begin{itemize}
\item Il programma va compilato con l'opzione \texttt{-lm} posta dopo \texttt{punti.c}, per utilizzare la funzione \texttt{sqrt()} della libreria \texttt{math.h}:\\
 \texttt{gcc punti.c -lm}
\end{itemize}

\end{document}
