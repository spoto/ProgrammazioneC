\documentclass[12pt]{article}
\usepackage{amssymb}
\usepackage{amsmath}
\usepackage{color}
\usepackage{graphicx}
\usepackage{mdframed}
\usepackage{listings, xcolor}
\usepackage{textcomp}

\definecolor{verylightgray}{rgb}{.97,.97,.97}

\lstdefinelanguage{myC}{
        keywords=[1]{break, case, continue, default, do
, else, false, for, if, const, return, switch, true, while}, % generic keywords
        keywordstyle=[1]\color{blue}\bfseries,
        keywords=[2]{bool, int, long, float, double, byte, short, char, void, signed, unsigned}, % types
        keywordstyle=[2]\color{teal}\bfseries,
        keywordstyle=[2]\color{violet}\bfseries,
        keywords=[3]{NULL},
        keywordstyle=[3]\color{teal}\bfseries,
        identifierstyle=\color{black},
        sensitive=false,
        comment=[l]{//},
        morecomment=[s]{/*}{*/},
        commentstyle=\color{violet}\ttfamily,
        commentstyle=\scriptsize\ttfamily\color{myGreen},
        stringstyle=\color{red}\ttfamily,
        morestring=[b]',
        morestring=[b]"
}

\lstset{
        language=myC,
        backgroundcolor=\color{verylightgray},
        extendedchars=true,
        basicstyle=\small\ttfamily,
        showstringspaces=false,
        showspaces=false,
        numbers=none,
        numberstyle=\small,
        numbersep=9pt,
        tabsize=2,
        upquote=true,
        breaklines=true,
        showtabs=false,
        captionpos=b
        otherkeywords={define,include,\# }
}

\definecolor{myBlue}{rgb}{0.5,0.5,1}
\definecolor{myLightBlue}{rgb}{0.35,0.6,0.8}
\definecolor{myBlack}{rgb}{0,0,0}
\definecolor{myGreen}{rgb}{0.1,0.6,0.2}
\definecolor{myGray}{rgb}{0.5,0.5,0.5}
\definecolor{myLightgray}{rgb}{0.95,0.95,0.95}
\definecolor{myMauve}{rgb}{0.58,0,0.82}

\lstdefinelanguage{customc}{
    language=C,
    backgroundcolor = \color{myLightgray},
    basicstyle=\scriptsize\ttfamily\color{myBlack},
    keywordstyle=\color{myLightBlue},
    keywordstyle=[2]\color{red},
    commentstyle=\scriptsize\ttfamily\color{myGreen},
    morekeywords={RequirePackage,ProvidesPackage},
    %
    % The special highlighting works for '!if', '!endif' and '!else'
    % But it doesn't work for '#if', '#endif' and '#else'.
    alsoletter = {!},
    keywords=[2]{!if,!endif,!else},
    otherkeywords={define,include,\# }
}

\lstdefinestyle{myCustomc}{
    language = customc,
    % keywordstyle = \color{myMauve},
}

\lstset{escapechar=@,style=myCustomc}


\definecolor{grey}{rgb}{0.3,0.3,0.3}
\definecolor{lightgrey}{rgb}{0.9,0.9,0.9}

\thispagestyle{empty}
\setlength{\textwidth}{18.5cm}
\setlength{\topmargin}{-2.5cm}
\setlength{\textheight}{24.5cm}
\setlength{\oddsidemargin}{-1cm}
\setlength{\evensidemargin}{-1cm}
\begin{document}
\begin{center}{\LARGE Esame Completo di Programmazione I - Bioinformatica}\\
%\vspace*{-1ex}
\begin{center}
  \large 29 giugno 2021 (tempo disponibile: 2 ore)
\end{center}
\end{center}

\vspace*{1ex}
\begin{center}{\Large Esercizio 1} ($15$ punti)\\
  \textbf{(si consegni \texttt{average.c})}
\end{center}
Si completi il seguente programma \texttt{average.c} che implementa le seguenti funzioni:

\begin{center}
\begin{lstlisting}[language=myC]
// inizializza arr, lungo length, con numeri interi casuali tra -40 e 50 inclusi
void init_random(int arr[], int length) {
  // DA COMPLETARE
}

// stampa su un'unica riga il contenuto dell'array arr, lungo length, poi va a capo
void print_int(int arr[], int length) {
  // DA COMPLETARE
}

// come sopra, ma stampa tre cifre decimali dopo la virgola
void print_double(double arr[], int length) {
  // DA COMPLETARE
}

// considera l'array arr diviso in blocchi consecutivi di step elementi
// e calcola per ciascun blocco la media dei suoi step elementi,
// scrivendola dentro result; quindi ogni elemento di result
// e' la media di tre elementi consecutivi di arr
//
// e' sempre vero che arr ha lunghezza length
// e' sempre vero che (la lunghezza di result) * step == length
// e' sempre vero che step > 0
void average(int arr[], int length, int step, double result[]) {
  // DA COMPLETARE
}

int main(void) {
  int arr[33];
  double result[11];
  init_random(arr, 33);
  print_int(arr, 33);
  // calcolo la media degli elementi di arr a gruppi di 3;
  // result ha infatti 11 elementi
  average(arr, 33, 3, result);
  print_double(result, 11);
  return 0;
}
\end{lstlisting}
\end{center}

\noindent
L'esecuzione del programma dovr\`a stampare qualcosa del tipo:

\begin{mdframed}[backgroundcolor=lightgrey] 
{\scriptsize\begin{verbatim}
15 28 41 11 18 -29 28 -25 -6 5 -2 4 2 13 -7 -16 -36 31 -29 47 -33 6 15 -34 -17 -19 -38 -2 6 -15 30 -30 16 
28.000 0.000 -1.000 2.333 2.667 -7.000 -5.000 -4.333 -24.667 -3.667 5.333 
\end{verbatim}}
\end{mdframed}

\newpage
\mbox{}\\
\begin{center}{\Large Esercizio 2} ($16$ punti)\\
  \textbf{(si consegni \texttt{canzoni.c})}\end{center}
Si completi il seguente programma \texttt{canzoni.c}, la cui funzione \texttt{main()} crea una canzone con cinque accordi, li stampa su video e poi libera la memoria:

\begin{center}
\begin{lstlisting}[language=myC]
#include <stdbool.h>
#include <stdio.h>
#include <stdlib.h>

struct accordo_t *add_accordo(struct accordo_t *head, char nota, bool minore);
void print_canzone(struct accordo_t *head);
void free_canzone(struct accordo_t *head);

struct accordo_t {
    char nota;
    bool minore;
    struct accordo_t *next;
};

// Program entry point. Output atteso: Am G D Am Em
int main() {
    struct accordo_t *song = NULL;
    song = add_accordo(song, 'A', true);
    add_accordo(song, 'G', false);
    add_accordo(song, 'D', false);
    add_accordo(song, 'A', true);
    add_accordo(song, 'E', true);
    print_canzone(song);
    printf("\n");
    free_canzone(song);
    return 0;
}

// Crea un accordo e lo aggiunge in fondo a una canzone,
// restituendo il puntatore al suo primo accordo.
struct accordo_t *add_accordo(struct accordo_t *head, char nota, bool minore) {
  // DA COMPLETARE
}

// Stampa gli accordi di una canzone
void print_canzone(struct accordo_t *head) {
  // DA COMPLETARE
}

// Dealloca gli accordi della canzone
void free_canzone(struct accordo_t *head) {
  // DA COMPLETARE
}
\end{lstlisting}
\end{center}
La sua esecuzione dovr\`a stampare sul video:
\begin{mdframed}[backgroundcolor=lightgrey] 
\begin{verbatim}
Am G D Am Em
\end{verbatim}
\end{mdframed}

\mbox{}\\
Si osservi che:
\begin{itemize}
\item Gli accordi vanno da A a G.
\item Nella stampa di un accordo minore, oltre alla lettera maiuscola, viene visualizzata una \texttt{m}.
\item Per indicare l'aggiunta di un accordo minore, la funzione
  \texttt{add\_accordo()} deve ricevere \texttt{true} come ultimo argomento.
\end{itemize}
\end{document}
