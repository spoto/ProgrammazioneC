\documentclass[12pt]{article}
\usepackage{amssymb}
\usepackage{amsmath}
\usepackage{color}
\usepackage{graphicx}

\definecolor{grey}{rgb}{0.3,0.3,0.3}

\usepackage{listings, framed}
\lstset{
  language=Java,
  showstringspaces=false,
  columns=flexible,
  basicstyle={\small\ttfamily},
  frame=none,
  numbers=none,
  keywordstyle=\bfseries\color{grey},
  commentstyle=\itshape\color{red},
  identifierstyle=\color{black},
  stringstyle=\color{blue},
  numberstyle={\ttfamily},
%  breaklines=true,
  breakatwhitespace=true,
  tabsize=3,
  escapechar=|
}

\thispagestyle{empty}
\setlength{\textwidth}{18.5cm}
\setlength{\topmargin}{-2.5cm}
\setlength{\textheight}{24.5cm}
\setlength{\oddsidemargin}{-1cm}
\setlength{\evensidemargin}{-1cm}
\begin{document}
\begin{center}{\LARGE Parziale di Programmazione I - BioInformatica}\\
\vspace*{-2ex}
\begin{center}
  \large 30 gennaio 2019 (tempo disponibile: 2 ore)
\end{center}
\end{center}

\vspace*{1ex}
\begin{center}{\Large Esercizio 1} ($16$ punti)
\end{center}
Si scriva un programma \texttt{swap.c} che implementa le seguenti tre funzioni su array di caratteri:
\begin{verbatim}
// inizializza arr, lungo length, con caratteri alfabetici
// inglesi minuscoli scelti a caso
void init_random(char arr[], int length);

// stampa su un'unica riga i caratteri nell'array arr, lungo length, poi va a capo
void print(char arr[], int length);

// scambia ogni elemento in posizione pari di arr, lungo length,
// con quello in posizione dispari che lo segue.
// Se length fosse dispari, l'ultimo elemento restera' nella sua posizione
void swap(char arr[], int length);
\end{verbatim}
%
Per esempio, se \texttt{arr} fosse \texttt{\{'f','m','d','j','a'\}}, chiamando
\texttt{swap(arr, 5)} l'array \texttt{arr} dovr\`a diventare
\texttt{\{'m','f','j','d','a'\}}.

Si scriva quindi un file di header \texttt{swap.h} che dichiara le precedenti
tre funzioni.

\begin{center}{\Large Esercizio 2} ($16$ punti)\end{center}
%
Si scriva un programma \texttt{main\_swap.c} che include le funzioni dell'Esercizio~1 tramite
il file di header \texttt{swap.h}.
Il programma \texttt{main\_swap.c} deve contenere una funzione iniziale \texttt{main} che esegue
le seguenti operazioni:
\begin{enumerate}
\item legge da tastiera la lunghezza \texttt{length} di un array, richiedendola ad oltranza se fosse inserita negativa;
\item crea un array \texttt{elements} di \texttt{length} caratteri;
\item usa la funzione \texttt{init\_random} per inizializzare casualmente
  gli elementi di \texttt{elements};
\item usa la funzione \texttt{print} per stampare \texttt{elements};
\item usa la funzione \texttt{swap} per scambiare ogni elemento di \texttt{elements} in posizione pari con quello in posizione dispari che lo segue;
\item usa la funzione \texttt{print} per stampare nuovamente \texttt{elements}.
\end{enumerate}

\end{document}
