\documentclass[12pt]{article}
\usepackage{amssymb}
\usepackage{amsmath}
\usepackage{color}
\usepackage{graphicx}

\definecolor{grey}{rgb}{0.3,0.3,0.3}

\usepackage{listings, framed}
\lstset{
  language=Java,
  showstringspaces=false,
  columns=flexible,
  basicstyle={\small\ttfamily},
  frame=none,
  numbers=none,
  keywordstyle=\bfseries\color{grey},
  commentstyle=\itshape\color{red},
  identifierstyle=\color{black},
  stringstyle=\color{blue},
  numberstyle={\ttfamily},
%  breaklines=true,
  breakatwhitespace=true,
  tabsize=3,
  escapechar=|
}

\thispagestyle{empty}
\setlength{\textwidth}{18.5cm}
\setlength{\topmargin}{-2.5cm}
\setlength{\textheight}{24.5cm}
\setlength{\oddsidemargin}{-1cm}
\setlength{\evensidemargin}{-1cm}
\begin{document}
\begin{center}{\LARGE Compito di Programmazione I - BioInformatica}\\
\vspace*{-2ex}
\begin{center}
  \large 26 settembre 2019 (tempo disponibile: 2 ore)
\end{center}
\end{center}

\vspace*{1ex}
\begin{center}{\Large Esercizio 1} ($14$ punti) \textbf{[Si consegni \texttt{esercizio1.c}]}\end{center}

Si scriva un programma C che definisce e implementa una funzione \texttt{f} che riceve come
argomenti un array di interi e la sua lunghezza. Tale funzione deve modificare l'array passato
come argomento in modo tale che ciascun suo elemento diventi:

\begin{itemize}
\item $1$ se l'elemento al quadrato \`e strettamente maggiore della somma di tutti gli elementi
  inizialmente nell'array;
\item $0$ altrimenti.
\end{itemize}

Si scriva quindi una funzione \texttt{main} che:

\begin{enumerate}
\item dichiara l'array \texttt{\{ 3, -2, 5, 10, 3, 12, -6 \}};
\item lo passa alla funzione \texttt{f};
\item stampa gli elementi dell'array dopo la chiamata della funzione \texttt{f}.
\end{enumerate}

La stampa dovrebbe essere \texttt{0 0 0 1 0 1 1}.

\vspace*{2ex}
\begin{center}{\Large Esercizio 2} ($8$ punti) \textbf{[Si consegni \texttt{esercizio2.c}]}\end{center}
Si scriva un programma C che consenta ad un utente di acquisire una password da standard input.
Dopo avere acquisito la password, il programma deve svolgere le seguenti operazioni:
\begin{enumerate}
	\item stampare la password
	\item sostituire nella password ogni lettera ``a'' con ``@''
	\item sostituire nalla password ogni lettera ``s'' con ``\$''
	\item stampare la password modificata.
\end{enumerate} 

Per esempio, se la password specificata dall'utente \`e ``password'', l'output dovr\`a essere ``p@\$\$word''.

\vspace*{2ex}
\begin{center}{\Large Esercizio 3} ($10$ punti) \textbf{[Si consegni \texttt{esercizio3.c}]}\end{center}

Si scriva un programma che permetta all'utente di inserire una parola \texttt{p} da standard input.
Una volta acquisita la parola, il programma deve svolgere le seguenti operazioni:
\begin{enumerate}
	\item calcolare il numero delle occorrenze della parola \texttt{p} nel file \texttt{data.txt}
	\item stampare il numero delle occorenze a video.
\end{enumerate} 


\end{document}
