\documentclass[12pt]{article}
\usepackage{amssymb}
\usepackage{amsmath}
\usepackage{color}
\usepackage{graphicx}

\definecolor{grey}{rgb}{0.3,0.3,0.3}

\usepackage{listings, framed}
\lstset{
  language=Java,
  showstringspaces=false,
  columns=flexible,
  basicstyle={\small\ttfamily},
  frame=none,
  numbers=none,
  keywordstyle=\bfseries\color{grey},
  commentstyle=\itshape\color{red},
  identifierstyle=\color{black},
  stringstyle=\color{blue},
  numberstyle={\ttfamily},
%  breaklines=true,
  breakatwhitespace=true,
  tabsize=3,
  escapechar=|
}

\thispagestyle{empty}
\setlength{\textwidth}{18.5cm}
\setlength{\topmargin}{-2.5cm}
\setlength{\textheight}{24.5cm}
\setlength{\oddsidemargin}{-1cm}
\setlength{\evensidemargin}{-1cm}
\begin{document}
\begin{center}{\LARGE Parziale di Programmazione I - Bioinformatica}\\
\vspace*{-2ex}
\begin{center}
  \large 1 febbraio 2021, turno delle 16:00 (tempo disponibile: 2 ore)
\end{center}
\end{center}

\vspace*{1ex}
\begin{center}{\Large Esercizio 1} ($20$ punti)\\
  \textbf{(si consegni \texttt{harshad.c} e \texttt{harshad.h})}
\end{center}
Si scriva un programma \texttt{harshad.c} che implementa le seguenti funzioni:
\begin{verbatim}
// inizializza arr, lungo length, con numeri interi lunghi casuali tra 0 a 999,
// usando srand() e rand(), facendo in modo che alla fine
// non ci siano elementi consecutivi uguali nell'array
void init_random(long arr[], int length);

// stampa su un'unica riga il contenuto dell'array arr, lungo length, poi va a capo
void print(long arr[], int length);

// determina se n e' un numero Harshad, cioe' e' positivo e divisibile
// per la somma delle proprie cifre. Per esempio, 1729 e' Harshad
// poiche' 1+7+2+9 fa 19 e 1729 e' divisibile per 19
int is_harshad(long n);

// modifica l'array, lungo length, in modo da riempirlo con i primi
// length numeri Harshad
void fill_with_harshad(long arr[], int length);
\end{verbatim}
%
Si scriva quindi un file di header \texttt{harshad.h} che dichiara le precedenti funzioni.

\mbox{}\\
\begin{center}{\Large Esercizio 2} ($12$ punti)\\
  \textbf{(si consegni \texttt{main.c})}\end{center}
%
Si scriva un programma \texttt{main.c} che include le funzioni dell'Esercizio~1 tramite
il file \texttt{harshad.h}.
Il programma \texttt{main.c} deve contenere una funzione iniziale \texttt{main} che esegue
le seguenti operazioni:
\begin{enumerate}
\item legge da tastiera la lunghezza \texttt{length} di un array, richiedendola ad oltranza se fosse inserita negativa;
\item crea un array \texttt{elements} di \texttt{length} interi lunghi;
\item chiama la funzione \texttt{init\_random} per inizializzare \texttt{elements} in modo casuale;
\item chiama la funzione \texttt{print} per stampare \texttt{elements};
\item chiama la funzione \texttt{fill\_with\_harshad} con l'array \texttt{elements} come parametro;
\item chiama la funzione \texttt{print} per stampare \texttt{elements}.
\end{enumerate}

\end{document}
