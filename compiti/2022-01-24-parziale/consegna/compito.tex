\documentclass[12pt]{article}
\usepackage{amssymb}
\usepackage{amsmath}
\usepackage{color}
\usepackage{graphicx}
\usepackage{mdframed}
\usepackage{listings, xcolor}
\usepackage{textcomp}

\definecolor{verylightgray}{rgb}{.97,.97,.97}

\lstdefinelanguage{myC}{
        keywords=[1]{break, case, continue, default, do
, else, false, for, if, const, return, switch, true, while}, % generic keywords
        keywordstyle=[1]\color{blue}\bfseries,
        keywords=[2]{bool, int, long, float, double, byte, short, char, void, signed, unsigned}, % types
        keywordstyle=[2]\color{teal}\bfseries,
        keywordstyle=[2]\color{violet}\bfseries,
        keywords=[3]{NULL},
        keywordstyle=[3]\color{teal}\bfseries,
        identifierstyle=\color{black},
        sensitive=false,
        comment=[l]{//},
        morecomment=[s]{/*}{*/},
        commentstyle=\color{violet}\ttfamily,
        commentstyle=\small\ttfamily\color{myGreen},
        stringstyle=\color{red}\ttfamily,
        morestring=[b]',
        morestring=[b]"
}

\lstset{
        language=myC,
        backgroundcolor=\color{verylightgray},
        extendedchars=true,
        basicstyle=\small\ttfamily,
        showstringspaces=false,
        showspaces=false,
        numbers=none,
        numberstyle=\small,
        numbersep=9pt,
        tabsize=2,
        upquote=true,
        breaklines=true,
        showtabs=false,
        captionpos=b
        otherkeywords={define,include,\# }
}

\definecolor{myBlue}{rgb}{0.5,0.5,1}
\definecolor{myLightBlue}{rgb}{0.35,0.6,0.8}
\definecolor{myBlack}{rgb}{0,0,0}
\definecolor{myGreen}{rgb}{0.1,0.6,0.2}
\definecolor{myGray}{rgb}{0.5,0.5,0.5}
\definecolor{myLightgray}{rgb}{0.95,0.95,0.95}
\definecolor{myMauve}{rgb}{0.58,0,0.82}
\lstdefinelanguage{customc}{
    language=C,
    backgroundcolor = \color{myLightgray},
    basicstyle=\small\ttfamily\color{myBlack},
    keywordstyle=\color{myLightBlue},
    keywordstyle=[2]\color{red},
    commentstyle=\small\ttfamily\color{myGreen},
    morekeywords={RequirePackage,ProvidesPackage},
    %
    % The special highlighting works for '!if', '!endif' and '!else'
    % But it doesn't work for '#if', '#endif' and '#else'.
    alsoletter = {!},
    keywords=[2]{!if,!endif,!else},
    otherkeywords={define,include,\# }
}

\lstdefinestyle{myCustomc}{
    language = customc,
    % keywordstyle = \color{myMauve},
}

\lstset{escapechar=@,style=myCustomc}


\definecolor{grey}{rgb}{0.3,0.3,0.3}
\definecolor{lightgrey}{rgb}{0.9,0.9,0.9}

\thispagestyle{empty}
\setlength{\textwidth}{18.5cm}
\setlength{\topmargin}{-2.5cm}
\setlength{\textheight}{24.5cm}
\setlength{\oddsidemargin}{-1cm}
\setlength{\evensidemargin}{-1cm}

\begin{document}
\begin{center}{\LARGE Parziale di Programmazione I - Bioinformatica}\\
\begin{center}
  \large 24 gennaio 2022 (tempo disponibile: 2 ore)
\end{center}
\end{center}

\vspace*{1ex}
\begin{center}{\Large Esercizio 1} ($9$ punti)
\end{center}
Si scriva un programma \texttt{sum.c} che implementa una funzione \texttt{int sum(int arr[], int length)}. Tale funzione deve ricevere un array \texttt{arr} di interi, lungo \texttt{length}, e deve restituire la somma degli elementi di \texttt{arr} che siano pi\`u grandi della somma dell'elemento che lo precede pi\`u l'elemento che lo segue. Per esempio, se \texttt{arr} fosse $\{2,8,4,1,5\}$, la funzione dovrebbe restituire $13$ (la somma di $8$ e $5$). Si scriva il file di header \texttt{sum.h} in cui si dichiara tale funzione.

\vspace*{1ex}
\begin{center}{\Large Esercizio 2} ($11$ punti)\end{center}
%
Si scriva un programma \texttt{main\_sum.c} che include la funzione dell'Esercizio~1 tramite
il file di header \texttt{sum.h}.
Il programma \texttt{main\_sum.c} deve contenere una funzione iniziale \texttt{main} che esegue
le seguenti operazioni:
\begin{enumerate}
\item legge da tastiera la lunghezza \texttt{length} di un array, richiedendola ad oltranza se fosse inserita negativa;
\item crea un array \texttt{elements} di \texttt{length} interi;
\item legge da tastiera gli elementi di tale array, uno alla volta;
\item chiama la funzione \texttt{sum} dell'Esercizio~1, passando \texttt{elements} e \texttt{length};
\item stampa sul video il risultato di tale chiamata.
\end{enumerate}

\vspace*{1ex}
\begin{center}{\Large Esercizio 3} ($12$ punti)\end{center}

Si scriva un programma \texttt{arrow.c} con una funzione \textbf{ricorsiva}
\texttt{void arrow(int size)} che
stampa su video una doppia parentesi angolare di asterischi, alta \texttt{size}.
Il programma deve avere una
una funzione iniziale \texttt{main}
che esegue le seguenti operazioni:
\begin{enumerate}
\item legge da tastiera una dimensione intera \texttt{size}, che deve essere un numero positivo dispari. Se non lo fosse, la richiede ad oltranza;
\item chiama \texttt{arrow} per stampare su video una doppia parentesi angolare di asterischi, alta \texttt{size}.
\end{enumerate}

Per esempio, se l'utente inserisse \texttt{5} come \texttt{size}, il programma dovrebbe stampare:
%
\begin{mdframed}[backgroundcolor=lightgrey] 
\begin{verbatim}
  * *
 * *
* *
 * *
  * *
\end{verbatim}
\end{mdframed}

Se invece l'utente inserisse \texttt{7} come \texttt{size}, il programma dovrebbe stampare:
%
\begin{mdframed}[backgroundcolor=lightgrey]
\begin{verbatim}
   * *
  * *
 * *
* *
 * *
  * *
   * *
\end{verbatim}
\end{mdframed}

Se invece l'utente inserisse \texttt{1} come \texttt{size}, il programma dovrebbe stampare:
%
\begin{mdframed}[backgroundcolor=lightgrey]
\begin{verbatim}
* *
\end{verbatim}
\end{mdframed}

\end{document}
