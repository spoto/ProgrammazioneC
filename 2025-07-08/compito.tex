\documentclass[12pt]{article}
\usepackage{amssymb}
\usepackage{amsmath}
\usepackage{color}
\usepackage{graphicx}
\usepackage{mdframed}
\usepackage{listings, xcolor}
\usepackage{booktabs}
% \usepackage[dvipsnames]{xcolor} % Non va!
\usepackage{textcomp}
\usepackage[table]{xcolor}
\newcommand{\gr}{\rowcolor[HTML]{EFEFEF}}
\lstdefinelanguage{myC}{
	keywords=[1]{break, case, continue, default, do
		, else, false, for, if, const, return, switch, true, while}, % generic keywords
	keywordstyle=[1]\color{blue}\bfseries,
	keywords=[2]{bool, int, long, float, double, byte, short, char, void, signed, unsigned}, % types
	keywordstyle=[2]\color{teal}\bfseries,
	keywordstyle=[2]\color{violet}\bfseries,
	keywords=[3]{NULL},
	keywordstyle=[3]\color{violet}\bfseries,
	identifierstyle=\color{black},
	sensitive=false,
	comment=[l]{//},
	morecomment=[s]{/*}{*/},
	commentstyle=\color{violet}\ttfamily,
	commentstyle=\normalsize\ttfamily\color{grey}, % scriptsize
	stringstyle=\color{forestgreen}\ttfamily,
	morestring=[b]',
	morestring=[b]"
}

\lstset{
	language=myC,
	backgroundcolor=\color{veryLightgray},
	extendedchars=true,
	basicstyle=\small\ttfamily,
	showstringspaces=false,
	showspaces=false,
	numbers=none,
	numberstyle=\normalsize,
	numbersep=9pt,
	tabsize=2,
	upquote=true,
	breaklines=true,
	showtabs=false,
	captionpos=b
}

\definecolor{myBlue}{rgb}{0.5,0.5,1}
\definecolor{myLightBlue}{rgb}{0.35,0.6,0.8}
\definecolor{myBlack}{rgb}{0,0,0}
\definecolor{myGreen}{rgb}{0.1,0.6,0.2}
\definecolor{myGray}{rgb}{0.5,0.5,0.5}
\definecolor{myLightgray}{rgb}{0.95,0.95,0.95}
\definecolor{verylightgray}{rgb}{.97,.97,.97}
\definecolor{myMauve}{rgb}{0.58,0,0.82}
\definecolor{forestgreen}{rgb}{0.13, 0.55, 0.13}
\definecolor{forestgreen}{rgb}{0.13, 0.55, 0.13}
\lstdefinelanguage{customc}{
    language=C,
    backgroundcolor = \color{myLightgray},
    basicstyle=\small\ttfamily\color{myBlack},
    keywordstyle=\color{myLightBlue},
    keywordstyle=[2]\color{red},
    commentstyle=\small\ttfamily\color{myGreen},
    morekeywords={RequirePackage,ProvidesPackage},
    %
    % The special highlighting works for '!if', '!endif' and '!else'
    % But it doesn't work for '#if', '#endif' and '#else'.
    alsoletter = {!},
    keywords=[2]{!if,!endif,!else},
    otherkeywords={define,include,\# }
}

\lstdefinestyle{myCustomc}{
    language = customc,
    % keywordstyle = \color{myMauve},
}

\lstset{escapechar=@,style=myCustomc}


\definecolor{grey}{rgb}{0.3,0.3,0.3}
\definecolor{lightgrey}{rgb}{0.9,0.9,0.9}

\thispagestyle{empty}
\setlength{\textwidth}{18.5cm}
\setlength{\topmargin}{-2.5cm}
\setlength{\textheight}{24.5cm}
\setlength{\oddsidemargin}{-1cm}
\setlength{\evensidemargin}{-1cm}

\begin{document}
\begin{center}
  {\LARGE Compito di Programmazione - Bioinformatica}\\
  {8 Luglio 2025 (tempo disponibile: 2 ore)}
\end{center}

\vspace*{1ex}
\begin{center}{\Large Esercizio 1} ($31$ punti)\\
  (\textbf{si consegnino} i file \texttt{list.h}, e \texttt{list.c} \textbf{opportunamente modificati})
\end{center}

\vspace{10pt}
\noindent
Si vuole gestire un elenco di giocatori (\emph{players}) per un torneo composto
da un numero fissato \texttt{SCORE\_MAX} di partite.
Ciascun giocatore \`e rappresentato da una struttura contenente:
un identificativo intero, una stringa \texttt{nickname} di lunghezza massima \texttt{CHAR\_MAX}
e l'elenco dei punteggi ottenuti al termine delle \texttt{SCORE\_MAX} partite disputate
(cio\`e un array di \texttt{float} di lunghezza \texttt{SCORE\_MAX}).
Il programma dovr\`a gestire i dati dei giocatori tramite una lista singolarmente concatenata
non necessariamente ordinata.

\vspace{10pt}
\begin{mdframed}[backgroundcolor=blue!20!white]
\vspace*{-0.5ex}
\textbf{La funzione \texttt{main} \`e gi\`a scritta e completa, non va modificata.}

\vspace{5pt}
\noindent
\textbf{Completare} invece i file \texttt{list.h} e \texttt{list.c} implementando le funzioni specificate. \`E consentito esclusivamente l'utilizzo delle funzioni appartenenti alle librerie gi\`a incluse nel codice fornito. 

\noindent Se ritenuto necessario, si possono aggiungere funzioni ausiliarie.
\end{mdframed}

\vspace{10pt}
\noindent
In particolare si completino le funzioni:
\begin{itemize}

\item \texttt{newPlayer()}: crea e inizializza un nuovo giocatore a partire da identificativo, \texttt{nickname} e punteggi.

\item \texttt{insertPlayer()}: inserisce un giocatore \textbf{in coda} alla lista.

\item \texttt{listLen()}: calcola la lunghezza della lista di giocatori fornita, in maniera \textbf{iterativa}.

\item \texttt{getMean()}: calcola la media di un array di punteggi fornito, lungo \texttt{SCORE\_MAX};

\item \texttt{printList()}: stampa l'elenco dei giocatori della lista fornita, nel formato: identificativo, \texttt{nickname}, array dei punteggi e punteggio medio.

\item \texttt{insertInOrder()}: inserisce un nuovo giocatore nella lista fornita, in base all'ordinamento decrescente per media dei punteggi. In questo caso si assume che la lista fornita sia gi\`a decrescente per media dei punteggi.

\item \texttt{createTopPlayersList()}: partendo dalla lista passata come argomento, crea una nuova lista contenente copie
  dei migliori giocatori, ordinata in maniera decrescente per media dei punteggi
  e contentente solo i giocatori con punteggio medio maggiore o uguale a un certo valore
  soglia \texttt{mediaTop}. \textbf{Questa funzione deve essere ricorsiva}.
  \textbf{Attenzione:} i nodi della nuova lista devono essere \textbf{copie} dei nodi originali,
  non riferimenti agli stessi elementi.

\item \texttt{destroyList()}: dealloca la memoria occupata dall'intera lista fornita.

\end{itemize}

\newpage
Se tutto \`e corretto, l'output del programma sar\`a il seguente:
%
\begin{mdframed}[backgroundcolor=verylightgray] 
\begin{verbatim}
Stampa lista inserita:
ID: 0, Nickname: FastLane, Best Scores: 7.00 8.50 9.00 8.50, Media: 8.25
ID: 1, Nickname: StackMaster, Best Scores: 6.00 7.00 7.50 7.00, Media: 6.88
ID: 2, Nickname: LoopHunter, Best Scores: 7.50 8.50 8.50 7.50, Media: 8.00
ID: 3, Nickname: NullPointer, Best Scores: 5.00 6.00 5.50 6.50, Media: 5.75
ID: 4, Nickname: ZeroLag, Best Scores: 4.00 5.00 6.00 5.50, Media: 5.12
ID: 5, Nickname: CodeSurfer, Best Scores: 9.00 8.50 9.50 8.00, Media: 8.75
ID: 6, Nickname: BugFinder, Best Scores: 3.00 6.50 7.00 6.00, Media: 5.62

Lunghezza della lista: 7

Lista dei top players con media superiore a 8.00, decrescente per media:
ID: 5, Nickname: CodeSurfer, Best Scores: 9.00 8.50 9.50 8.00, Media: 8.75
ID: 0, Nickname: FastLane, Best Scores: 7.00 8.50 9.00 8.50, Media: 8.25
ID: 2, Nickname: LoopHunter, Best Scores: 7.50 8.50 8.50 7.50, Media: 8.00

Deallocazione lista...
Deallocazione lista dei top players...
\end{verbatim}
\end{mdframed}


\end{document}
