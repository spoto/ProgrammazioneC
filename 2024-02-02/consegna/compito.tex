\documentclass[12pt]{article}
\usepackage{amssymb}
\usepackage{amsmath}
\usepackage{color}
\usepackage{graphicx}
\usepackage{mdframed}
\usepackage{listings, xcolor}
\usepackage{textcomp}

\definecolor{verylightgray}{rgb}{.97,.97,.97}
\definecolor{lightred}{rgb}{.97,.50,.50}

\lstdefinelanguage{myC}{
        keywords=[1]{break, case, continue, default, do
, else, false, for, if, const, return, switch, true, while}, % generic keywords
        keywordstyle=[1]\color{blue}\bfseries,
        keywords=[2]{bool, int, long, float, double, byte, short, char, void, signed, unsigned}, % types
        keywordstyle=[2]\color{teal}\bfseries,
        keywordstyle=[2]\color{violet}\bfseries,
        keywords=[3]{NULL},
        keywordstyle=[3]\color{teal}\bfseries,
        identifierstyle=\color{black},
        sensitive=false,
        comment=[l]{//},
        morecomment=[s]{/*}{*/},
        commentstyle=\color{violet}\ttfamily,
        commentstyle=\small\ttfamily\color{myGreen},
        stringstyle=\color{red}\ttfamily,
        morestring=[b]',
        morestring=[b]"
}

\lstset{
        language=myC,
        backgroundcolor=\color{verylightgray},
        extendedchars=true,
        basicstyle=\small\ttfamily,
        showstringspaces=false,
        showspaces=false,
        numbers=none,
        numberstyle=\small,
        numbersep=9pt,
        tabsize=2,
        upquote=true,
        breaklines=true,
        showtabs=false,
        captionpos=b
        otherkeywords={define,include,\# }
}

\definecolor{myBlue}{rgb}{0.5,0.5,1}
\definecolor{myLightBlue}{rgb}{0.35,0.6,0.8}
\definecolor{myBlack}{rgb}{0,0,0}
\definecolor{myGreen}{rgb}{0.1,0.6,0.2}
\definecolor{myGray}{rgb}{0.5,0.5,0.5}
\definecolor{myLightgray}{rgb}{0.95,0.95,0.95}
\definecolor{myMauve}{rgb}{0.58,0,0.82}
\lstdefinelanguage{customc}{
    language=C,
    backgroundcolor = \color{myLightgray},
    basicstyle=\small\ttfamily\color{myBlack},
    keywordstyle=\color{myLightBlue},
    keywordstyle=[2]\color{red},
    commentstyle=\small\ttfamily\color{myGreen},
    morekeywords={RequirePackage,ProvidesPackage},
    %
    % The special highlighting works for '!if', '!endif' and '!else'
    % But it doesn't work for '#if', '#endif' and '#else'.
    alsoletter = {!},
    keywords=[2]{!if,!endif,!else},
    otherkeywords={define,include,\# }
}

\lstdefinestyle{myCustomc}{
    language = customc,
    % keywordstyle = \color{myMauve},
}

\lstset{escapechar=@,style=myCustomc}


\definecolor{grey}{rgb}{0.3,0.3,0.3}
\definecolor{lightgrey}{rgb}{0.9,0.9,0.9}

\thispagestyle{empty}
\setlength{\textwidth}{18.5cm}
\setlength{\topmargin}{-2.5cm}
\setlength{\textheight}{24.5cm}
\setlength{\oddsidemargin}{-1cm}
\setlength{\evensidemargin}{-1cm}

\begin{document}
\begin{center}{\LARGE Compito di Programmazione I - Bioinformatica}\\
\begin{center}
  \large 2 febbraio 2024 (tempo disponibile: 2 ore)
\end{center}
\end{center}

\vspace*{1ex}
\begin{center}{\Large Esercizio 1} ($31$ punti)\\
  \textbf{(si consegni \texttt{cifre.c} e \texttt{cifre.h})}
\end{center}

Si crei il file di header \texttt{cifre.h} (2 punti) che definisce una lista di stringhe:
%
\begin{center}
  \begin{lstlisting}[language=myC]
struct list {
  char *head;
  struct list *tail;
};
  \end{lstlisting}
\end{center}
%
e che dichiara le sei funzioni riportate sotto,
che vanno invece completate dentro \texttt{cifre.c}:

\begin{center}
  \begin{lstlisting}[language=myC]
// aggiungete #include se servissero

// inizializza l'array indicato, lungo length,
// in modo che diventi una stringa lunga length-1,
// i cui caratteri siano caratteri di cifre decimali,
// casuali, tali che nell'array risultante non ci
// siano mai due caratteri consecutivi uguali
void init(char arr[], int length) { // completare, 8 punti
}

// le stringhe italiane delle dieci cifre decimali
char *digits[] = {
  "zero", "uno", "due", "tre", "quattro",
  "cinque", "sei", "sette", "otto", "nove"
};

// si assuma che s sia una stringa di caratteri di cifre decimali;
// questa funzione crea e restituisce una lista di stringhe, fatta
// dalle stringhe italiane corrispondenti ai caratteri di s; per
// esempio, se i primi due caratteri di s fossero '3' e '5', allora
// la lista risultante deve avere come primi due elementi la stringa
// "tre" e poi la stringa "cinque";
// questa funzione mette nel risultato le stringhe dell'array digits
// definito sopra, direttamente, senza farne copie;
// QUESTA FUNZIONE DEVE ESSERE RICORSIVA
struct list *create_list(char *s) { // completare, 9 punti
}

// dealloca una lista di stringhe che era stata precedentemente
// creata da create_list; QUESTA FUNZIONE DEVE ESSERE RICORSIVA
void free_list(struct list *l) { // completare, 5 punti
}

// identica alla funzione create_list, ma fa delle copie delle stringhe
// dell'array digits; QUESTA FUNZIONE DEVE ESSERE RICORSIVA
struct list *create_list2(char *s) { // completare, 4 punti
}

// identica alla funzione free_list, ma si applica a liste
// precedentemente allocate dalla funzione create_list2;
// QUESTA FUNZIONE DEVE ESSERE RICORSIVA
void free_list2(struct list *l) { // completare, 3 punti
}

// stampa una lista di stringhe, con uno spazio fra gli elementi,
// andando a capo alla fine
void print_list(struct list *l) {
  while (l) {
    printf("%s ", l -> head);
    l = l -> tail;
  }
  printf("\n");
}
  \end{lstlisting}
\end{center}

Per esempio, l'esecuzione di \texttt{main.c}:
%
\begin{center}
  \begin{lstlisting}[language=myC]
#include <stdlib.h>
#include <time.h>
#include "cifre.h"

int main(void) {
  srand(time(NULL));
  char s[31];
  init(s, 30);
  struct list *l = create_list(s);
  print_list(l);
  free_list(l);
  l = create_list2(s);
  print_list(l);
  free_list2(l);
  return 0;
}
  \end{lstlisting}
\end{center}
%
potrebbe stampare qualcosa del tipo (la stampa \`e qui accorciata per mancanza di spazio):

\begin{mdframed}[backgroundcolor=verylightgray] 
\begin{verbatim}
due quattro cinque quattro sette nove [...altre cifre...] quattro uno quattro zero 
due quattro cinque quattro sette nove [...altre cifre...] quattro uno quattro zero
\end{verbatim}
\end{mdframed}

\begin{mdframed}[backgroundcolor=lightred] 
  \textbf{Il file \texttt{main.c} \`e gi\`a scritto e completo,
    non va modificato e non va consegnato. Se servisse, si possono aggiungere funzioni ausiliarie dentro \texttt{cifre.c}.}
\end{mdframed}

\end{document}
