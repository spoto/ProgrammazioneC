\documentclass[12pt]{article}
\usepackage{amssymb}
\usepackage{amsmath}
\usepackage{color}
\usepackage{graphicx}

\definecolor{grey}{rgb}{0.3,0.3,0.3}

\usepackage{listings, framed}
\lstset{
  language=Java,
  showstringspaces=false,
  columns=flexible,
  basicstyle={\small\ttfamily},
  frame=none,
  numbers=none,
  keywordstyle=\bfseries\color{grey},
  commentstyle=\itshape\color{red},
  identifierstyle=\color{black},
  stringstyle=\color{blue},
  numberstyle={\ttfamily},
%  breaklines=true,
  breakatwhitespace=true,
  tabsize=3,
  escapechar=|
}

\thispagestyle{empty}
\setlength{\textwidth}{18.5cm}
\setlength{\topmargin}{-2.5cm}
\setlength{\textheight}{24.5cm}
\setlength{\oddsidemargin}{-1cm}
\setlength{\evensidemargin}{-1cm}
\begin{document}
\begin{center}{\LARGE Compito di Programmazione I - BioInformatica}\\
\vspace*{-2ex}
\begin{center}
  \large 25 luglio 2019 (tempo disponibile: 2 ore)
\end{center}
\end{center}

\vspace*{1ex}
\begin{center}{\Large Esercizio 1} ($14$ punti) \textbf{[Si consegni \texttt{esercizio1.c}]}\end{center}

Si scriva un programma C che definisce e implementa una funzione \texttt{f} che riceve come
argomenti un array di interi e la sua lunghezza. Tale funzione deve modificare l'array passato
come argomento nel seguente modo:

\begin{itemize}
\item ogni elemento in posizione dispari deve diventare la somma di se stesso e degli elementi
  che lo precedevano nell'array (cio\`e alla sua sinistra);
\item ogni elementi in posizione pari deve diventare 0.
\end{itemize}

Si scriva quindi una funzione \texttt{main} che:

\begin{enumerate}
\item dichiara l'array \texttt{\{ 3, -2, 5, 6, 3, 11, -5 \}};
\item lo passa alla funzione \texttt{f};
\item stampa gli elementi dell'array dopo la chiamata della funzione \texttt{f}.
\end{enumerate}

La stampa dovrebbe essere \texttt{0 1 0 12 0 26 0}.

\begin{center}{\Large Esercizio 2} ($18$ punti) \textbf{[Si consegni \texttt{esercizio2.c}]}\end{center}
Si scriva un programma C che definisce una funzione \texttt{crealista} che crea una  lista i cui elementi contengono abbreviazioni di nomi di citt\`a (ad esempio TO, MI,
RM) e i corrispondenti nomi di citt\`a estesi (ad esempio Torino, Milano, Roma). Gli elementi della lista sono rappresentati
dalla seguente struttura: 

\begin{lstlisting}
struct elem {
	char abbr[2];
	char *estesa;
	struct elem *next;
}
\end{lstlisting}

Scrivere una funzione \texttt{cerca} che riceve come parametri il puntatore all'inizio della lista, una abbreviazione di nome di citt\`a e  il corrispondente nome esteso. La funzione cerca l'abbreviazione nella lista e:

\begin{enumerate}
\item[a)] restituisce 0 se l'abbreviazione \`e presente nella lista e ad essa corrisponde
  lo stesso nome di citt\`a esteso passato alla funzione;
\item[b)] restituisce 1 se l'abbrevazione non \`e presente nella lista; in questo caso, inoltre,
  la funzione aggiunge un nuovo elemento in coda alla lista, con l'abbreviazione e il nome di citt\`a passati alla funzione;
\item[c)] restituisce 2 se l'abbreviazione \`e presente nella lista ma ad essa corrisponde un nome di citt\`a esteso differente
  da quello passato alla funzione; in questo caso, inoltre, la funzione rimpiazza il nome di citt\`a esteso presente nella lista
  con quello passato come parametro alla funzione.
\end{enumerate}

\end{document}
