\documentclass[12pt]{article}
\usepackage{amssymb}
\usepackage{amsmath}
\usepackage{color}
\usepackage{graphicx}
\usepackage{mdframed}
\usepackage{listings, xcolor}
\usepackage{textcomp}

\definecolor{verylightgray}{rgb}{.97,.97,.97}

\lstdefinelanguage{myC}{
        keywords=[1]{break, case, continue, default, do
, else, false, for, if, const, return, switch, true, while}, % generic keywords
        keywordstyle=[1]\color{blue}\bfseries,
        keywords=[2]{bool, int, long, float, double, byte, short, char, void, signed, unsigned}, % types
        keywordstyle=[2]\color{teal}\bfseries,
        keywordstyle=[2]\color{violet}\bfseries,
        keywords=[3]{NULL},
        keywordstyle=[3]\color{teal}\bfseries,
        identifierstyle=\color{black},
        sensitive=false,
        comment=[l]{//},
        morecomment=[s]{/*}{*/},
        commentstyle=\color{violet}\ttfamily,
        commentstyle=\small\ttfamily\color{myGreen},
        stringstyle=\color{red}\ttfamily,
        morestring=[b]',
        morestring=[b]"
}

\lstset{
        language=myC,
        backgroundcolor=\color{verylightgray},
        extendedchars=true,
        basicstyle=\small\ttfamily,
        showstringspaces=false,
        showspaces=false,
        numbers=none,
        numberstyle=\small,
        numbersep=9pt,
        tabsize=2,
        upquote=true,
        breaklines=true,
        showtabs=false,
        captionpos=b
        otherkeywords={define,include,\# }
}

\definecolor{myBlue}{rgb}{0.5,0.5,1}
\definecolor{myLightBlue}{rgb}{0.35,0.6,0.8}
\definecolor{myBlack}{rgb}{0,0,0}
\definecolor{myGreen}{rgb}{0.1,0.6,0.2}
\definecolor{myGray}{rgb}{0.5,0.5,0.5}
\definecolor{myLightgray}{rgb}{0.95,0.95,0.95}
\definecolor{myMauve}{rgb}{0.58,0,0.82}
\lstdefinelanguage{customc}{
    language=C,
    backgroundcolor = \color{myLightgray},
    basicstyle=\small\ttfamily\color{myBlack},
    keywordstyle=\color{myLightBlue},
    keywordstyle=[2]\color{red},
    commentstyle=\small\ttfamily\color{myGreen},
    morekeywords={RequirePackage,ProvidesPackage},
    %
    % The special highlighting works for '!if', '!endif' and '!else'
    % But it doesn't work for '#if', '#endif' and '#else'.
    alsoletter = {!},
    keywords=[2]{!if,!endif,!else},
    otherkeywords={define,include,\# }
}

\lstdefinestyle{myCustomc}{
    language = customc,
    % keywordstyle = \color{myMauve},
}

\lstset{escapechar=@,style=myCustomc}


\definecolor{grey}{rgb}{0.3,0.3,0.3}
\definecolor{lightgrey}{rgb}{0.9,0.9,0.9}

\thispagestyle{empty}
\setlength{\textwidth}{18.5cm}
\setlength{\topmargin}{-2.5cm}
\setlength{\textheight}{24.5cm}
\setlength{\oddsidemargin}{-1cm}
\setlength{\evensidemargin}{-1cm}

\begin{document}
\begin{center}{\LARGE Parziale di Programmazione I - Bioinformatica}\\
\begin{center}
  \large 7 febbraio 2022, turno delle 14:30 (tempo disponibile: 2 ore)
\end{center}
\end{center}

\vspace*{1ex}
\begin{center}{\Large Esercizio 1} ($18$ punti)\\
  \textbf{(si consegni \texttt{mediano.c} e \texttt{mediano.h})}
\end{center}

Si scriva un programma \texttt{mediano.c} che implementa le seguenti funzioni:

\begin{center}
\begin{lstlisting}[language=myC]
// inizializza l'array, lungo length, con numeri casuali,
// tutti pari, tra 0 e 100 inclusi
void init(int array[], int length) { ... }

// stampa l'array, lungo length, su una riga e poi va a capo
void print(int array[], int length) { ... }

// ritorna l'elemento dell'array che e' piu' vicino alla media
// degli elementi dell'array, supponendo che l'array abbia lunghezza length
// e che length > 0
int mediano(int array[], int length) {
  // si dia per scontato che length > 0
  ...
}
\end{lstlisting}
\end{center}
%
Si scriva quindi un file di header \texttt{mediano.h} che dichiara le precedenti funzioni.

Per esempio, il seguente file \texttt{main.c}
(gi\`a fornito e da non modificare):

\begin{center}
  \begin{lstlisting}[language=myC]
#include <stdio.h>
#include "mediano.h"

int main(void) {
  int array1[8];
  init(array1, 8);
  printf("array1: ");
  print(array1, 8);
  printf("Mediano di array1 = %i\n", mediano(array1, 8));
  int array2[] = { 14, 5, 18, 8 };
  printf("array2: ");
  print(array2, 4);
  printf("Mediano di array2 = %i\n", mediano(array2, 4));
  int array3[] = { 15, 12, 15, 19 };
  printf("array3: ");
  print(array3, 4);
  printf("Mediano di array3 = %i\n", mediano(array3, 4));
  int array4[] = { 1, 2, 3, 4, 5, 6 };
  printf("array4: ");
  print(array4, 6);
  printf("Mediano di array4 = %i\n", mediano(array4, 6));
  return 0;
}
  \end{lstlisting}
\end{center}
stampa qualcosa del tipo:

\begin{mdframed}[backgroundcolor=lightgrey] 
\begin{verbatim}
array1: 20 8 10 32 92 24 62 82 
Mediano di array1 = 32
array2: 14 5 18 8 
Mediano di array2 = 14
array3: 15 12 15 19 
Mediano di array3 = 15
array4: 1 2 3 4 5 6 
Mediano di array4 = 3
\end{verbatim}
\end{mdframed}

\mbox{}\\
\begin{center}{\Large Esercizio 2} ($13$ punti)\\
  \textbf{(si consegni \texttt{count1.c})}\end{center}
%
Si completi il seguente programma \texttt{count1.c} in modo che la funzione
\texttt{count1} conti il numero di bit $1$ nella rappresentazione in binario di $n$.
\textbf{La funzione count1 deve essere ricorsiva}:

\begin{center}
  \begin{lstlisting}[language=myC]
#include <stdio.h>

int count1(int n) {
  // si dia per scontato che n >= 0
  // DA COMPLETARE: QUESTA FUNZIONE DEVE ESSERE RICORSIVA
}

int main(void) {
  // DA COMPLETARE:
  // 1) legge un intero n >= 0, se non lo fosse lo richiede ad oltranza
  // 2) chiama count1(n) e ne stampa il risultato
  return 0;
}
  \end{lstlisting}
\end{center}

Per esempio, un utilizzo del programma potrebbe essere il seguente:

\begin{mdframed}[backgroundcolor=lightgrey] 
\begin{verbatim}
Inserisci n >= 0: 255
count1(255) = 8
\end{verbatim}
\end{mdframed}

\end{document}
