\documentclass[12pt]{article}
\usepackage{amssymb}
\usepackage{amsmath}
\usepackage{color}
\usepackage{graphicx}
\usepackage{mdframed}
\usepackage{listings, xcolor}
\usepackage{textcomp}

\definecolor{verylightgray}{rgb}{.97,.97,.97}
\definecolor{lightred}{rgb}{.97,.50,.50}

\lstdefinelanguage{myC}{
        keywords=[1]{break, case, continue, default, do
, else, false, for, if, const, return, switch, true, while}, % generic keywords
        keywordstyle=[1]\color{blue}\bfseries,
        keywords=[2]{bool, int, long, float, double, byte, short, char, void, signed, unsigned}, % types
        keywordstyle=[2]\color{teal}\bfseries,
        keywordstyle=[2]\color{violet}\bfseries,
        keywords=[3]{NULL},
        keywordstyle=[3]\color{teal}\bfseries,
        identifierstyle=\color{black},
        sensitive=false,
        comment=[l]{//},
        morecomment=[s]{/*}{*/},
        commentstyle=\color{violet}\ttfamily,
        commentstyle=\small\ttfamily\color{myGreen},
        stringstyle=\color{red}\ttfamily,
        morestring=[b]',
        morestring=[b]"
}

\lstset{
        language=myC,
        backgroundcolor=\color{verylightgray},
        extendedchars=true,
        basicstyle=\small\ttfamily,
        showstringspaces=false,
        showspaces=false,
        numbers=none,
        numberstyle=\small,
        numbersep=9pt,
        tabsize=2,
        upquote=true,
        breaklines=true,
        showtabs=false,
        captionpos=b
        otherkeywords={define,include,\# }
}

\definecolor{myBlue}{rgb}{0.5,0.5,1}
\definecolor{myLightBlue}{rgb}{0.35,0.6,0.8}
\definecolor{myBlack}{rgb}{0,0,0}
\definecolor{myGreen}{rgb}{0.1,0.6,0.2}
\definecolor{myGray}{rgb}{0.5,0.5,0.5}
\definecolor{myLightgray}{rgb}{0.95,0.95,0.95}
\definecolor{myMauve}{rgb}{0.58,0,0.82}
\lstdefinelanguage{customc}{
    language=C,
    backgroundcolor = \color{myLightgray},
    basicstyle=\small\ttfamily\color{myBlack},
    keywordstyle=\color{myLightBlue},
    keywordstyle=[2]\color{red},
    commentstyle=\small\ttfamily\color{myGreen},
    morekeywords={RequirePackage,ProvidesPackage},
    %
    % The special highlighting works for '!if', '!endif' and '!else'
    % But it doesn't work for '#if', '#endif' and '#else'.
    alsoletter = {!},
    keywords=[2]{!if,!endif,!else},
    otherkeywords={define,include,\# }
}

\lstdefinestyle{myCustomc}{
    language = customc,
    % keywordstyle = \color{myMauve},
}

\lstset{escapechar=@,style=myCustomc}


\definecolor{grey}{rgb}{0.3,0.3,0.3}
\definecolor{lightgrey}{rgb}{0.9,0.9,0.9}

\thispagestyle{empty}
\setlength{\textwidth}{18.5cm}
\setlength{\topmargin}{-2.5cm}
\setlength{\textheight}{24.5cm}
\setlength{\oddsidemargin}{-1cm}
\setlength{\evensidemargin}{-1cm}

\begin{document}
\begin{center}
  {\LARGE Compito di Programmazione - Bioinformatica}\\
  {13 settembre 2023 (tempo disponibile: 2 ore)}
\end{center}

\begin{center}
  Esercizio 1 ($15$ punti, \textbf{si consegni \texttt{parentesi.c}})
\end{center}

Una sequenza di caratteri \`e una \emph{struttura di parentesi} se e solo se:
$(1)$ \`e vuota (cio\`e ha lunghezza zero), oppure $(2)$
\`e composta da una parentesi aperta (tonda, quadra o graffa), seguita da una struttura
  di parentesi, seguita da una parentesi chiusa (corrispondente come tipo a quella aperta).
Si noti che una struttura di parentesi ha sempre lunghezza pari.
Le seguenti sono tutte strutture di parentesi:
%
\begin{mdframed}[backgroundcolor=verylightgray] 
\begin{verbatim}
((({{({()})}})))
[(({}))]

(({[{({{[]}})}]}))
\end{verbatim}
\end{mdframed}
%
(la terza \`e una sequenza vuota).
Mentre le seguenti sequenze di caratteri non lo sono:
%
\begin{mdframed}[backgroundcolor=verylightgray] 
\begin{verbatim}
{]([[({{}})]]))} (la seconda parentesi non e' chiusa dalla penultima)
{$([[({{}})]]))} (il secondo carattere non e' una parentesi)
{([[({{}})]]))}  (la terza parentesi e' di tipo diverso dalla terzultima)
[                (la prima parentesi non e' mai chiusa)
\end{verbatim}
\end{mdframed}
%
Si completi il seguente file \texttt{parentesi.c} (\textbf{dove
\texttt{struttura\_di\_parentesi} deve essere ricorsiva}):
%
\begin{center}
  \vspace*{-0.8ex}
  \begin{lstlisting}[language=myC]
// riempie arr, lungo length, con una struttura di parentesi casuale lunga
// length; si dia per scontato che length >= 0 e che length sia pari
void init_random(char arr[], int length) {} // COMPLETATE

// determina se arr, lungo length, e' una struttura di parentesi;
// si dia per scontato che length >= 0 e che length sia pari;
int struttura_di_parentesi(char arr[], int length) {} // COMPLETATE RICORS.
  \end{lstlisting}
\end{center}
%
\begin{mdframed}[backgroundcolor=lightred] 
  \vspace*{-0.5ex}
  \textbf{I file \texttt{parentesi.h} e \texttt{main\_parentesi.c} sono gi\`a scritti e completi, non vanno modificati e non vanno consegnati. Si possono aggiungere funzioni ausiliarie dentro \texttt{parentesi.c}.}
\end{mdframed}
%
Se tutto \`e corretto, un esempio di esecuzione di \texttt{main\_parentesi.c} potrebbe essere:
%
\begin{mdframed}[backgroundcolor=verylightgray] 
\begin{verbatim}
((({{({()})}}))) si'
[(({}))] si'
[((([{[{}]}])))] si'
({([(({{}}))])}) si'
[{((([{}])))}] si'
{[{{({{()}})}}]} si'
{(([[({{}})]]))} si'
 si'
[{(({(([]))}))}] si'
(({[{({{[]}})}]})) si'
(({{(][([()])]})}})) no
\end{verbatim}
\end{mdframed}

\begin{center}Esercizio 2 ($15$ punti, \textbf{si consegni {MaxLoc.c}})
\end{center}

Data una lista di interi si dice che un elemento della lista con valore intero ${n}$ è un massimo locale se non è direttamente preceduto e non è direttamente seguito da un elemento ${m\ge n}.$ 
%

Si modifichi il seguente file \texttt{MaxLoc.c} completando opportunamente le funzioni add e isLocMax.


\begin{lstlisting}[language=myC]completi
#include <stdio.h>
#include <stdlib.h>

struct Lista {
    int val;
    struct Lista * next;
};
typedef struct Lista LL;
LL * add(int, LL* );
int isLocMax(LL* , int );

//funzione che aggiunge un elemento in testa ad una lista di interi.
LL * add(int num, LL* l) {
    // ... a cura delle/i studentesse/i
}

//funzione che verifica se esiste almeno un elemento nella lista h 
//con campo valore uguale a n che e' un massimo locale.
//La funzione deve restituire 0 in caso di riposta negativa ed 1 in 
//caso di risposta positiva.

int isLocMax(LL* h, int n) {
    // ... a cura delle/i studentesse/i
}

int main(void) {
    int n;
    
    LL* l0=add(1,NULL);
    LL* l1=add(1,add(1,NULL));
    LL* l2=add(1,add(1,add(3,NULL)));
    LL* l3= add(9,add(1,add(3,add(2,add(6,NULL)))));
    n=1;
    if (isLocMax(l0, n)) {
        printf("%d e' locmax di l0 \n", n);
    } else {
        printf("%d non e' locmax di l0\n", n);
    }
    if (isLocMax(l1, n)) {
        printf("%d e' locmax di l1\n", n);
    } else {
        printf("%d non e' locmax di l1\n", n);
    }
    n = 3;
    if (isLocMax(l2, n)) {
        printf("%d e' locmax di l2\n", n);
    } else {
        printf("%d non e' locmax di l2\n", n);
    }
    if (isLocMax(l3, n)) {
        printf("%d e' locmax di l3\n", n);
    } else {
        printf("%d non e' locmax di l3 \n", n);
    }
    return 0;
}
\end{lstlisting}
\begin{mdframed}[backgroundcolor=lightred] 
  \vspace*{-0.5ex}
  \textbf{La funzione main non va modificata.}
\end{mdframed}
%
Se tutto \`e corretto, l'esecuzione di \texttt{MaxLoc.c} stamperà:

%
\begin{mdframed}[backgroundcolor=verylightgray] 
\begin{verbatim}
1 e' locmax di l0
1 non e' locmax di l1
3 e' locmax di l2
3 e' locmax di l3
\end{verbatim}
\end{mdframed}

	
	
	


\end{document}