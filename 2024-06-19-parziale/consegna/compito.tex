\documentclass[12pt]{article}
\usepackage{amssymb}
\usepackage{amsmath}
\usepackage{color}
\usepackage{graphicx}
\usepackage{mdframed}
\usepackage{listings, xcolor}
% \usepackage[dvipsnames]{xcolor} % Non va!
\usepackage{textcomp}

\lstdefinelanguage{myC}{
	keywords=[1]{break, case, continue, default, do
		, else, false, for, if, const, return, switch, true, while}, % generic keywords
	keywordstyle=[1]\color{blue}\bfseries,
	keywords=[2]{bool, int, long, float, double, byte, short, char, void, signed, unsigned}, % types
	keywordstyle=[2]\color{teal}\bfseries,
	keywordstyle=[2]\color{violet}\bfseries,
	keywords=[3]{NULL},
	keywordstyle=[3]\color{violet}\bfseries,
	identifierstyle=\color{black},
	sensitive=false,
	comment=[l]{//},
	morecomment=[s]{/*}{*/},
	commentstyle=\color{violet}\ttfamily,
	commentstyle=\normalsize\ttfamily\color{grey}, % scriptsize
	stringstyle=\color{forestgreen}\ttfamily,
	morestring=[b]',
	morestring=[b]"
}

\lstset{
	language=myC,
	backgroundcolor=\color{veryLightgray},
	extendedchars=true,
	basicstyle=\small\ttfamily,
	showstringspaces=false,
	showspaces=false,
	numbers=none,
	numberstyle=\normalsize,
	numbersep=9pt,
	tabsize=2,
	upquote=true,
	breaklines=true,
	showtabs=false,
	captionpos=b
}

\definecolor{myBlue}{rgb}{0.5,0.5,1}
\definecolor{myLightBlue}{rgb}{0.35,0.6,0.8}
\definecolor{myBlack}{rgb}{0,0,0}
\definecolor{myGreen}{rgb}{0.1,0.6,0.2}
\definecolor{myGray}{rgb}{0.5,0.5,0.5}
\definecolor{myLightgray}{rgb}{0.95,0.95,0.95}
\definecolor{verylightgray}{rgb}{.97,.97,.97}
\definecolor{myMauve}{rgb}{0.58,0,0.82}
\definecolor{forestgreen}{rgb}{0.13, 0.55, 0.13}
\lstdefinelanguage{customc}{
    language=C,
    backgroundcolor = \color{myLightgray},
    basicstyle=\small\ttfamily\color{myBlack},
    keywordstyle=\color{myLightBlue},
    keywordstyle=[2]\color{red},
    commentstyle=\small\ttfamily\color{myGreen},
    morekeywords={RequirePackage,ProvidesPackage},
    %
    % The special highlighting works for '!if', '!endif' and '!else'
    % But it doesn't work for '#if', '#endif' and '#else'.
    alsoletter = {!},
    keywords=[2]{!if,!endif,!else},
    otherkeywords={define,include,\# }
}

\lstdefinestyle{myCustomc}{
    language = customc,
    % keywordstyle = \color{myMauve},
}

\lstset{escapechar=@,style=myCustomc}


\definecolor{grey}{rgb}{0.3,0.3,0.3}
\definecolor{lightgrey}{rgb}{0.9,0.9,0.9}

\thispagestyle{empty}
\setlength{\textwidth}{18.5cm}
\setlength{\topmargin}{-2.5cm}
\setlength{\textheight}{24.5cm}
\setlength{\oddsidemargin}{-1cm}
\setlength{\evensidemargin}{-1cm}

\begin{document}
\begin{center}
  {\LARGE Compito di Programmazione - Bioinformatica}\\
  {19 giugno 2024 (tempo disponibile: 2 ore)}
\end{center}

\begin{center}
  Esercizio 1 ($22$ punti, \textbf{si consegni \texttt{dipendenti.c} opportunamente modificato})
\end{center}
Si costruisca una lista concatenata di dipendenti caratterizzati ciascuno da codice fiscale e stipendio. Il codice fiscale e' una stringa alfanumerica formata da 16 caratteri.
Esempio: CZZVTR\textbf{81}S\textbf{51}D086N.
I primi due caratteri in grassetto rappresentano l'anno di nascita, 81 per il  codice fiscale nell'esempio.
Il secondo gruppo di caratteri in grassetto invece indica il giorno di nascita e il genere.
In particolare il secondo gruppo di caratteri è pari al giorno di nascita per un uomo, al giorno di nascita più 40 per una donna. Ad esempio 51 indica una donna nata giorno 11; 10 un uomo nato giorno 10.

Si completi il file dipendenti.c di seguito riportato:
%
\begin{mdframed}[backgroundcolor=myLightBlue] 
  \vspace*{-0.5ex}
  \textbf{La funzione main e' gi\`a scritta e completa, non va modificata. Si possono (e si suggerisce di) aggiungere funzioni ausiliarie dentro \texttt{dipendenti.c}.}
\end{mdframed}
%
\begin{lstlisting}[language=myC]
#include <stdio.h>
struct entry {
	  //da completare
};

struct entry *construct_entry(char fiscal_code[], float income,
  struct entry *next) {
	//da completare
}

int list_length(struct entry *l) {
	//da completare
}
float list_sum(struct entry *l) {
	//da completare
}

void print_list(struct entry *l) {
 	//da completare
}

void destroy_list(struct entry *l) {
 	//da completare
}

struct entry *senior_women_manager(struct entry *l, float avg_income) {
	//da completare
}

int main(void) {
	 struct entry *l = NULL;
      l = construct_entry( "czzvtr81s51d086n", 1300.0, construct_entry( "czzvtr61s51d086n", 3300, construct_entry( "vnrmzz93s42d086n", 1100.0, construct_entry( "abclmn00s71d086n", 3000.0, construct_entry( "mrtmtt93d48f205j", 3500.0, construct_entry( "rmqxss61s01d086n", 2500.0, l))))));
	  
	 printf("La lista e' lunga %d\n", list_length(l));
	 printf("I dipendenti sono:\n");
	 print_list(l);
	 
	 float avg_income= list_length(l)!=0? list_sum(l)/list_length(l): 0;
	 printf("La media degli elementi e' %2.f\n", avg_income);
	 printf("I dipendenti senior e con stipendio maggiore della media sono:\n");
	  struct entry * swm = senior_women_manager(l, avg_income);
	  print_list(swm);
	  destroy_list(l);
	  destroy_list(swm);
  return 0;
}
\end{lstlisting}
In particolare si completino le funzioni:
\begin{itemize}
\item construct$\_$entry  che crea e inizializza una nodo della lista dipendenti;
\item list$\_$length che calcola la lunghezza della lista.
\item list$\_$sum che prende in input una lista e restituisce la somma degli stipendi di tutti i dipendenti.
\item senior$\_$women$\_$manager che restituisce una lista, a partire da una lista data, che contiene solo i codici fiscali e gli stipendi delle donne che sono nate prima del 94 e che hanno degli stipendi sopra la media.
\end{itemize}

Le ultime tre funzioni indicate vanno implementate in modo \textbf{ricorsivo}.
Se tutto \`e corretto, un esempio di esecuzione di \texttt{dipendenti.c} con la lista come inizializzata nel main \`e:
%
\begin{mdframed}[backgroundcolor=verylightgray] 
\begin{verbatim}
La lista e' lunga 6
I dipendenti sono:
czzvtr81s51d086n: 1300
czzvtr61s51d086n: 3300
vnrmzz93s42d086n: 1100
abclmn00s71d086n: 3000
mrtmtt93d48f205j: 3500
rmqxss61s01d086n: 2500

La media degli elementi e' 2450
I dipendenti senior e con stipendio maggiore della media sono:
czzvtr61s51d086n: 3300
mrtmtt93d48f205j: 3500
\end{verbatim}
\end{mdframed}

\begin{center}Esercizio 2 ($8$ punti, \textbf{si consegni {sottostringa.c} opportunamente modificato})
\end{center}

Data una stringa s1 e una stringa s2, vogliamo verificare se s2 e' sottostringa di s1.
%
Si modifichi il seguente file \texttt{sottostringa.c} completando opportunamente la funzione isSubstring:

\begin{lstlisting}[language=myC]
#include <stdio.h>

int isSubstring(char*b, char *a){
	//da completare
}
int main(void){
	printf("%s%s e' sottostringa di %s\n", "am", isSubstring( "am","amici")==0? " non": "", "amici");
	printf("%s%s e' sottostringa di %s\n", "pia", isSubstring("pia", "amici")==0? " non": "", "amici");
	printf("%s%s e' sottostringa di %s\n", "riva", isSubstring("riva", "arrivare")==0? " 	non": "", "arrivare");
	return 0;
}

\end{lstlisting}
L'implementazione della funzione deve utilizzare \textbf{i puntatori e l'aritmetica dei puntatori} per accedere alle stringhe.
\begin{mdframed}[backgroundcolor=myLightBlue] 
  \vspace*{-0.5ex}
  \textbf{La funzione main non va modificata.}
\end{mdframed}
%
Se tutto \`e corretto, l'esecuzione di \texttt{sottostringa.c} stamperà:

%
\begin{mdframed}[backgroundcolor=verylightgray] 
\begin{verbatim}
am e' sottostringa di amici
pia non e' sottostringa di amici
riva e' sottostringa di arrivare
\end{verbatim}
\end{mdframed}

	
	
	


\end{document}