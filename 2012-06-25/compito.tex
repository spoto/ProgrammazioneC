\documentclass{article}[10pt]
\usepackage[pdftex]{graphicx}
\usepackage{amsfonts}
\usepackage[italian]{babel}
%****************enlarge layout
\textheight     243.5mm
\topmargin      -20.0mm
\textwidth      480pt
\hoffset        -80pt
%*****************theorems and such
\newcounter{esnu}
\newenvironment{esercizio}{\medskip \noindent {\bf Esercizio\addtocounter{esnu}{1} \arabic{esnu}}}{}
\pagestyle{empty}
\newcommand{\liff}{\mathrel{\leftrightarrow}}   % Logical IFF Symbol
\newcommand{\metaiff}{\Longleftrightarrow}      %iff in metatheory

\begin{document}

\begin{tabular}{llclcr}
 \hspace{-35pt} &{\bf COGNOME:} & \hspace{100pt}        &{\bf NOME:}    & \hspace{100pt}        &{\bf MATRICOLA:}\hspace{35pt} \\
\hline
\end{tabular}
\begin{center} {\bf Esame di Programmazione I, 25 giugno 2012. 2 ore}\end{center}

\begin{esercizio}
\textbf{[11 punti]}
Si scriva una funzione \texttt{vocali\_consonanti} che riceve come parametro una stringa
e non restituisce niente. La stringa viene per\`o modificata dalla funzione, in modo da
spostare all'inizio le vocali (a, e, i, o, u) e alla fine le consonanti. Si scriva quindi
una fuzione \texttt{main} che legge una stringa \texttt{s} da tastiera (di massimo 100 caratteri),
la stampa sul video, chiama \texttt{vocali\_consonanti(s)}, ristampa \texttt{s} e termina.

Se tutto \`e corretto, una possibile esecuzione di tale programma sarebbe:
%
\begin{verbatim}
inserisci la stringa: PrecipiTevoLissimevolmentE!
prima: PrecipiTevoLissimevolmentE!
dopo: eiieoiieoeEPrcpTvLssmvlmnt!
\end{verbatim}
%
\end{esercizio}

\begin{esercizio}
\textbf{[10 punti]}
Si considerino le liste viste a lezione. Si implementi una funzione ricorsiva:
\begin{verbatim}
struct list *alterna(struct list *this, int pari);
\end{verbatim}
che restituisce un puntatore a una lista $l$. Se \texttt{pari} \`e vero, la lista $l$
deve contenere tutti e soli gli elementi in
\underline{posizione} pari di \texttt{this}; se invece \texttt{pari} \`e falso, $l$ deve contenere tutti e soli
gli elementi in \underline{posizione} dispari di \texttt{this}. La lista \texttt{this} non deve venire
modificata.

Si scriva quindi un file \texttt{main\_list.c} con una funzione \texttt{main} che crea la lista
\texttt{[11, -5, 7, 13, 17, 19, -1]}; la stampa; chiama \texttt{alterna} su tale lista in modo da selezionarne
gli elementi in posizione dispari e stampa il risultato; chiama \texttt{alterna} sulla stessa lista in modo
da selezionarne gli elementi in posizione pari e stampa il risultato. Se tutto \`e corretto, l'esecuzione
di tale \texttt{main} dovrebbe stampare qualcosa del tipo:
%
\begin{verbatim}
La lista originale: [11, -5, 7, 13, 17, 19, -1]
I suoi elementi in posizione dispari: [-5, 13, 19]
I suoi elementi in posizione pari: [11, 7, 17, -1]
\end{verbatim}
%
\end{esercizio}
%
\begin{esercizio}
\textbf{[11 punti]}
Si consideri un gioco in cui un \emph{serpente} \`e al centro di un quadrato di gioco di dimensione
$21$x$21$ caselle. Inizialmente, solo la testa del serpente esiste, alla posizione di
ascissa e ordinata $10$: se la \texttt{X} \`e la testa del serpente, possiamo rappresentare questa
situazione come segue:
%

{\scriptsize
\begin{verbatim}
*********************
*                   *
*                   *
*                   *
*                   *
*                   *
*                   *
*                   *
*                   *
*                   *
*         X         *
*                   *
*                   *
*                   *
*                   *
*                   *
*                   *
*                   *
*                   *
*                   *
*********************
\end{verbatim}}

\noindent
Se a questo punto si sposta il serpente di tre posizioni a sinistra e di due in alto, si
ottiene:

\newpage
{\scriptsize
\begin{verbatim}
*********************
*                   *
*                   *
*                   *
*                   *
*                   *
*                   *
*                   *
*      X            *
*      *            *
*      ****         *
*                   *
*                   *
*                   *
*                   *
*                   *
*                   *
*                   *
*                   *
*                   *
*********************
\end{verbatim}}

\noindent
Si noti che il serpente \emph{lascia una scia} quando si muove.
Se spostiamo ulteriormente il serpente di due posizioni a destra e di una in basso,
otteniamo:

{\scriptsize
\begin{verbatim}
*********************
*                   *
*                   *
*                   *
*                   *
*                   *
*                   *
*                   *
*      ***          *
*      * X          *
*      ****         *
*                   *
*                   *
*                   *
*                   *
*                   *
*                   *
*                   *
*                   *
*                   *
*********************
\end{verbatim}}

\noindent
Si noti che, da questa posizione, il serpente non pu\`o spostarsi n\'e in basso
n\'e in alto, poich\'e
non permettiamo di urtare la propria scia o i bordi del quadrato di gioco.

Si scrivano i file \texttt{snake.h} e \texttt{snake.c} che implementano il gioco del serpente.
In particolare, devono essere definite e implementate le seguenti funzioni:
%
\begin{itemize}
\item \texttt{construct\_snake()}, che restituisce un nuovo gioco con un serpente
      posizionato nella posizioni iniziale di ascissa e ordinata $10$;
\item \texttt{destruct\_snake(struct snake *this)}, che dealloca il gioco del serpente indicato;
\item \texttt{goLeft(struct snake *this)}, che sposta il serpente a sinistra di una casella.
      Deve ritornare vero se e solo se \`e possibile lo spostamento, cio\`e non si
      urta la propria scia n\'e i bordi del gioco;
\item \texttt{goRight, goUp, goDown}, simili a \texttt{goLeft};
\item \texttt{toString(struct snake *this)}, che restituisce una nuova stringa
      contenente una descrizione dello stato attuale del gioco, come negli esempi
      stampati sopra.
\end{itemize}
%
\end{esercizio}
%
\end{document}
