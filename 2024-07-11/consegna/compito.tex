\documentclass[12pt]{article}
\usepackage{amssymb}
\usepackage{amsmath}
\usepackage{color}
\usepackage{graphicx}
\usepackage{mdframed}
\usepackage{listings, xcolor}
\usepackage{textcomp}

\definecolor{verylightgray}{rgb}{.97,.97,.97}
\definecolor{lightred}{rgb}{.97,.50,.50}

\lstdefinelanguage{myC}{
        keywords=[1]{break, case, continue, default, do
, else, false, for, if, const, return, switch, true, while}, % generic keywords
        keywordstyle=[1]\color{blue}\bfseries,
        keywords=[2]{bool, int, long, float, double, byte, short, char, void, signed, unsigned}, % types
        keywordstyle=[2]\color{teal}\bfseries,
        keywordstyle=[2]\color{violet}\bfseries,
        keywords=[3]{NULL},
        keywordstyle=[3]\color{teal}\bfseries,
        identifierstyle=\color{black},
        sensitive=false,
        comment=[l]{//},
        morecomment=[s]{/*}{*/},
        commentstyle=\color{violet}\ttfamily,
        commentstyle=\small\ttfamily\color{myGreen},
        stringstyle=\color{red}\ttfamily,
        morestring=[b]',
        morestring=[b]"
}

\lstset{
        language=myC,
        backgroundcolor=\color{verylightgray},
        extendedchars=true,
        basicstyle=\small\ttfamily,
        showstringspaces=false,
        showspaces=false,
        numbers=none,
        numberstyle=\small,
        numbersep=9pt,
        tabsize=2,
        upquote=true,
        breaklines=true,
        showtabs=false,
        captionpos=b
        otherkeywords={define,include,\# }
}

\definecolor{myBlue}{rgb}{0.5,0.5,1}
\definecolor{myLightBlue}{rgb}{0.35,0.6,0.8}
\definecolor{myBlack}{rgb}{0,0,0}
\definecolor{myGreen}{rgb}{0.1,0.6,0.2}
\definecolor{myGray}{rgb}{0.5,0.5,0.5}
\definecolor{myLightgray}{rgb}{0.95,0.95,0.95}
\definecolor{myMauve}{rgb}{0.58,0,0.82}
\lstdefinelanguage{customc}{
    language=C,
    backgroundcolor = \color{myLightgray},
    basicstyle=\small\ttfamily\color{myBlack},
    keywordstyle=\color{myLightBlue},
    keywordstyle=[2]\color{red},
    commentstyle=\small\ttfamily\color{myGreen},
    morekeywords={RequirePackage,ProvidesPackage},
    %
    % The special highlighting works for '!if', '!endif' and '!else'
    % But it doesn't work for '#if', '#endif' and '#else'.
    alsoletter = {!},
    keywords=[2]{!if,!endif,!else},
    otherkeywords={define,include,\# }
}

\lstdefinestyle{myCustomc}{
    language = customc,
    % keywordstyle = \color{myMauve},
}

\lstset{escapechar=@,style=myCustomc}


\definecolor{grey}{rgb}{0.3,0.3,0.3}
\definecolor{lightgrey}{rgb}{0.9,0.9,0.9}

\thispagestyle{empty}
\setlength{\textwidth}{18.5cm}
\setlength{\topmargin}{-2.5cm}
\setlength{\textheight}{24.5cm}
\setlength{\oddsidemargin}{-1cm}
\setlength{\evensidemargin}{-1cm}

\begin{document}
\begin{center}{\LARGE Compito di Programmazione I - Bioinformatica}\\
\begin{center}
  \large 11 luglio 2024 (tempo disponibile: 2 ore)
\end{center}
\end{center}

\vspace*{1ex}
\begin{center}{\Large Esercizio 1} ($12$ punti)\\
  \textbf{(si consegni \texttt{prefix.c} e \texttt{prefix.h})}
\end{center}

Si completi \texttt{prefix.c} in modo che la funzione \texttt{prefix(s1,s2)} determini
se la stringa \texttt{s1} \`e un prefisso della stringa \texttt{s2}
(cio\`e se \texttt{s2} comincia con \texttt{s1}).
Tale funzione deve essere ricorsiva.
Si scriva quindi il file di header \texttt{prefix.h} per l'esportazione di tale funzione.

Se tutto \`e corretto, compilando insieme a \texttt{main\_prefix.c} (gi\`a scritto, da non
modificare) ed ese\-guendo il risultato, verr\`a stampato:

\begin{mdframed}[backgroundcolor=verylightgray] 
\begin{verbatim}
prefix("anno","annata"): falso
prefix("anno","annoso"): vero
prefix("annata","annoso"): falso
prefix("annata","anno"): falso
prefix("annoso","annoso"): vero
prefix("","annata"): vero
prefix("annata",""): falso
\end{verbatim}
\end{mdframed}

\vspace*{1ex}
\begin{center}{\Large Esercizio 2} ($19$ punti)\\
  \textbf{(si consegni \texttt{panopto.c})}
\end{center}
Un corso e' caratterizzato da un codice univoco alfanumerico di 5 caratteri, il numero di CFU, la data di inizio corso.
Il primo carattere del codice corso rappresenta il settore del corso, quindi ad esempio per il corso "BIO03", la B indica che fa parte del gruppo dei corsi di biologia.

Si costruisca una catalogo di corsi, rappresentato come lista concatenata, completando opportunamente il seguente programma.


\begin{mdframed}[backgroundcolor=myLightBlue] 
  \vspace*{-0.5ex}
  \textbf{La funzione main e' gi\`a scritta e completa, non va modificata. Si possono aggiungere funzioni ausiliarie dentro \texttt{panopto.c}.}
\end{mdframed}
%
\begin{lstlisting}[language=myC]
#include <stdio.h>
#include <stdlib.h>
#include <string.h>

struct data {
	int giorno;
    int mese;
    int anno;
};

struct corso {
   char codice_corso[6];
   int CFU;
   struct data data_inizio;
   struct corso *next;
};

struct corso *construct_entry(char codice_corso[], int CFU, struct data d,
  struct corso *next) {
//da completare
}

int list_length(struct corso *l) {
//da completare
}

void print_list(struct corso *l) {
//da completare
}

void destroy_list(struct corso *l) {	
//da completare
}

int valid_study_plan (struct corso *l, int anno){
//da completare
}

struct corso * create_by_sector(struct corso *l, char s){
//da completare
}

struct corso * add_from_input (struct corso *l) {
//da completare	
}

int main(void) {
    struct corso *l = NULL;
    l = construct_entry( "BI006", 10, (struct data){2,3,2024}, l);
    l = construct_entry( "INF04", 10,(struct data) {2,3,2024}, l);
    l = construct_entry("INF03",8,(struct data){11,11,2023}, l);
    l = construct_entry( "BI005", 12,(struct data) {1,3,2024}, l);
    l = construct_entry( "INF02", 12,(struct data) {12,4,2023}, l);
    l = construct_entry("INF01",8,(struct data){11,6,2023}, l);
    l = construct_entry( "BI004", 12,(struct data) {21,5,2024}, l);
    l = construct_entry( "BI003", 12,(struct data) {21,5,2024}, l);
    l = construct_entry( "BI002", 12,(struct data) {21,5,2024}, l);
    l = construct_entry( "BI001", 12,(struct data) {21,5,2024}, l);
    l = add_from_input(l);
	 
    printf("La lista di corsi e' lunga %d\n", list_length(l));
    printf("I corsi sono:\n");
    print_list(l);
    printf("Piano di studio valido: %d\n", valid_study_plan (l, 2023));
    printf("Piano di studio valido: %d\n", valid_study_plan (l, 2024));
    struct corso * sector_plan=create_by_sector(l,'B');
    print_list(sector_plan);
    destroy_list(sector_plan);
    destroy_list(l);
    return 0;
}
\end{lstlisting}
In particolare si completino le funzioni:

\begin{itemize}
\item construct$\_$entry  che crea (in testa) e inizializza un nodo della lista di corsi;
\item add\_from\_input che riceve in input una lista e aggiunge 5 corsi da input, verificando la correttezza dei valori inseriti. I CFU devono essere al più 12, mentre l'anno di inizio corso deve essere compreso tra il 1990 ed il 2025. Un esempio di dati di test è contenuto nel file \texttt{corsi.txt} allegato.
\item list$\_$length che calcola la lunghezza della lista.
\item print$\_$list che stampa la lista.
\item destroy$\_$list che cancella la lista.
\item valid$\_$study$\_$plan che riceve in input una lista e un anno, e verifica se la somma dei CFU dei corsi iniziati nell'anno dato e' almeno 60.
\item create$\_$by$\_$sector che riceve in input una lista e un settore, e crea una lista contenente solo i corsi di quel settore.  create$\_$by$\_$sector NON deve essere implementata in modo ricorsivo.

\end{itemize}

Se tutto \`e corretto, un esempio di esecuzione di \texttt{panopto.c} con la lista come inizializzata nel main e con i corsi inclusi in \texttt{corsi.txt} \`e:
\begin{mdframed}[backgroundcolor=verylightgray] 
\begin{verbatim}
La lista di corsi e' lunga 15
I corsi sono:
anno 2022: MAT09: 11 CFU
anno 2022: MAT10: 10 CFU
anno 2022: MAT11: 11 CFU
anno 2022: FIS01: 6 CFU
anno 2022: MAT12: 12 CFU
anno 2024: BI001: 12 CFU
anno 2024: BI002: 12 CFU
anno 2024: BI003: 12 CFU
anno 2024: BI004: 12 CFU
anno 2023: INF01: 8 CFU
anno 2023: INF02: 12 CFU
anno 2024: BI005: 12 CFU
anno 2023: INF03: 8 CFU
anno 2024: INF04: 10 CFU
anno 2024: BI006: 10 CFU
totale cfu 28
Piano di studio valido: 0
totale cfu 80
Piano di studio valido: 1
anno 2024: BI001: 12 CFU
anno 2024: BI002: 12 CFU
anno 2024: BI003: 12 CFU
anno 2024: BI004: 12 CFU
anno 2024: BI005: 12 CFU
anno 2024: BI006: 10 CFU
\end{verbatim}
\end{mdframed}


\end{document}