\documentclass[12pt]{article}
\usepackage{amssymb}
\usepackage{amsmath}
\usepackage{color}
\usepackage{graphicx}
\usepackage{mdframed}
\usepackage{listings, xcolor}
\usepackage{textcomp}

\definecolor{verylightgray}{rgb}{.97,.97,.97}
\definecolor{lightred}{rgb}{.97,.50,.50}

\lstdefinelanguage{myC}{
        keywords=[1]{break, case, continue, default, do
, else, false, for, if, const, return, switch, true, while}, % generic keywords
        keywordstyle=[1]\color{blue}\bfseries,
        keywords=[2]{bool, int, long, float, double, byte, short, char, void, signed, unsigned}, % types
        keywordstyle=[2]\color{teal}\bfseries,
        keywordstyle=[2]\color{violet}\bfseries,
        keywords=[3]{NULL},
        keywordstyle=[3]\color{teal}\bfseries,
        identifierstyle=\color{black},
        sensitive=false,
        comment=[l]{//},
        morecomment=[s]{/*}{*/},
        commentstyle=\color{violet}\ttfamily,
        commentstyle=\small\ttfamily\color{myGreen},
        stringstyle=\color{red}\ttfamily,
        morestring=[b]',
        morestring=[b]"
}

\lstset{
        language=myC,
        backgroundcolor=\color{verylightgray},
        extendedchars=true,
        basicstyle=\small\ttfamily,
        showstringspaces=false,
        showspaces=false,
        numbers=none,
        numberstyle=\small,
        numbersep=9pt,
        tabsize=2,
        upquote=true,
        breaklines=true,
        showtabs=false,
        captionpos=b
        otherkeywords={define,include,\# }
}

\definecolor{myBlue}{rgb}{0.5,0.5,1}
\definecolor{myLightBlue}{rgb}{0.35,0.6,0.8}
\definecolor{myBlack}{rgb}{0,0,0}
\definecolor{myGreen}{rgb}{0.1,0.6,0.2}
\definecolor{myGray}{rgb}{0.5,0.5,0.5}
\definecolor{myLightgray}{rgb}{0.95,0.95,0.95}
\definecolor{myMauve}{rgb}{0.58,0,0.82}
\lstdefinelanguage{customc}{
    language=C,
    backgroundcolor = \color{myLightgray},
    basicstyle=\small\ttfamily\color{myBlack},
    keywordstyle=\color{myLightBlue},
    keywordstyle=[2]\color{red},
    commentstyle=\small\ttfamily\color{myGreen},
    morekeywords={RequirePackage,ProvidesPackage},
    %
    % The special highlighting works for '!if', '!endif' and '!else'
    % But it doesn't work for '#if', '#endif' and '#else'.
    alsoletter = {!},
    keywords=[2]{!if,!endif,!else},
    otherkeywords={define,include,\# }
}

\lstdefinestyle{myCustomc}{
    language = customc,
    % keywordstyle = \color{myMauve},
}

\lstset{escapechar=@,style=myCustomc}


\definecolor{grey}{rgb}{0.3,0.3,0.3}
\definecolor{lightgrey}{rgb}{0.9,0.9,0.9}

\thispagestyle{empty}
\setlength{\textwidth}{18.5cm}
\setlength{\topmargin}{-2.5cm}
\setlength{\textheight}{24.5cm}
\setlength{\oddsidemargin}{-1cm}
\setlength{\evensidemargin}{-1cm}

\begin{document}
\begin{center}{\LARGE Compito di Programmazione - Bioinformatica}\\
\begin{center}
  \large 5 luglio 2023 (tempo disponibile: 2 ore)
\end{center}
\end{center}

\vspace*{1ex}
\begin{center}{\Large Esercizio 1} ($15$ punti)\\
  \textbf{(si consegni \texttt{bits.c})}
\end{center}

Un numero naturale \`e detto avere ripetizione due se
nella sua rappresentazione binaria ci sono almeno due bit ad 1 consecutivi. Per esempio, il
numero $22$ ha ripetizione due poich\'e la sua rappresentazione
binaria $10110$ ha almeno due bit ad 1 consecutivi. Similmente, il numero $92$ ha ripetizione due
poich\'e la sua rappresentazione binaria $1011100$ ha almeno due bit ad 1 consecutivi.
Invece il numero $18$ non ha ripetizione due poich\'e la sua rappresentazione binaria
$10010$ non ha almeno due bit ad 1 consecutivi.

Si completi il seguente file \texttt{bits.c} (\textbf{si noti che la funzione
\texttt{ripetizione2} deve essere ricorsiva}):

\begin{center}
  \begin{lstlisting}[language=myC]
// AGGIUNGERE QUI GLI #include NECESSARI

// riempie arr con interi casuali, divisibili per 3, tra 0 e 65535 inclusi;
// arr ha lunghezza length
void init_random(int arr[], int length) {
  // COMPLETARE
}

// determina se n ha ripetizione due, cioe' se la sua rappresentazione
// binaria contiene almeno due bit ad 1 consecutivi;
// si dia per scontato che n sia tra 0 e 65535 inclusi;
// questa funzione deve essere ricorsiva
int ripetizione2(int n) {
  return 0; // MODIFICARE E COMPLETARE, DEVE ESSERE RICORSIVA
}

// ordina gli elementi di arr, lungo length, in modo da spostare all'inizio
// i suoi elementi che hanno ripetizione due e alla fine i suoi
// elementi che non hanno ripetizione due
void sort_by_ripetizione2(int arr[], int length) {
  // COMPLETARE
}
  \end{lstlisting}
\end{center}

\begin{mdframed}[backgroundcolor=lightred] 
  \textbf{I file \texttt{bits.h} e \texttt{main\_bits.c} sono gi\`a scritti e completi, non vanno modificati e non vanno consegnati. Se servisse, si possono aggiungere funzioni ausiliarie dentro \texttt{bits.c}.}
\end{mdframed}

\vspace*{3ex}
Se tutto \`e corretto,
un esempio di esecuzione di \texttt{main\_bits.c} potrebbe essere:

\begin{mdframed}[backgroundcolor=verylightgray] 
\begin{verbatim}
Inserisci la lunghezza dell'array, non negativa: 100
60417 (ripetizione2=1)
7842 (ripetizione2=1)
36831 (ripetizione2=1)
42237 (ripetizione2=1)
.....
30150 (ripetizione2=1)
47757 (ripetizione2=1)
42273 (ripetizione2=0)
4362 (ripetizione2=0)
38226 (ripetizione2=0)
36936 (ripetizione2=0)
\end{verbatim}
\end{mdframed}

\begin{center}{\Large Esercizio 2} ($16$ punti)\\
  \textbf{(si consegni {es2\_lista\_multipli3.c})}
\end{center}

Si consideri il file allegato \texttt{es2\_lista\_multipli3.c}, riportato
per comodit\`a anche qui sotto.
Si completi la funzione \texttt{crea\_lista\_multipli\_3()} in modo che,
dato un numero intero $N$ maggiore di $100$, generi una lista contenente
tutti i numeri positivi multipli di $3$ e minori di $N$. Si completi anche
la funzione \texttt{calcola\_media()} in modo che, data una lista di interi,
calcoli la media aritmetica di tutti i valori della lista.
Si usi il \texttt{main} gi\`a scritto per testare le due funzioni.
\textbf{Si} possono aggiungere altre funzioni ma
\textbf{non} si deve modificare il \texttt{main}.

\begin{lstlisting}[language=myC]
#include <stdio.h>
#include <stdlib.h>

struct nodo *crea_lista_multipli_3(struct nodo *head,int valore);
float calcola_media(struct nodo *head);
struct nodo {
    int value;
    struct nodo *next;
};

// Program entry point
int main() {
    int n = 150;
    struct nodo* head = (struct nodo *)malloc(sizeof(struct nodo));
    head=crea_lista_multipli_3(head,n);
    printf("La media e' %f", calcola_media(head));
   
    return 0;
}

/**
 * Crea un lista con i multipli di 3 positivi e inferiori a n
 */
struct nodo *crea_lista_multipli_3(struct nodo *head, int n) {
   //da completare
}

/**
 * Restituisce la media degli elementi di una lista
 */
float calcola_media(struct nodo *head) {
    //da completare
}
\end{lstlisting}
	
	
	
	
	


\end{document}
