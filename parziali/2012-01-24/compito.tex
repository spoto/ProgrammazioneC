\documentclass[italian,12pt]{article}
\usepackage{babel}
\usepackage{amssymb}
\usepackage{amsmath}
\usepackage{graphicx}

\thispagestyle{empty}
\setlength{\textwidth}{18.5cm}
\setlength{\topmargin}{-2.5cm}
\setlength{\textheight}{24.5cm}
\setlength{\oddsidemargin}{-1cm}
\setlength{\evensidemargin}{-1cm}
\begin{document}
\begin{center}{\LARGE Prima prova parziale di Programmazione I}\\
\begin{center}
  \Large 24 gennaio 2012 (tempo disponibile: 2 ore)
\end{center}
\end{center}
%\mbox{}\\
\begin{center}{\Large Esercizio 1}\\
($9$ punti)
\end{center}
Si scriva un programma \texttt{integrale.c} che definisce una funzione
\begin{verbatim}
  double integra(double da, double a, double dx)
\end{verbatim}
che restituisce un'approssimazione di
\[
  \int_{\mathtt{da}}^{\mathtt{a}}\mathit{sin}(x)\,\mathit{dx}
\]
calcolata facendo iterare una variabile \texttt{x} dal valore \texttt{da} al valore \texttt{a}
a passi di \texttt{dx} e sommando ogni volta il valore del seno a \texttt{x} per il valore di \texttt{dx}.

Tale programma deve avere anche un \texttt{main()} che legge \texttt{da}, \texttt{a} e \texttt{dx}
da tastiera (di tipo \texttt{double}: si usi il formato \texttt{\%lf} per \texttt{scanf()}) e quindi
stampa a video l'approssimazione dell'integrale.
 
Se tutto \`e corretto, tale programma si deve comportare ad esempio come segue:
%
{\small
\begin{verbatim}
$ ./a.out
da: 0
a: 3.1415
dx: 0.2
integrale di sin(x) tra 0.000000 e 3.141500 d0.200000 = 1.997467

$ ./a.out
da: 3.1415
a: 6.283
dx: 0.001
integrale di sin(x) tra 3.141500 e 6.283000 d0.001000 = -2.000000
\end{verbatim}
}
%
\begin{center}{\Large Esercizio 2}\\
($6$ punti)
\end{center}
%
Si scriva il file \texttt{next\_prime.c} che definisce la funzione
\texttt{next\_prime()}. Questa funzione deve restituire un diverso numero primo
ad ogni chiamata, dal $2$ in poi. Solo \texttt{next\_prime()} deve
essere visibile all'esterno, non altre funzioni ausiliare che potreste
scrivere. Scrivete poi il file \texttt{next\_prime.h} che esporta
la segnatura della funzione \texttt{next\_prime()}.

Se tutto \`e corretto, un'esecuzione del seguente programma:
%
{\small
\begin{verbatim}
#include <stdio.h>
#include "next_prime.h"

int main(void) {
  int number;
  int c;

  printf("quanti numeri primi vuoi stampare? ");
  scanf("%d", &number);

  for (c = 0; c < number; c++)
    printf("%d\n", next_prime());

  return 0;
}
\end{verbatim}
}
%
\noindent
potrebbe essere:
%
{\small
\begin{verbatim}
quanti numeri primi vuoi stampare? 7
2
3
5
7
11
13
17
\end{verbatim}
}
%
\begin{center}{\Large Esercizio 3}\\
($16$ punti)\end{center}
%
Si scriva un programma \texttt{numeri\_italiani.c} che definisce le funzioni:
\begin{itemize}
\item \texttt{leggi()}, che legge da tastiera un numero intero non negativo e lo restituisce.
      Se fosse negativo, deve continuare a chiederlo all'utente;
\item \texttt{stampa(int numero)}, che stampa le cifre del numero indicato, in italiano. Per
      esempio, se \texttt{numero} \`e \texttt{4301} allora deve stampare \texttt{quattro tre zero uno};
      se \texttt{numero} \`e \texttt{0} allora deve stampare \texttt{zero}.
\end{itemize}
%
\`E possibile definire ulteriori funzioni ausiliarie, se servono. La funzione \texttt{stampa()}
\textbf{deve essere ricorsiva o chiamare una vostra funzione ricorsiva}.
Si definiscano gli argomenti delle funzioni come \texttt{const}, quando possibile.

Infine, tale programma deve avere anche un \texttt{main()} che chiama
\texttt{leggi()} per leggere un numero non negativo e poi chiama \texttt{stampa()}
per stamparne le cifre in italiano.

Se tutto \`e corretto, il programma si deve comportare ad esempio come segue:
%
{\small
\begin{verbatim}
$ ./a.out
inserisci un numero: 10985
uno zero nove otto cinque

$ ./a.out
inserisci un numero positivo: -13
inserisci un numero positivo: 8901
otto nove zero uno 

$ ./a.out
inserisci un numero: 300896
tre zero zero otto nove sei

$ ./a.out
inserisci un numero: 0
zero

$ ./a.out
inserisci un numero: 0006
sei
\end{verbatim}
}

\end{document}
