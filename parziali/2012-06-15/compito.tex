\documentclass[italian,12pt]{article}
\usepackage{babel}
\usepackage{amssymb}
\usepackage{amsmath}
\usepackage{graphicx}

\thispagestyle{empty}
\setlength{\textwidth}{18.5cm}
\setlength{\topmargin}{-2.5cm}
\setlength{\textheight}{24.5cm}
\setlength{\oddsidemargin}{-1cm}
\setlength{\evensidemargin}{-1cm}
\begin{document}
\begin{center}{\LARGE Seconda prova parziale di Programmazione I}\\
\begin{center}
  \Large 15 giugno 2012 (tempo disponibile: 2 ore)
\end{center}
\end{center}
%\mbox{}\\
\vspace*{2ex}
\begin{center}{\Large Esercizio 1}\\
($19$ punti)
\end{center}
Si scrivano i file \texttt{polinomio.h} e \texttt{polinomio.c} che definiscono un polinomio
a coefficienti interi in una sola variabile, cio\`e una cosa del tipo
$-3x^4 + 9x^3 - 11x^2 - 2$. Si devono definire e implementare le seguenti funzioni:
%
\begin{itemize}
\item \texttt{struct polinomio *construct\_polinomio(int coefficienti[], int grado)}, che costruisce
      un nuovo polinomio del grado indicato con i coefficienti indicati. Il grado \`e l'esponente del
      monomio pi\`u significativo. Si noti che il grado \`e
      sempre la lunghezza dell'array dei coefficienti meno 1. I coefficienti iniziano con quello
      del monomio pi\`u significativo. Per esempio, per costruire
      $-3x^4 + 9x^3 - 11x^2 - 2$ si deve poter scrivere
      \texttt{construct\_polinomio(coefficienti, 4)} dove
      \verb!int coefficienti[] = { -3, 9, -11, 0, -2 }!;
\item \texttt{void destruct\_polinomio(struct polinomio *this)}, che dealloca il polinomio indicato;
\item \texttt{int grado(struct polinomio *this)}, che restituisce il grado del polinomio indicato;
\item \texttt{struct polinomio *add(struct polinomio *this, struct polinomio *other)}, che resti\-tui\-sce un
      nuovo polinomio che \`e la somma dei due polinomi indicati;
\item \texttt{int evaluate(struct polinomio *this, int x)}, che restituisce il valore del polinomio indicato
      valutato nel punto indicato;
\item \texttt{char *toString(struct polinomio *this)}, che restituisce una nuova stringa che descrive
      il polinomio indicato, una cosa del tipo \verb!- 3x^4 + 9x^3 - 11x^2 - 2x^0! (i monomi di coefficiente
      0 non devono essere riportati).
\end{itemize}

\noindent\textbf{Suggerimento}: per realizzare la \texttt{toString}, pu\`o essere comodo ricordarsi che la
funzione della libreria standard \texttt{sprintf(buffer, format, valori\ldots)} si comporta come
\texttt{printf(format, valori\ldots)} ma stampa dentro la stringa \texttt{buffer} invece che sul video.
Inoltre \texttt{sprintf} ritorna un \texttt{int} che \`e il numero di caratteri stampati.

Se tutto \`e corretto, il seguente programma:
%
{\small
\begin{verbatim}
#include <stdio.h>
#include <stdlib.h>
#include "polinomio.h"

int main() {
  int coefficienti1[] = { -3, 4, 0, 0, -2 };
  int coefficienti2[] = { 5, -11, 0, 0 };

  struct polinomio *poly1 = construct_polinomio(coefficienti1, 4);
  struct polinomio *poly2 = construct_polinomio(coefficienti2, 3);
  struct polinomio *somma = add(poly1, poly2);

  char *s;

  printf("primo polinomio: %s di grado %d\n", s = toString(poly1), grado(poly1));
  free(s);

  printf("secondo polinomio: %s di grado %d\n", s = toString(poly2), grado(poly2));
  free(s);

  printf("somma: %s di grado %d\n", s = toString(somma), grado(somma));
  free(s);

  printf("la somma valutata in x = 7 vale %d\n", evaluate(somma, 7));

  destruct_polinomio(somma);
  destruct_polinomio(poly2);
  destruct_polinomio(poly1);

  return 0;
}
\end{verbatim}
}
%
\noindent stamper\`a:
%
{\small
\begin{verbatim}
primo polinomio:  - 3x^4 + 4x^3 - 2x^0 di grado 4
secondo polinomio:  + 5x^3 - 11x^2 di grado 3
somma:  - 3x^4 + 9x^3 - 11x^2 - 2x^0 di grado 4
la somma valutata in x = 7 vale -4657
\end{verbatim}
}
%
\vspace*{2ex}
\begin{center}{\Large Esercizio 2}\\
($13$ punti)\end{center}
%
Si scriva una funzione ricorsiva \texttt{sup} che riceve come argomento una lista
\texttt{this}
e restituisce una lista contenente solo gli elementi di
\texttt{this} che sono strettamente maggiori di tutti quelli che li seguono in \texttt{this}. La lista
\texttt{this} non deve essere modificata. Si deve gestire anche il caso
in cui \texttt{this} \`e la lista vuota. Per esempio,
se \texttt{this} fosse la lista \texttt{[5,13,8,9,8,3,2,4,2]} allora
\texttt{sup(this)} deve essere la lista \texttt{[13,9,8,4,2]}.
Si scriva quindi un programma
\texttt{main\_list.c} con una funzione \texttt{main} che costruisce la lista
\texttt{[5,13,8,9,8,3,2,4,2]}, la stampa con la funzione \texttt{print\_list},
chiama la funzione \texttt{sup} su tale lista e stampa il risultato
di \texttt{sup} con \texttt{print\_list}. La definizione delle liste e le funzioni
di costruzione e di stampa delle liste sono state viste a lezione: non riscrivetele
nella soluzione.

\end{document}
