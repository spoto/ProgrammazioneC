\documentclass[italian,12pt]{article}
\usepackage{amssymb}
\usepackage{amsmath}
\usepackage{graphicx}

\thispagestyle{empty}
\setlength{\textwidth}{18.5cm}
\setlength{\topmargin}{-2.5cm}
\setlength{\textheight}{24.5cm}
\setlength{\oddsidemargin}{-1cm}
\setlength{\evensidemargin}{-1cm}
\begin{document}
\begin{center}{\LARGE Seconda prova parziale di Programmazione I}\\
\begin{center}
  \Large 23 giugno 2014 (tempo disponibile: 2 ore)
\end{center}
\end{center}
%\mbox{}\\
\vspace*{2ex}
\begin{center}{\Large Esercizio 1} ($12$ punti)
\end{center}
Il coefficiente binomiale degli interi non negativi $n$ e $k$ \`e definito come
\[
  {{n}\choose{k}}=\frac{n!}{k!\cdot(n - k)!}
\]
%
Si definisca una funzione
%
\begin{verbatim}
char *stringa_binomiale(int n, int k)
\end{verbatim}
%
che restituisce una nuova stringa del tipo \verb|"n su k fa b"| dove $n$ e $k$ sono i valori dei parametri
della funzione e $b$ \`e il valore di $n\choose k$. Si assuma che per scrivere un intero siano sempre sufficienti
$10$ caratteri.

Se tutto \`e corretto, un'esecuzione del seguente programma:

{\scriptsize
\begin{verbatim}
int main(void) {
  int n, k;
  char *s;
  printf("n: " ); scanf("%i", &n);
  printf("k: " ); scanf("%i", &k);
  printf("%s\n", s = stringa_binomiale(n, k));
  free(s);
  return 0;
}
\end{verbatim}}

\noindent
\`e la seguente:
%
{\scriptsize
\begin{verbatim}
n: 8
k: 3
8 su 3 fa 56
\end{verbatim}}

\begin{center}{\Large Esercizio 2} ($9$ punti)\end{center}
%
Si considerino le liste di interi come viste a lezione. Si definisca una funzione \textbf{ricorsiva}
%
\begin{verbatim}
int sempre_diversi(struct list *this)
\end{verbatim}
%
che restituisce \emph{vero} se e solo se nella lista \texttt{this} due elementi contigui sono sempre diversi.
%
\begin{center}{\Large Esercizio 3} ($11$ punti)\end{center}
%
Si consideri il seguente file \texttt{vocabolario.h} che definisce la struttura e le funzioni per un vocabolario italiano/inglese
che pu\`o contenere fino a 100 voci:

{\scriptsize
\begin{verbatim}
#ifndef VOCABOLARIO_H
#define VOCABOLARIO_H

struct voce {     // una singola voce italiano/inglese
  char *italiano;
  char *inglese;
};

#define SIZE 100

struct vocabolario {  // il vero e proprio vocabolario
  struct voce voci[SIZE];  // le voci contenute nel vocabolario, non tutte usate
  int voci_usate; // quante voci del vocabolario sono gia' state usate
};

struct vocabolario *construct_vocabolario(); // costruisce un vocabolario vuoto
void destroy_vocabolario(struct vocabolario *this); // dealloca un vocabolario
void print_vocabolario(struct vocabolario *this);  // stampa sul video un vocabolario
void add_in_vocabolario(struct vocabolario *this, char *italiano, char *inglese);  // aggiunge una voce al vocabolario
char *cerca_dallitaliano(struct vocabolario *this, char *italiano);  // cerca una traduzione dall'italiano
char *cerca_dallinglese(struct vocabolario *this, char *inglese);  // cerca una traduzione dall'inglese

#endif
\end{verbatim}}

\noindent
Si completi la seguente implementazione \texttt{vocabolario.c}:

{\scriptsize
\begin{verbatim}
#include <stdlib.h>
#include <stdio.h>
#include "vocabolario.h"

struct vocabolario *construct_vocabolario() { COMPLETATE }
void destroy_vocabolario(struct vocabolario *this) { COMPLETATE }

void print_vocabolario(struct vocabolario *this) {
  int pos;

  printf("%-20s | %-20s\n", "italiano", "inglese");
  printf("%-20s | %-20s\n", "--------", "-------");

  for (pos = 0; pos < this->voci_usate; pos++)
      printf("%-20s | %-20s\n", this->voci[pos].italiano, this->voci[pos].inglese);
}

void add_in_vocabolario(struct vocabolario *this, char *italiano, char *inglese) { COMPLETATE }
char *cerca_dallitaliano(struct vocabolario *this, char *italiano) { COMPLETATE }
char *cerca_dallinglese(struct vocabolario *this, char *inglese) { COMPLETATE }
\end{verbatim}}

\noindent
Se tutto \`e corretto, l'esecuzione del seguente programma:

{\scriptsize
\begin{verbatim}
#include <stdio.h>
#include "vocabolario.h"

int main(void) {
  struct vocabolario *v = construct_vocabolario();
  add_in_vocabolario(v, "cane", "dog");
  add_in_vocabolario(v, "topo", "mouse");
  add_in_vocabolario(v, "casa", "house");
  add_in_vocabolario(v, "simpatico", "nice");
  add_in_vocabolario(v, "correre", "to run");

  print_vocabolario(v);

  printf("La traduzione in inglese di 'simpatico' e' %s\n", cerca_dallitaliano(v, "simpatico"));
  printf("La traduzione in italiano di 'house' e' %s\n", cerca_dallinglese(v, "house"));

  destroy_vocabolario(v);

  return 0;
}
\end{verbatim}}

\noindent
dovr\`a stampare:

{\scriptsize
\begin{verbatim}
italiano             | inglese             
--------             | -------             
cane                 | dog                 
topo                 | mouse               
casa                 | house               
simpatico            | nice                
correre              | to run              
La traduzione in inglese di 'simpatico' e' nice
La traduzione in italiano di 'house' e' casa
\end{verbatim}}

\end{document}
