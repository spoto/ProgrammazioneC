\documentclass{article}[10pt]
\usepackage[pdftex]{graphicx}
\usepackage{amsfonts}
\usepackage[italian]{babel}
%****************enlarge layout
\textheight     243.5mm
\topmargin      -20.0mm
\textwidth      480pt
\hoffset        -80pt
%*****************theorems and such
\newcounter{esnu}
\newenvironment{esercizio}{\medskip \noindent {\bf Esercizio\addtocounter{esnu}{1} \arabic{esnu}}}{}
\pagestyle{empty}
\newcommand{\liff}{\mathrel{\leftrightarrow}}   % Logical IFF Symbol
\newcommand{\metaiff}{\Longleftrightarrow}      %iff in metatheory

\begin{document}

\begin{center} {\bf Esame di Programmazione I, 15 luglio 2013. 2 ore}\end{center}

\begin{esercizio}
\textbf{[10 punti]}
Si scriva una funzione

\begin{verbatim}
void split(char *s)
\end{verbatim}

\noindent
che modifica la stringa \texttt{s} in modo da spostare all'inizio i caratteri alfabetici,
al centro le cifre numeriche e a destra i restanti caratteri.
Se tutto \`e corretto, l'esecuzione del seguente programma:

{\small
\begin{verbatim}
#include <stdio.h>
#include <string.h>

int main(void) {
  char buffer[100];
  strcpy(buffer, "de80f?& HY(k,;;F849=");
  split(buffer);
  printf("%s\n", buffer);
  return 0;
}
\end{verbatim}
}

\noindent
dovr\`a stampare qualcosa del tipo \texttt{defHYkF80849?\& (,;;=}\quad
(ci sono altre stampe possibili e corrette, basta che la tripartizione venga rispettata).
\end{esercizio}

\begin{esercizio}
\textbf{[12 punti]}
Si considerino le liste di stringhe di caratteri (quindi il tipo di \texttt{head}
\`e \texttt{char *}). Si implementi una funzione \underline{ricorsiva}:
\begin{verbatim}
struct list *insert_ordered(char *s, struct list *this)
\end{verbatim}
che riceve una stringa \texttt{s} e una lista \texttt{this} ordinata alfabeticamente e restituisce una lista
in cui \texttt{s} \`e stata piazzata in ordine alfabetico dentro \texttt{this}. La lista \texttt{this} non deve
venire modificata dalla funzione. \textbf{Nota}: \texttt{this}
\`e
gi\`a ordinata quando viene passata alla funzione \texttt{insert\_ordered}.

Se tutto \`e corretto, l'esecuzione del programma:

{\small
\begin{verbatim}
#include <stdio.h>
#include "list.h"

int main(void) {
  struct list *l = construct_list("domani", construct_list("piove", construct_list("tantissimo", NULL)));
  struct list *r;
  printf("l = "); print_list(l); printf("\n");
  r = insert_ordered("non", l);
  printf("l = "); print_list(l); printf("\n");
  printf("r = "); print_list(r); printf("\n");
  return 0;
}

\end{verbatim}
}
\noindent
dovr\`a stampare:

{\small
\begin{verbatim}
l = [domani, piove, tantissimo]
l = [domani, piove, tantissimo]
r = [domani, non, piove, tantissimo]
\end{verbatim}
}
%
\end{esercizio}

\begin{esercizio}
\textbf{[9 punti]}
%
Si scrivano i file \texttt{portata.h}, \texttt{portata.c}, \texttt{conto.h} e \texttt{conto.c} che
definiscono e implementano le portate e il conto di un ristorante. Sulle portate devono essere definite
le seguenti funzioni:

{\small
\begin{verbatim}
struct portata *construct_portata(const char *nome, int prezzo);
void destroy_portata(struct portata *this);
\end{verbatim}
}

\noindent
e sul conto:

{\small
\begin{verbatim}
struct conto *construct_conto(struct portata *portate[], int numero_portate);
void destroy_conto(struct conto *this);
void print_conto(struct conto *this);
\end{verbatim}
}

\noindent
dove la funzione \texttt{print\_conto} stampa il conto come si pu\`o vedere sotto, includendo
2 euro di coperto e il 21\% di iva.
Se tutto \`e corretto, l'esecuzione del seguente programma:

{\small
\begin{verbatim}
#include <stdio.h>
#include "conto.h"
#include "portata.h"

int main(void) {
  struct portata *p1 = construct_portata("risotto agli asparagu", 10);
  struct portata *p2 = construct_portata("pasta alla carbonara", 6);
  struct portata *p3 = construct_portata("vitello tonnato", 12);
  struct portata *p4 = construct_portata("frutta fresca", 9);
  struct portata *p5 = construct_portata("macedonia di frutta", 7);
  struct portata *p6 = construct_portata("tiramisu", 5);
  struct portata *p7 = construct_portata("acqua naturale", 4);
  struct portata *p8 = construct_portata("caffe espresso", 2);
  struct portata *cliente1[] = { p1, p3, p4, p7, p7, p8 };
  struct portata *cliente2[] = { p2, p3, p5, p6, p7, p8, p8 };
  struct conto *conto1 = construct_conto(cliente1, 6);
  struct conto *conto2 = construct_conto(cliente2, 7);

  print_conto(conto1); printf("\n"); print_conto(conto2);

  destroy_conto(conto2); destroy_conto(conto1);
  destroy_portata(p8); destroy_portata(p7); destroy_portata(p6); destroy_portata(p5);
  destroy_portata(p4); destroy_portata(p3); destroy_portata(p2); destroy_portata(p1);
  return 0;
}
\end{verbatim}}
\noindent
dovr\`a stampare qualcosa del tipo:

{\small
\begin{verbatim}
                TRATTORIA DA NONNA PAPERA

         risotto agli asparagu       10 euro
               vitello tonnato       12 euro
                 frutta fresca        9 euro
                acqua naturale        4 euro
                acqua naturale        4 euro
                caffe espresso        2 euro
                       coperto        2 euro
                -------------------------
                        TOTALE      43 euro
                       IVA 21%      9.03 euro
                -------------------------
               TOTALE A PAGARE      52.03 euro

                TRATTORIA DA NONNA PAPERA

          pasta alla carbonara        6 euro
               vitello tonnato       12 euro
           macedonia di frutta        7 euro
                      tiramisu        5 euro
                acqua naturale        4 euro
                caffe espresso        2 euro
                caffe espresso        2 euro
                       coperto        2 euro
                -------------------------
                        TOTALE      40 euro
                       IVA 21%      8.40 euro
                -------------------------
               TOTALE A PAGARE      48.40 euro
\end{verbatim}
}

\end{esercizio}
%
\end{document}
