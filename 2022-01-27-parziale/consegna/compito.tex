\documentclass[12pt]{article}
\usepackage{amssymb}
\usepackage{amsmath}
\usepackage{color}
\usepackage{graphicx}
\usepackage{mdframed}
\usepackage{listings, xcolor}
\usepackage{textcomp}

\definecolor{verylightgray}{rgb}{.97,.97,.97}

\lstdefinelanguage{myC}{
        keywords=[1]{break, case, continue, default, do
, else, false, for, if, const, return, switch, true, while}, % generic keywords
        keywordstyle=[1]\color{blue}\bfseries, 
        keywords=[2]{bool, int, long, float, double, byte, short, char, void, signed, unsigned}, % types
        keywordstyle=[2]\color{teal}\bfseries,
        keywordstyle=[2]\color{violet}\bfseries,
        keywords=[3]{NULL},
        keywordstyle=[3]\color{teal}\bfseries,
        identifierstyle=\color{black},
        sensitive=false,
        comment=[l]{//},
        morecomment=[s]{/*}{*/},
        commentstyle=\color{violet}\ttfamily,
        commentstyle=\small\ttfamily\color{myGreen},
        stringstyle=\color{red}\ttfamily,
        morestring=[b]',
        morestring=[b]"
}

\lstset{
        language=myC,
        backgroundcolor=\color{verylightgray},
        extendedchars=true,
        basicstyle=\small\ttfamily,
        showstringspaces=false,
        showspaces=false,
        numbers=none,
        numberstyle=\small,
        numbersep=9pt,
        tabsize=2,
        upquote=true,
        breaklines=true,
        showtabs=false,
        captionpos=b
        otherkeywords={define,include,\# }
}

\definecolor{myBlue}{rgb}{0.5,0.5,1}
\definecolor{myLightBlue}{rgb}{0.35,0.6,0.8}
\definecolor{myBlack}{rgb}{0,0,0}
\definecolor{myGreen}{rgb}{0.1,0.6,0.2}
\definecolor{myGray}{rgb}{0.5,0.5,0.5}
\definecolor{myLightgray}{rgb}{0.95,0.95,0.95}
\definecolor{myMauve}{rgb}{0.58,0,0.82}
\lstdefinelanguage{customc}{
    language=C,
    backgroundcolor = \color{myLightgray},
    basicstyle=\small\ttfamily\color{myBlack},
    keywordstyle=\color{myLightBlue},
    keywordstyle=[2]\color{red},
    commentstyle=\small\ttfamily\color{myGreen},
    morekeywords={RequirePackage,ProvidesPackage},
    %
    % The special highlighting works for '!if', '!endif' and '!else'
    % But it doesn't work for '#if', '#endif' and '#else'.
    alsoletter = {!},
    keywords=[2]{!if,!endif,!else},
    otherkeywords={define,include,\# }
}

\lstdefinestyle{myCustomc}{
    language = customc,
    % keywordstyle = \color{myMauve},
}

\lstset{escapechar=@,style=myCustomc}


\definecolor{grey}{rgb}{0.3,0.3,0.3}
\definecolor{lightgrey}{rgb}{0.9,0.9,0.9}

\thispagestyle{empty}
\setlength{\textwidth}{18.5cm}
\setlength{\topmargin}{-2.5cm}
\setlength{\textheight}{24.5cm}
\setlength{\oddsidemargin}{-1cm}
\setlength{\evensidemargin}{-1cm}

\begin{document}
\begin{center}{\LARGE Parziale di Programmazione I - Bioinformatica}\\
\begin{center}
  \large 27 gennaio 2022 (tempo disponibile: 2 ore)
\end{center}
\end{center}

\begin{center}{\Large Esercizio 1} ($20$ punti)\\
  \textbf{(si consegni \texttt{harshad.c} e \texttt{harshad.h})}
\end{center}
Si scriva un programma \texttt{harshad.c} che implementa le seguenti funzioni:
\begin{center}
\begin{lstlisting}[language=myC]
// inizializza arr, lungo length, con numeri interi lunghi casuali
// tra 0 a 20 inclusi,
// usando srand() e rand(), facendo in modo che alla fine
// non ci siano elementi consecutivi uguali nell'array
void init_random(long arr[], int length);

// stampa su un'unica riga il contenuto dell'array arr, lungo length,
// con parentesi quadre all'inizio e alla fine e con virgole
// fra gli elementi; poi va a capo
void print(long arr[], int length);

// determina se n e' un numero Harshad, cioe' e' positivo e divisibile
// per la somma delle proprie cifre. Per esempio, 1729 e' Harshad
// poiche' 1+7+2+9 fa 19 e 1729 e' divisibile per 19
int is_harshad(long n);

// modifica l'array, lungo length, in modo da sostituire con -1
// gli elementi che sono numeri Harshad
void delete_harshad(long arr[], int length);
\end{lstlisting}
\end{center}
%
Si scriva quindi un file di header \texttt{harshad.h} che dichiara le precedenti funzioni.

\mbox{}\\
\begin{center}{\Large Esercizio 2} ($12$ punti)\\
  \textbf{(si consegni \texttt{main.c})}\end{center}
%
Si scriva un programma \texttt{main.c} che include le funzioni dell'Esercizio~1 tramite
il file \texttt{harshad.h}.
Il programma \texttt{main.c} deve contenere una funzione iniziale \texttt{main} che esegue
le seguenti operazioni:
\begin{enumerate}
\item legge da tastiera la lunghezza \texttt{length} di un array, richiedendola ad oltranza se fosse inserita negativa;
\item crea un array \texttt{elements} di \texttt{length} interi lunghi;
\item chiama la funzione \texttt{init\_random} per inizializzare \texttt{elements} in modo casuale;
\item chiama la funzione \texttt{print} per stampare \texttt{elements};
\item chiama la funzione \texttt{delete\_harshad} con l'array \texttt{elements} come parametro;
\item chiama la funzione \texttt{print} per stampare \texttt{elements}.
\end{enumerate}

\end{document}
