\documentclass[12pt]{article}
\usepackage{amssymb}
\usepackage{amsmath}
\usepackage{color}
\usepackage{graphicx}

\definecolor{grey}{rgb}{0.3,0.3,0.3}

\usepackage{listings, framed}
\lstset{
  language=Java,
  showstringspaces=false,
  columns=flexible,
  basicstyle={\small\ttfamily},
  frame=none,
  numbers=none,
  keywordstyle=\bfseries\color{grey},
  commentstyle=\itshape\color{red},
  identifierstyle=\color{black},
  stringstyle=\color{blue},
  numberstyle={\ttfamily},
%  breaklines=true,
  breakatwhitespace=true,
  tabsize=3,
  escapechar=|
}

\thispagestyle{empty}
\setlength{\textwidth}{18.5cm}
\setlength{\topmargin}{-2.5cm}
\setlength{\textheight}{24.5cm}
\setlength{\oddsidemargin}{-1cm}
\setlength{\evensidemargin}{-1cm}
\begin{document}
\begin{center}{\LARGE Parziale di Programmazione I - BioInformatica}\\
\vspace*{-2ex}
\begin{center}
  \large 22 gennaio 2021 (tempo disponibile: 2 ore)
\end{center}
\end{center}

\vspace*{1ex}
\begin{center}{\Large Esercizio 1} ($20$ punti)\\
  \textbf{(si consegni \texttt{permutations.c} e \texttt{permutations.h})}
\end{center}
Si scriva un programma \texttt{permutations.c} che implementa le seguenti tre funzioni:
\begin{verbatim}
// inizializza arr, lungo length, con una permutazione casuale
// dei numeri interi da 0 a length-1, usando srand() e rand()
void init_random(int arr[], int length);

// stampa su un'unica riga il contenuto dell'array arr, lungo length, poi va a capo
void print(int arr[], int length);

// determina se gli elementi dell'array arr, lungo length, sono tutti diversi fra loro
int different(int arr[], int length);
\end{verbatim}
%
Per esempio, chiamando la funzione \texttt{init\_random(arr, 5)} l'array \texttt{arr} potr\`a diventare
\texttt{\{3,0,1,2,4\}}, oppure \texttt{\{2,4,1,3,0\}},
oppure \texttt{\{0,1,2,3,4\}}, oppure \texttt{\{4,3,2,1,0\}}, oppure qualsiasi altra
permutazione di $0\ldots 4$.

Si scriva quindi un file di header \texttt{permutations.h} che dichiara le precedenti tre funzioni.

\begin{center}{\Large Esercizio 2} ($12$ punti)\\
  \textbf{(si consegni \texttt{main.c})}\end{center}
%
Si scriva un programma \texttt{main.c} che include le funzioni dell'Esercizio~1 tramite
il file \texttt{permutations.h}.
Il programma \texttt{main.c} deve contenere una funzione iniziale \texttt{main} che esegue
le seguenti operazioni:
\begin{enumerate}
\item legge da tastiera la lunghezza \texttt{length} di un array, richiedendola ad oltranza se fosse inserita negativa;
\item crea un array \texttt{elements} di \texttt{length} interi;
\item chiama la funzione \texttt{init\_random} per inizializzare \texttt{elements} a una permutazione casuale;
\item chiama la funzione \texttt{print} per stampare \texttt{elements};
\item chiama la funzione \texttt{different} con l'array \texttt{elements} come parametro e stampa il suo risultato.
\end{enumerate}

\end{document}
