\documentclass{article}[10pt]
\usepackage[pdftex]{graphicx}
\usepackage{amsfonts}
\usepackage[italian]{babel}
%****************enlarge layout
\textheight     243.5mm
\topmargin      -20.0mm
\textwidth      480pt
\hoffset        -80pt
%*****************theorems and such
\newcounter{esnu}
\newenvironment{esercizio}{\medskip \noindent {\bf Esercizio\addtocounter{esnu}{1} \arabic{esnu}}}{}
\pagestyle{empty}
\newcommand{\liff}{\mathrel{\leftrightarrow}}   % Logical IFF Symbol
\newcommand{\metaiff}{\Longleftrightarrow}      %iff in metatheory

%

%\`e

\begin{document}

\begin{center} {\bf Esame di Programmazione I, 28 febbraio 2014. 2 ore}\end{center}

\begin{esercizio}
\textbf{[12 punti]}
Si scriva una funzione

\begin{verbatim}
char *replace(const char *where, const char *what) {
\end{verbatim}

\noindent
che restituisce una nuova stringa ottenuta da \texttt{where} sostituendo tutti i caratteri
\texttt{\$} con la stringa \texttt{what}.

Per esempio, l'esecuzione del seguente \texttt{main}:

{\small
\begin{verbatim}
int main(void) {
  const char *s = "ciao $$ amico $";

  char *r = replace(s, "caro");

  printf("%s\n", r);

  free(r);

  return 0;
}
\end{verbatim}
}

\noindent
dovr\`a stampare:
%
{\small
\begin{verbatim}
ciao carocaro amico caro
\end{verbatim}
}
\end{esercizio}

\begin{esercizio}
\textbf{[13 punti]}
Considerando le liste di caratteri come viste a lezione, si scriva una funzione \underline{ricorsiva}:

\begin{verbatim}
struct list *replace(struct list *this, struct list *what)
\end{verbatim}

\noindent
che restituisce una lista derivata da \texttt{this} sostituendo il carattere
\texttt{\$} con la lista \texttt{what}.

Per esempio, il seguente programma:
%
{\small
\begin{verbatim}
#include <stdlib.h>
#include "list.h"

int main(void) {
  struct list *where = construct_list
    ('d', construct_list('$', construct_list('$', construct_list('E', construct_list('$', NULL)))));
  struct list *what = construct_list('A', construct_list('3', NULL));

  struct list *r = replace(where, what);
  print_list(r);
  
  return 0;
}
\end{verbatim}
}

\noindent
dovr\`a stampare:
%
{\small
\begin{verbatim}
[d, A, 3, A, 3, E, A, 3]
\end{verbatim}
}
\end{esercizio}

\newpage

\begin{esercizio}
\textbf{[7 punti]}
Si consideri il seguente file di include \texttt{cartellone.h} che specifica
un cartellone del tipo di quelli delle stazioni ferroviarie, che indica una sequenza
di treni specificando il numero e la destinazione di ciascun treno:

{\small
\begin{verbatim}
#ifndef CARTELLONE_H
#define CARTELLONE_H

struct cartellone {
  int numero_treno[8];
  const char *destinazione[8];
  int usati; // il numero di elementi degli array che sono usati
};

struct cartellone *construct_cartellone(void);
void destroy_cartellone(struct cartellone *this);
void print_cartellone(struct cartellone *this);
void aggiungi_treno(struct cartellone *this, int numero_treno, const char *destinazione);
void rimuovi_treno(struct cartellone *this);

#endif
\end{verbatim}
}

\noindent
Il cartellone nasce vuoto ed \`e
possibile aggiungere treni con la funzione
\texttt{aggiungi\_treno} (specificando il numero del treno e la sua destinazione. Se il cartellone
\`e pieno, questa funzione non fa nulla),
stamparlo con la funzione \texttt{print\_cartellone} e rimuovere il treno che sta in cima
al cartellone con la funzione \texttt{rimuovi\_treno}. Quest'ultima funzione
sposta tutti i treni di una posizione verso l'alto.

Se tutto \`e corretto, l'esecuzione del programma:

{\small
\begin{verbatim}
#include <stdio.h>
#include "cartellone.h"

int main(void) {
  struct cartellone *c = construct_cartellone();

  aggiungi_treno(c, 3408, "Vicenza");
  aggiungi_treno(c, 129, "Milano");
  aggiungi_treno(c, 891, "Roma");

  print_cartellone(c);
  rimuovi_treno(c);  // fa scomparire il treno 3408 per Vicenza
  printf("\n");
  print_cartellone(c);
  destroy_cartellone(c);

  return 0;
}
\end{verbatim}
}
\noindent
dovr\`a stampare:

{\small
\begin{verbatim}
3408 - Vicenza
 129 - Milano
 891 - Roma

 129 - Milano
 891 - Roma
\end{verbatim}
}
%
\end{esercizio}
%
\end{document}
