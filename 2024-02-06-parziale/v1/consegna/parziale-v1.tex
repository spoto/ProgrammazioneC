\documentclass[12pt]{article}
\usepackage{amssymb}
\usepackage{amsmath}
\usepackage{color}
\usepackage{graphicx}
\usepackage{mdframed}
\usepackage{listings, xcolor}
\usepackage{textcomp}

\definecolor{verylightgray}{rgb}{.97,.97,.97}
\definecolor{lightred}{rgb}{.97,.50,.50}

\lstdefinelanguage{myC}{
        keywords=[1]{break, case, continue, default, do
, else, false, for, if, const, return, switch, true, while}, % generic keywords
        keywordstyle=[1]\color{blue}\bfseries,
        keywords=[2]{bool, int, long, float, double, byte, short, char, void, signed, unsigned}, % types
        keywordstyle=[2]\color{teal}\bfseries,
        keywordstyle=[2]\color{violet}\bfseries,
        keywords=[3]{NULL},
        keywordstyle=[3]\color{teal}\bfseries,
        identifierstyle=\color{black},
        sensitive=false,
        comment=[l]{//},
        morecomment=[s]{/*}{*/},
        commentstyle=\color{violet}\ttfamily,
        commentstyle=\small\ttfamily\color{myGreen},
        stringstyle=\color{red}\ttfamily,
        morestring=[b]',
        morestring=[b]"
}

\lstset{
        language=myC,
        backgroundcolor=\color{verylightgray},
        extendedchars=true,
        basicstyle=\small\ttfamily,
        showstringspaces=false,
        showspaces=false,
        numbers=none,
        numberstyle=\small,
        numbersep=9pt,
        tabsize=2,
        upquote=true,
        breaklines=true,
        showtabs=false,
        captionpos=b
        otherkeywords={define,include,\# }
}

\definecolor{myBlue}{rgb}{0.5,0.5,1}
\definecolor{myLightBlue}{rgb}{0.35,0.6,0.8}
\definecolor{myBlack}{rgb}{0,0,0}
\definecolor{myGreen}{rgb}{0.1,0.6,0.2}
\definecolor{myGray}{rgb}{0.5,0.5,0.5}
\definecolor{myLightgray}{rgb}{0.95,0.95,0.95}
\definecolor{myMauve}{rgb}{0.58,0,0.82}
\lstdefinelanguage{customc}{
    language=C,
    backgroundcolor = \color{myLightgray},
    basicstyle=\small\ttfamily\color{myBlack},
    keywordstyle=\color{myLightBlue},
    keywordstyle=[2]\color{red},
    commentstyle=\small\ttfamily\color{myGreen},
    morekeywords={RequirePackage,ProvidesPackage},
    %
    % The special highlighting works for '!if', '!endif' and '!else'
    % But it doesn't work for '#if', '#endif' and '#else'.
    alsoletter = {!},
    keywords=[2]{!if,!endif,!else},
    otherkeywords={define,include,\# }
}

\lstdefinestyle{myCustomc}{
    language = customc,
    % keywordstyle = \color{myMauve},
}

\lstset{escapechar=@,style=myCustomc}


\definecolor{grey}{rgb}{0.3,0.3,0.3}
\definecolor{lightgrey}{rgb}{0.9,0.9,0.9}

\thispagestyle{empty}
\setlength{\textwidth}{18.5cm}
\setlength{\topmargin}{-2.5cm}
\setlength{\textheight}{24.5cm}
\setlength{\oddsidemargin}{-1cm}
\setlength{\evensidemargin}{-1cm}

\begin{document}
\begin{center}{\LARGE Parziale di Programmazione I - Bioinformatica}\\
  \large 6 febbraio 2024, turno delle 9:00 (tempo disponibile: 2 ore)
\end{center}

\vspace*{1ex}
\begin{center}{\Large Esercizio 1} ($19$ punti)\\
  \textbf{(si consegni \texttt{letters.c} e \texttt{letters.h})}
\end{center}

Si completi il seguente file \texttt{letters.c}:

\begin{center}
\begin{lstlisting}[language=myC]
// aggiungete #include e funzioni ausiliarie se servono

// inizializza arr, lungo length, con lettere casuali dell'alfabeto
// inglese, maiuscole o minuscole, in modo tale che non ci siano
// mai due vocali di seguito (si consideri la definizione italiana di
// vocale: aeiou, minuscole o maiuscole); non randomizza la sequenza
// casuale con srand(time(NULL)): ci pensa il main
void init(char arr[], int length) {
  // completare
}

// stampa (senza fare spazi in mezzo) i caratteri di arr, lungo length,
// e va a capo alla fine
void print(char arr[], int length) {
  // completare
}

// modifica l'array arr, lungo length, in modo che le lettere
// minuscole diventino maiuscole e viceversa
void invert(char arr[], int length) {
  // completare
}
\end{lstlisting}
\end{center}
%
Si scriva quindi un file di header \texttt{letters.h} che dichiara le precedenti tre funzioni.

Per esempio, il seguente file \texttt{main.c}
(gi\`a fornito e da non modificare):

\begin{center}
  \begin{lstlisting}[language=myC]
#include <stdio.h>
#include <stdlib.h>
#include <time.h>
#include "letters.h"

int main(void) {
  srand(time(NULL)); // inizializza la sequenza casuale
  char array[70];    // crea l'array
  init(array, 70);   // inizializza l'array in modo casuale
  printf("array prima: ");
  print(array, 70);  // stampa l'array
  invert(array, 70); // inverte maiuscole con minuscole e viceversa
  printf(" array dopo: ");
  print(array, 70);  // stampa nuovamente l'array
  return 0;
}
  \end{lstlisting}
\end{center}
deve stampare qualcosa del tipo:

\begin{mdframed}[backgroundcolor=lightgrey] 
\begin{verbatim}
array prima: fxkScXYCdhJXCfIhsOKXkZIlfvFusdMPTilsdIVzOvSdTnvMKWeVivWishVWTvqhExURXk
 array dopo: FXKsCxycDHjxcFiHSokxKziLFVfUSDmptILSDivZoVsDtNVmkwEvIVwISHvwtVQHeXurxK
\end{verbatim}
\end{mdframed}

\begin{center}
  {\Large Esercizio 2} ($12$ punti)\\
  \textbf{(si consegni \texttt{triangle.c})}
\end{center}
%
Si completi il seguente programma \texttt{triangle.c}.
\textbf{La funzione triangle deve essere ricorsiva}:

\begin{center}
  \begin{lstlisting}[language=myC]
#include <stdio.h>

// stampa un triangolo rettangolo di altezza h, con
// l'angolo retto in alto a sinistra, con bordi sinistro e destro
// fatti dal carattere chiocciolina e con l'interno riempito dal
// carattere asterisco (si veda la stampa di esempio)
// QUESTA FUNZIONE DEVE ESSERE RICORSIVA
void triangle(int h) {
  // completare
}

int main(void) {
  triangle(5);
  triangle(8);
  return 0;
}
  \end{lstlisting}
\end{center}

Se tutto \`e corretto, tale programma deve stampare due triangoli, il
primo di altezza 5 e il secondo di altezza 8:

\begin{mdframed}[backgroundcolor=lightgrey] 
\begin{verbatim}
@***@
@**@
@*@
@@
@
@******@
@*****@
@****@
@***@
@**@
@*@
@@
@
\end{verbatim}
\end{mdframed}

\begin{mdframed}[backgroundcolor=lightred] 
  \textbf{Se servisse, negli esercizi si possono aggiungere funzioni ausiliarie.}
\end{mdframed}

\end{document}
