\documentclass[12pt]{article}
\usepackage{amssymb}
\usepackage{amsmath}
\usepackage{color}
\usepackage{graphicx}
\usepackage{mdframed}
\usepackage{listings, xcolor}
\usepackage{textcomp}

\definecolor{verylightgray}{rgb}{.97,.97,.97}
\definecolor{lightred}{rgb}{.97,.50,.50}

\lstdefinelanguage{myC}{
        keywords=[1]{break, case, continue, default, do
, else, false, for, if, const, return, switch, true, while}, % generic keywords
        keywordstyle=[1]\color{blue}\bfseries,
        keywords=[2]{bool, int, long, float, double, byte, short, char, void, signed, unsigned}, % types
        keywordstyle=[2]\color{teal}\bfseries,
        keywordstyle=[2]\color{violet}\bfseries,
        keywords=[3]{NULL},
        keywordstyle=[3]\color{teal}\bfseries,
        identifierstyle=\color{black},
        sensitive=false,
        comment=[l]{//},
        morecomment=[s]{/*}{*/},
        commentstyle=\color{violet}\ttfamily,
        commentstyle=\small\ttfamily\color{myGreen},
        stringstyle=\color{red}\ttfamily,
        morestring=[b]',
        morestring=[b]"
}

\lstset{
        language=myC,
        backgroundcolor=\color{verylightgray},
        extendedchars=true,
        basicstyle=\small\ttfamily,
        showstringspaces=false,
        showspaces=false,
        numbers=none,
        numberstyle=\small,
        numbersep=9pt,
        tabsize=2,
        upquote=true,
        breaklines=true,
        showtabs=false,
        captionpos=b
        otherkeywords={define,include,\# }
}

\definecolor{myBlue}{rgb}{0.5,0.5,1}
\definecolor{myLightBlue}{rgb}{0.35,0.6,0.8}
\definecolor{myBlack}{rgb}{0,0,0}
\definecolor{myGreen}{rgb}{0.1,0.6,0.2}
\definecolor{myGray}{rgb}{0.5,0.5,0.5}
\definecolor{myLightgray}{rgb}{0.95,0.95,0.95}
\definecolor{myMauve}{rgb}{0.58,0,0.82}
\lstdefinelanguage{customc}{
    language=C,
    backgroundcolor = \color{myLightgray},
    basicstyle=\small\ttfamily\color{myBlack},
    keywordstyle=\color{myLightBlue},
    keywordstyle=[2]\color{red},
    commentstyle=\small\ttfamily\color{myGreen},
    morekeywords={RequirePackage,ProvidesPackage},
    %
    % The special highlighting works for '!if', '!endif' and '!else'
    % But it doesn't work for '#if', '#endif' and '#else'.
    alsoletter = {!},
    keywords=[2]{!if,!endif,!else},
    otherkeywords={define,include,\# }
}

\lstdefinestyle{myCustomc}{
    language = customc,
    % keywordstyle = \color{myMauve},
}

\lstset{escapechar=@,style=myCustomc}


\definecolor{grey}{rgb}{0.3,0.3,0.3}
\definecolor{lightgrey}{rgb}{0.9,0.9,0.9}

\thispagestyle{empty}
\setlength{\textwidth}{18.5cm}
\setlength{\topmargin}{-2.5cm}
\setlength{\textheight}{24.5cm}
\setlength{\oddsidemargin}{-1cm}
\setlength{\evensidemargin}{-1cm}

\begin{document}
\begin{center}{\LARGE Compito di Programmazione I - Bioinformatica}\\
\begin{center}
  \large 24 febbraio 2023 (tempo disponibile: 2 ore)
\end{center}
\end{center}

\vspace*{1ex}
\begin{center}{\Large Esercizio 1} ($31$ punti)\\
  \textbf{(si consegni \texttt{marziano.c})}
\end{center}

Un array di caratteri \`e detto \emph{alfabetico} se i suoi caratteri
sono tutti lettere minuscole o maiuscole dell'alfabeto inglese (eventualmente ripetute).
Per esempio, l'array \texttt{['i','P','b','J','j','E']} \`e alfabetico
mentre l'array \texttt{['j','(','b','j','A','e']} non \`e alfabetico.

Un array di caratteri \`e detto \emph{marziano} se \`e alfabetico
e, inoltre, \`e composto da una prima parte che contiene
consonanti, in ordine alfabetico inverso, e da una seconda parte che contiene
vocali (italiane), sempre in ordine alfabetico inverso (si ricordi che nell'ordine alfabetico
le maiuscole vengono prima delle minuscole).
Per esempio, l'array
\[
\mbox{\texttt{[}}\underbrace{\mbox{\texttt{'j','b','P','J'}}}_{\text{consonanti in ordine alfabetico inverso}}\mbox{\texttt{,}}\underbrace{\mbox{\texttt{'i','E'}}}_{\text{vocali in ordine alfabetico inverso}}\mbox{\texttt{]}}
\]
\`e marziano.
Invece l'array \texttt{['i','P','b','J','j','E']} non \`e marziano, perch\'e la
vocale \texttt{i} precede la consonante \texttt{J}.
Neanche l'array \texttt{['b','j','P','J','i','E']} \`e marziano, perch\'e
le consonanti non sono in ordine alfabetico inverso.
E neanche l'array \texttt{['j','b,'P','J','E','i']} \`e marziano, perch\'e
le vocali non sono in ordine alfabetico inverso.

Si completino le cinque funzioni del programma \texttt{marziano.c}:

\begin{center}
  \begin{lstlisting}[language=myC]
// aggiungete gli #include necessari
#include "marziano.h"

// inizializza l'array indicato, lungo length,
// in modo che diventi un array alfabetico casuale
// (caratteri alfabetici minuscoli o maiuscoli, eventualmente ripetuti)
void init_random(char arr[], int length) {
  // completare
}

// stampa l'array indicato, su una riga, senza spazi fra i caratteri,
// andando a capo alla fine
void print(char arr[], int length) {
  // completare
}

// riceve un array alfabetico arr e ne sposta i caratteri in modo che
// arr diventi marziano (prima le consonanti, in ordine alfabetico inverso,
// poi le vocali, in ordine alfabetico inverso)
void ordina_marziano(char arr[], int length) {
  // completare
}

// stampa una lista di caratteri, senza spazio fra i caratteri,
// andando a capo alla fine
void print_list(struct element_t *l) {
  // completare
}

// riceve un array marziano lungo length e restituisce una lista
// che contiene solo le vocali dell'array e senza ripetizioni:
// una vocale viene inserita nella lista solo la prima volta che
// compare, mentre la seconda volta non viene inserita nella lista
struct element_t *vocali_non_ripetute(char arr[], int length) {
  // completare
}
  \end{lstlisting}
\end{center}

\begin{mdframed}[backgroundcolor=lightred] 
  \textbf{I file \texttt{marziano.h} e \texttt{main.c} sono gi\`a scritti e completi, non vanno modificati e non vanno consegnati. Se servisse, si possono aggiungere funzioni ausiliarie dentro \texttt{marziano.c}.}
\end{mdframed}

\vspace*{5ex}
Se tutto \`e corretto,
un esempio di esecuzione del \texttt{main.c} potrebbe essere:

\begin{mdframed}[backgroundcolor=verylightgray] 
\begin{verbatim}
Inserisci la lunghezza dell'array, non negativa: 30
                           Array alfabetico casuale: lvPqQgGOIadSbSeUafTKuRtbIafQYI
                    Array trasformato in marziano: vtqlgffdbbYTSSRQQPKGueaaaUOIII
Lista derivata dall'array eliminando le consonanti e le vocali ripetute: ueaUOI

\end{verbatim}
\end{mdframed}

\end{document}
