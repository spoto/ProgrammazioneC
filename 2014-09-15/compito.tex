\documentclass[12pt]{article}
\usepackage{amssymb}
\usepackage{amsmath}
\usepackage{graphicx}

\thispagestyle{empty}
\setlength{\textwidth}{18.5cm}
\setlength{\topmargin}{-2.5cm}
\setlength{\textheight}{24.5cm}
\setlength{\oddsidemargin}{-1cm}
\setlength{\evensidemargin}{-1cm}
\begin{document}
\begin{center}{\LARGE Esame di Programmazione I}\\
\vspace*{-2ex}
\begin{center}
  \large 15 settembre 2014 (tempo disponibile: 2 ore)
\end{center}
\end{center}
%\mbox{}\\
%\vspace*{1ex}
\begin{center}{\Large Esercizio 1} ($11$ punti)
\end{center}
Si scriva una funzione \texttt{char *histo(int values[], int length)}
che restituisce una nuova stringa che rappresenta \texttt{length} istogrammi orizzontali fatti
da asterischi. Il primo \`e
fatto da \texttt{values[0]} asterischi; il secondo da \texttt{values[1]} asterischi ecc.

Se tutto \`e corretto, l'esecuzione del seguente programma:

{\small
\begin{verbatim}
int main(void) {
  int values[] = { 3, 8, 0, 1, 5 };
  char *s = histo(values, 5);
  printf("%s", s);
  free(s);
  return 0;
}
\end{verbatim}}

\noindent
dovr\`a stampare:
%
{\small
\begin{verbatim}
***
********

*
*****
\end{verbatim}}

\vspace*{1ex}
\begin{center}{\Large Esercizio 2} ($11$ punti)\end{center}
%
Si considerino le liste di \texttt{int} come viste a lezione. Si definisca una funzione
\textbf{ricorsiva}
%
\begin{verbatim}
int compare_sum(struct list *this, struct list *that)
\end{verbatim}
%
che restituisce un intero negativo se la somma degli elementi di \texttt{this} \`e maggiore
della somma degli elementi di \texttt{that}; un intero positivo se la somma degli elementi di
\texttt{that} \`e maggiore
della somma degli elementi di \texttt{this}; e ritorna 0 altrimenti. Per semplicit\`a, si assuma
che \texttt{this} e \texttt{that} abbiano la stessa lunghezza. La funzione {\tt compare\_sum} non deve richiamare altre funzioni ricorsive, se non se stessa.

Se tutto \`e corretto, l'esecuzione del seguente programma:

{\small
\begin{verbatim}
int main(void) {
  struct list *l1 = construct_list
    (2, construct_list(5, construct_list(4, construct_list(-3, NULL))));
  struct list *l2 = construct_list
    (-5, construct_list(13, construct_list(16, construct_list(26, NULL))));
  printf("The Answer to The Ultimate Question of Life: %i\n", compare_sum(l1, l2));
  return 0;
}
\end{verbatim}}

\noindent
dovr\`a stampare un numero positivo (non necessariamente $42$\ldots):
%
{\small
\begin{verbatim}
The Answer to The Ultimate Question of Life: 42
\end{verbatim}}

\vspace*{1ex}
\begin{center}{\Large Esercizio 3} ($10$ punti)\end{center}
%
Si scrivano i file \texttt{casa.h} e \texttt{casa.c} che definiscono una casa, con le tre metrature
essenziali che possono interessare un'agenzia immobiliare, con le seguenti funzioni sulle case:

{\small
\begin{verbatim}
struct casa *construct_casa(int metratura_casa, int metratura_terrazzi, int metratura_garage);
void destroy_casa(struct casa *this);
double metratura_commerciale(struct casa *this);
\end{verbatim}}

\noindent
La funzione di costruzione \texttt{construct\_casa} richiede di specificare i metri quadri della casa,
dei suoi eventuali terrazzi e del suo eventuale garage. La metratura commerciale di una casa \`e
definita come i metri quadri della casa, pi\'u il $70\%$ dei metri quadri dei terrazzi pi\'u il $50\%$
dei metri quadri del garage.

Quindi si definiscano i file \texttt{vendita.h} e \texttt{vendita.c} che definiscono una vendita immobiliare di
una casa. Una vendita \`e specificata dalla casa e dal prezzo a cui \`e venduta. Devono essere definite e implementate
le seguenti funzioni:

{\small
\begin{verbatim}
struct vendita *construct_vendita(struct casa* casa, int prezzo);
void destroy_vendita(struct vendita *this);
double prezzo_m2(struct vendita *this);
\end{verbatim}}

\noindent
dove la funzione \texttt{prezzo\_m2} calcola il prezzo della vendita al metro quadro (prezzo della vendita
diviso i metri quadri commerciali della casa venduta).

Se tutto \`e corretto, l'esecuzione del seguente programma:

{\small
\begin{verbatim}
#include <stdio.h>
#include "casa.h"
#include "vendita.h"

int main(void) {
  struct casa *c1 = construct_casa(80, 20, 15);
  struct casa *c2 = construct_casa(70, 0, 0);
  printf("c1 ha una metratura commerciale di %.2f m2\n", metratura_commerciale(c1));
  printf("c2 ha una metratura commerciale di %.2f m2\n", metratura_commerciale(c2));
  // la stessa casa viene venduta due volte a prezzi diversi
  struct vendita *v1 = construct_vendita(c1, 200000);
  struct vendita *v2 = construct_vendita(c1, 250000);
  printf("c1 e' stata venduta una prima volta a %.2f euro al m2\n", prezzo_m2(v1));
  printf("c1 e' stata venduta una seconda volta a %.2f euro al m2\n", prezzo_m2(v2));

  return 0;
}
\end{verbatim}}

\noindent
dovr\`a stampare:

{\small
\begin{verbatim}
c1 ha una metratura commerciale di 101.50 m2
c2 ha una metratura commerciale di 70.00 m2
c1 e' stata venduta una prima volta a 1970.44 euro al m2
c1 e' stata venduta una seconda volta a 2463.05 euro al m2
\end{verbatim}}

\end{document}
