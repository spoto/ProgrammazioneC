\documentclass[12pt]{article}
\usepackage{amssymb}
\usepackage{amsmath}
\usepackage{color}
\usepackage{graphicx}

\definecolor{grey}{rgb}{0.3,0.3,0.3}

\usepackage{listings, framed}
\lstset{
  language=Java,
  showstringspaces=false,
  columns=flexible,
  basicstyle={\small\ttfamily},
  frame=none,
  numbers=none,
  keywordstyle=\bfseries\color{grey},
  commentstyle=\itshape\color{red},
  identifierstyle=\color{black},
  stringstyle=\color{blue},
  numberstyle={\ttfamily},
%  breaklines=true,
  breakatwhitespace=true,
  tabsize=3,
  escapechar=|
}

\thispagestyle{empty}
\setlength{\textwidth}{18.5cm}
\setlength{\topmargin}{-2.5cm}
\setlength{\textheight}{24.5cm}
\setlength{\oddsidemargin}{-1cm}
\setlength{\evensidemargin}{-1cm}
\begin{document}

\begin{center}
{\LARGE Secondo parziale di Programmazione I - BioInformatica}\\
\vspace*{-2ex}
\begin{center}
{\large 21 Giugno 2018 (tempo disponibile: 2 ore)}
\end{center}
\end{center}

\begin{center}
{\Large Esercizio 1} ($14$ punti)
\end{center}

Si definisca una struttura \texttt{studente} che implementa uno studente.
Si scrivano i file \texttt{studente.h} e \texttt{studente.c} implementando le funzioni:
%
\begin{itemize}
\itemsep0em 
\item \texttt{struct studente *construct\_studente(char *nome)} che restituisce un
      nuovo studente con il nome indicato;
\item \texttt{void destruct\_studente(struct studente *this)} che dealloca lo studente \texttt{this};
\item \texttt{void fa\_esame(struct studente *this, int voto)},
      che registra il voto indicato per lo studente \texttt{this}, se il voto \`e
      fra $18$ e $30$ inclusi, e non fa nulla altrimenti; uno studente pu\`o fare al pi\`u 20 esami:
      oltre tale soglia, questa funzione non registra pi\`u ulteriori esami;
\item \texttt{float media(struct studente *this)}, che restituisce la media dei voti
      degli esami sostenuti dallo studente \texttt{this}; se lo studente non ha ancora fatto esami,
      restituisce $0.0$;
\item \texttt{char *toString(struct studente *this)}, che restituisce una nuova stringa
      fatta dal nome dello studente \texttt{this} seguito dalla media degli esami sostenuti da
      \texttt{this}.
\end{itemize}
%
Se tutto \`e corretto, l'esecuzione del seguente programma:

\begin{lstlisting}
#include <stdlib.h>
#include <stdio.h>
#include "studente.h"

int main(void) {
  struct studente *s1 = construct_studente("Giacomo");
  struct studente *s2 = construct_studente("Elisa");
  char *s;
  fa_esame(s1, 18);
  fa_esame(s1, 15);  // non viene registrato
  fa_esame(s2, 30);
  fa_esame(s1, 25);
  fa_esame(s2, 22);
  fa_esame(s2, 29);
  fa_esame(s2, 27);
  printf("%s\n", s = toString(s1));
  free(s);
  printf("%s\n", s = toString(s2));
  free(s);
  destruct_studente(s1);
  destruct_studente(s2);
  return 0;
}
\end{lstlisting}

\noindent
deve stampare:

{\small
\begin{verbatim}
Giacomo  21.50
Elisa  27.00
\end{verbatim}
}


\vspace*{1ex}

% -----------------------------------------------------------------------------

\begin{center}
{\Large Esercizio 2} ($4$ punti)
\end{center}

\noindent Cosa stampa il seguente programma C?

\begin{lstlisting}
#include <stdio.h>
 
struct date {
  int day;
  int month;
  int year;
}

void changeDay(struct date d) {
  d.day = 12;  
}
 
int main(void) {
  struct date today;
  today.day = 21;
  today.month = 6;
  today.year = 2018;  
  changeDay(today);
  printf("La data modificata e': %i/%2i/%i", today.day, today.month, today.year); // cosa stampa?
  return 0;
}
\end{lstlisting}

\vspace*{1ex}

% -----------------------------------------------------------------------------

\begin{center}{\Large Esercizio 3} ($14$ punti)\end{center}

% vedi lezione 4 laboratorio 

Si completi il seguente programma scrivendo la  funzione 
\texttt{void somme(int *arr, int length)} che, dato un puntatore ad un array di interi \texttt{arr} di lunghezza \texttt{length}, modifica gli elementi di \texttt{arr} in modo tale che ogni elemento in posizione pari diventi uguale alla somma di quelli in posizione pari dall'array originario dall'inizio fino ad esso, e ogni elemento in posizione dispari diventi uguale alla somma di quelli in posizione dispari dell'array originario dall'inizio fino ad esso. Tale funzione dovr\`a utilizzare l'aritmetica dei puntatori per accedere agli elementi dell'array.


Se tutto \`e corretto l'esecuzione del seguente programma

\begin{lstlisting}
#include <stdio.h>

void somme(int*, int);

int main(void) {
  int arr[] = {5,2,10,6,8,9,7,5,6,7};
  int i, length = 10;
  somme( arr, length );  
  for(i = 0; i < length; i++) {
    printf( "%i ", *(arr+i) );
  }
  return 0;
}
\end{lstlisting}

\noindent deve stampare:
%
\texttt{5 2 15 8 23 17 30 22 36 29}

\end{document}
