\documentclass[12pt]{article}
\usepackage{amssymb}
\usepackage{amsmath}
\usepackage{color}
\usepackage{graphicx}
\usepackage{mdframed}
\usepackage{listings, xcolor}
\usepackage{textcomp}

\definecolor{verylightgray}{rgb}{.97,.97,.97}
\definecolor{lightred}{rgb}{.97,.50,.50}

\lstdefinelanguage{myC}{
        keywords=[1]{break, case, continue, default, do
, else, false, for, if, const, return, switch, true, while}, % generic keywords
        keywordstyle=[1]\color{blue}\bfseries,
        keywords=[2]{bool, int, long, float, double, byte, short, char, void, signed, unsigned}, % types
        keywordstyle=[2]\color{teal}\bfseries,
        keywordstyle=[2]\color{violet}\bfseries,
        keywords=[3]{NULL},
        keywordstyle=[3]\color{teal}\bfseries,
        identifierstyle=\color{black},
        sensitive=false,
        comment=[l]{//},
        morecomment=[s]{/*}{*/},
        commentstyle=\color{violet}\ttfamily,
        commentstyle=\small\ttfamily\color{myGreen},
        stringstyle=\color{red}\ttfamily,
        morestring=[b]',
        morestring=[b]"
}

\lstset{
        language=myC,
        backgroundcolor=\color{verylightgray},
        extendedchars=true,
        basicstyle=\small\ttfamily,
        showstringspaces=false,
        showspaces=false,
        numbers=none,
        numberstyle=\small,
        numbersep=9pt,
        tabsize=2,
        upquote=true,
        breaklines=true,
        showtabs=false,
        captionpos=b
        otherkeywords={define,include,\# }
}

\definecolor{myBlue}{rgb}{0.5,0.5,1}
\definecolor{myLightBlue}{rgb}{0.35,0.6,0.8}
\definecolor{myBlack}{rgb}{0,0,0}
\definecolor{myGreen}{rgb}{0.1,0.6,0.2}
\definecolor{myGray}{rgb}{0.5,0.5,0.5}
\definecolor{myLightgray}{rgb}{0.95,0.95,0.95}
\definecolor{myMauve}{rgb}{0.58,0,0.82}
\lstdefinelanguage{customc}{
    language=C,
    backgroundcolor = \color{myLightgray},
    basicstyle=\small\ttfamily\color{myBlack},
    keywordstyle=\color{myLightBlue},
    keywordstyle=[2]\color{red},
    commentstyle=\small\ttfamily\color{myGreen},
    morekeywords={RequirePackage,ProvidesPackage},
    %
    % The special highlighting works for '!if', '!endif' and '!else'
    % But it doesn't work for '#if', '#endif' and '#else'.
    alsoletter = {!},
    keywords=[2]{!if,!endif,!else},
    otherkeywords={define,include,\# }
}

\lstdefinestyle{myCustomc}{
    language = customc,
    % keywordstyle = \color{myMauve},
}

\lstset{escapechar=@,style=myCustomc}


\definecolor{grey}{rgb}{0.3,0.3,0.3}
\definecolor{lightgrey}{rgb}{0.9,0.9,0.9}

\thispagestyle{empty}
\setlength{\textwidth}{18.5cm}
\setlength{\topmargin}{-2.5cm}
\setlength{\textheight}{24.5cm}
\setlength{\oddsidemargin}{-1cm}
\setlength{\evensidemargin}{-1cm}

\begin{document}
\begin{center}{\LARGE Parziale di Programmazione - Bioinformatica}\\
  \large 6 febbraio 2025 (tempo disponibile: 2 ore)
\end{center}

\vspace*{1ex}
\begin{center}{\Large Esercizio 1} ($9$ punti)\\
  \textbf{(si consegni \texttt{array.c})}
\end{center}

Si completi il seguente file \texttt{array.c}:

\begin{center}
\begin{lstlisting}[language=myC]
// aggiungete #include se servono

void processa(int arr[], int length, int threshold) {
  // modifica arr, lungo length, azzerando gli elementi che sono
  // maggiori o uguali a threshold
}

void print(int arr[], int length) {
  // stampa arr, lungo length, su una riga, separando gli elementi
  // con uno spazio e andando a capo alla fine
}

int main(void) { // gia' fatto, non modificate
  int arr1[] = { 3, 5, 5, 15, 11, 30, 87 };
  processa(arr1, 7, 13);
  print(arr1, 7);

  int arr2[] = { -3, 2, 5, 20, -80, 21, 7, -11, -40 };
  processa(arr2, 9, -1);
  print(arr2, 9);

  return 0;
}
\end{lstlisting}
\end{center}
%
Se tutto \`e corretto, la sua esecuzione dovr\`a stampare:

\begin{mdframed}[backgroundcolor=lightgrey] 
\begin{verbatim}
3 5 5 0 11 0 0 
-3 0 0 0 -80 0 0 -11 -40 
\end{verbatim}
\end{mdframed}

\newpage
\begin{center}
  {\Large Esercizio 2} ($22$ punti)\\
  \textbf{(si consegnino \texttt{freccia.c} e \texttt{freccia.h})}
\end{center}

Si completi il seguente programma \texttt{freccia.c}.
\textbf{La funzione \texttt{freccia} deve essere ricorsiva}:

\begin{center}
  \begin{lstlisting}[language=myC]
#include <stdio.h>

void asterisco(int s) {
  // stampa s spazi, un asterisco e va a capo
}

void freccia(int s, int a) {
  // stampa una freccia di asterischi con la punta rivolta a destra,
  // alta a e con s spazi a sinistra; questa funzione deve essere ricorsiva;
  // si noti che a e' sempre dispari
}
  \end{lstlisting}
\end{center}

Si scriva il file \texttt{freccia.h} che dichiara le due funzioni in \texttt{freccia.c}.

Se tutto \`e corretto, tale programma deve potersi compilare insieme a \texttt{main.c}
(gi\`a fatto, da non modificare) e le seguenti sono  delle possibili esecuzioni:

\begin{mdframed}[backgroundcolor=lightgrey] 
\begin{verbatim}
$ ./freccia
Inserisci s non negativo: 0
Inserisci a positivo e dispari: 5
*
 *
  *
 *
*
$ ./freccia
Inserisci s non negativo: 2
Inserisci a positivo e dispari: 5
  *
   *
    *
   *
  *
$ ./freccia 
Inserisci s non negativo: 3
Inserisci a positivo e dispari: 3
   *
    *
   *
$ ./freccia
Inserisci s non negativo: 4
Inserisci a positivo e dispari: 1
    *
\end{verbatim}
\end{mdframed}

\end{document}
