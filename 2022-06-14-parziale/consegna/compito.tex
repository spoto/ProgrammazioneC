\documentclass[12pt]{article}
\usepackage{amssymb}
\usepackage{amsmath}
\usepackage{color}
\usepackage{graphicx}
\usepackage{mdframed}
\usepackage{listings, xcolor}
% \usepackage[dvipsnames]{xcolor} % Non va!
\usepackage{textcomp}

\lstdefinelanguage{myC}{
        keywords=[1]{break, case, continue, default, do
, else, false, for, if, const, return, switch, true, while}, % generic keywords
        keywordstyle=[1]\color{blue}\bfseries,
        keywords=[2]{bool, int, long, float, double, byte, short, char, void, signed, unsigned}, % types
        keywordstyle=[2]\color{teal}\bfseries,
        keywordstyle=[2]\color{violet}\bfseries,
        keywords=[3]{NULL},
        keywordstyle=[3]\color{violet}\bfseries,
        identifierstyle=\color{black},
        sensitive=false,
        comment=[l]{//},
        morecomment=[s]{/*}{*/},
        commentstyle=\color{violet}\ttfamily,
        commentstyle=\normalsize\ttfamily\color{grey}, % scriptsize
        stringstyle=\color{forestgreen}\ttfamily,
        morestring=[b]',
        morestring=[b]"
}

\lstset{
        language=myC,
        backgroundcolor=\color{veryLightgray},
        extendedchars=true,
        basicstyle=\small\ttfamily,
        showstringspaces=false,
        showspaces=false,
        numbers=none,
        numberstyle=\normalsize,
        numbersep=9pt,
        tabsize=2,
        upquote=true,
        breaklines=true,
        showtabs=false,
        captionpos=b
}

\definecolor{myBlue}{rgb}{0.5,0.5,1}
\definecolor{myLightBlue}{rgb}{0.35,0.6,0.8}
\definecolor{myBlack}{rgb}{0,0,0}
\definecolor{myGreen}{rgb}{0.1,0.6,0.2}
\definecolor{myGray}{rgb}{0.5,0.5,0.5}
\definecolor{myLightgray}{rgb}{0.95,0.95,0.95}
\definecolor{verylightgray}{rgb}{.97,.97,.97}
\definecolor{myMauve}{rgb}{0.58,0,0.82}
\definecolor{forestgreen}{rgb}{0.13, 0.55, 0.13}

\lstdefinelanguage{customc}{
    language=C,
    backgroundcolor = \color{white},
    basicstyle=\small\ttfamily\color{myBlack},
    keywordstyle=\color{myLightBlue},
    keywordstyle=[2]\color{red},
    commentstyle=\small\ttfamily\color{forestgreen},
    morekeywords={RequirePackage,ProvidesPackage},
    %
    % The special highlighting works for '!if', '!endif' and '!else'
    % But it doesn't work for '#if', '#endif' and '#else'.
    alsoletter = {!},
    keywords=[2]{!if,!endif,!else},
}

\lstdefinestyle{myCustomc}{
    language = customc,
    % keywordstyle = \color{myMauve},
}

\lstset{escapechar=@,style=myCustomc}


\definecolor{grey}{rgb}{0.3,0.3,0.3}
\definecolor{lightgrey}{rgb}{0.9,0.9,0.9}

\thispagestyle{empty}
\setlength{\textwidth}{18.5cm}
\setlength{\topmargin}{-2.5cm}
\setlength{\textheight}{24.5cm}
\setlength{\oddsidemargin}{-1cm}
\setlength{\evensidemargin}{-1cm}

\newcommand\ddfrac[2]{\frac{\displaystyle #1}{\displaystyle #2}}

\begin{document}
\begin{center}{\LARGE Secondo parziale di Programmazione I - Bioinformatica}\\
%\vspace*{-1ex}
\begin{center}
  \large 14 giugno 2022 (tempo disponibile: 2 ore)
\end{center}
\end{center}
\noindent\makebox[\linewidth]{\rule{\linewidth}{0.4pt}}
\begin{center}
INDICAZIONI GENERALI
\end{center}
\begin{itemize}
\item{Utilizzare il comando \texttt{ulimit -v 500000} per limitare l'utilizzo delle risorse al terminale su cui viene eseguito il comando ed evitare spiacevoli inconvenienti dovuti ad eccessive allocazioni di memoria.}
\item{Scaricare il file di ogni esercizio e riconsegnarlo senza modificarne il nome.}
\item{I file non consegnati o consegnati con errori di compilazione non verrano presi in considerazione. }
\item{Si possono utilizzare funzioni aggiuntive non presenti nei file modello e aggiungere linee di commento alle funzioni gi\'a implementate nel modello.}
\item{I file possono essere consegnati pi\'u volte. Per ogni esercizio, solo l'ultimo file consegnato sar\'a considerato valido.}
\item{Compilare con l'opzione \texttt{-Wall} per avere tutti i warning.}
\item {È vietato utilizzare funzioni di libreria diverse da \texttt{scanf()}, \texttt{printf()}, \texttt{malloc()} e \texttt{sizeof()}.}
\end{itemize}
\noindent\makebox[\linewidth]{\rule{\linewidth}{0.4pt}}


% ESERCIZIO 1
\begin{center}{\Large Esercizio 1} ($18$ punti)\\
  \textbf{(si consegni \texttt{letter\_list.c})}\end{center}
Si completi programma \texttt{letter\_list.c}, in modo che la funzione \texttt{main()} riportata di seguito:
\begin{center}
\begin{lstlisting}[language=myC]
int main() {
    int n = 10;
    struct letter_t *head = NULL;
    srand(0);
    for (int i = 0; i < n; i++) {
        char ch = 'A' + rand() % NUM_OF_LETTERS;
        head = insert_letter_in_tail(head, ch);
    }
    print_list(head);
    printf("Occorrenze di A: %d\n", count_letter(head, 'A'));
    return 0;
}

\end{lstlisting}
\end{center}
produca il seguente ouput:
\begin{mdframed}[backgroundcolor=lightgrey] 
\begin{verbatim}
B B A B C B B A A B 
Occorrenze di A: 3
\end{verbatim}
\end{mdframed}


\newpage
% ESERCIZIO 2
\vspace*{8ex}
\begin{center}{\Large Esercizio 2} ($14$ punti)\\
  \textbf{(si consegni \texttt{matrix\_density.c})}
\end{center}
Completare il file \texttt{matrix\_density.c} che ha la seguente funzione \texttt{main()}:

\begin{lstlisting}[language=myC]
int main() {
    int matrix1[SIZE][SIZE] = {{0, 2, 0}, {1, 0, 0}, {4, 0, 8}};
    int matrix2[SIZE][SIZE] = {{0, 2, 0}, {4, 0, 6}, {0, 8, 9}};

    print_matrix(matrix1, SIZE);
    printf("Matrice densa? %i\n---\n", is_dense(matrix1, SIZE));

    print_matrix(matrix2, SIZE);
    printf("Matrice densa? %i\n---\n", is_dense(matrix2, SIZE));

    return 0;
}
\end{lstlisting}
in modo che produca il seguente output:
\begin{mdframed}[backgroundcolor=verylightgray] 
\begin{verbatim}
0 2 0 
1 0 0 
4 0 8 
Matrice densa? 0
---
0 2 0 
4 0 6 
0 8 9 
Matrice densa? 1
---
\end{verbatim}
\end{mdframed}


\end{document}
