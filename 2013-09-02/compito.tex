\documentclass{article}[10pt]
\usepackage[pdftex]{graphicx}
\usepackage{amsfonts}
\usepackage[italian]{babel}
%****************enlarge layout
\textheight     243.5mm
\topmargin      -20.0mm
\textwidth      480pt
\hoffset        -80pt
%*****************theorems and such
\newcounter{esnu}
\newenvironment{esercizio}{\medskip \noindent {\bf Esercizio\addtocounter{esnu}{1} \arabic{esnu}}}{}
\pagestyle{empty}
\newcommand{\liff}{\mathrel{\leftrightarrow}}   % Logical IFF Symbol
\newcommand{\metaiff}{\Longleftrightarrow}      %iff in metatheory

\begin{document}

\begin{center} {\bf Esame di Programmazione I, 2 settembre 2013. 2 ore}\end{center}

\begin{esercizio}
\textbf{[7 punti]}
Si scriva una funzione \underline{ricorsiva}

\begin{verbatim}
int valida(const char *s)
\end{verbatim}

\noindent
che determina se la stringa \texttt{s} \`e un'alternanza di cifre e non-cifre
(una cifra, una non cifra, una cifra, una non cifra, ecc.). Per esempio,
\texttt{3e6y2\#0i5u} soddisfa tale propriet\`a. Si noti che tali
stringhe hanno sempre lunghezza pari.

\end{esercizio}

\begin{esercizio}
\textbf{[13 punti]}
Si scriva una funzione

\begin{verbatim}
char *espandi(const char *s)
\end{verbatim}

\noindent
che controlla se \texttt{s} \`e un'alternanza di cifre e non cifre e in caso
contrario ritorna \texttt{NULL}. Altrimenti ritorna una nuova stringa ottenuta
ripetendo le non cifre il numero di volte indicato dalla cifra alla loro sinistra. Per
esempio, se \texttt{s} \`e \texttt{3e6y2\#0i5u}, il risultato sar\`a la stringa
\texttt{eeeyyyyyy\#\#uuuuu}

Se tutto \`e corretto, una esecuzione del seguente programma:

{\small
\begin{verbatim}
#include <stdio.h>
#include <stdlib.h>

int main(void) {
  char buffer[100];
  char *res;

  printf("input: ");
  scanf("%s", buffer);
  res = espandi(buffer);

  if (res) {
    printf("output: %s\n", res);
    free(res);
  }
  else
    printf("formato errato\n");

  return 0;
}
\end{verbatim}
}

\noindent
potrebbe essere:

{\small
\begin{verbatim}
input: 3e4r2#0&9i               
output: eeerrrr##iiiiiiiii
\end{verbatim}
}
\end{esercizio}

\begin{esercizio}
\textbf{[11 punti]}
Si consideri la seguente specifica di un iscritto a un esame con il relativo voto:

{\small
\begin{verbatim}
#ifndef ISCRITTO_H
#define ISCRITTO_H

struct iscritto {
  const char *nome;
  const char *cognome;
  int matricola;
  int voto;
};

struct iscritto *construct_iscritto(const char *nome, const char *cognome, int matricola);
void destroy_iscritto(struct iscritto *this);
void print_iscritto(struct iscritto *this);

#endif
\end{verbatim}
}

\noindent
con la relativa implementazione:
%
{\small
\begin{verbatim}
#include <stdlib.h>
#include <stdio.h>
#include "iscritto.h"

struct iscritto *construct_iscritto(const char *nome, const char *cognome, int matricola) {
  struct iscritto *this = malloc(sizeof(struct iscritto));
  this->nome = nome;
  this->cognome = cognome;
  this->matricola = matricola;
  this->voto = -1;
  return this;
}

void destroy_iscritto(struct iscritto *this) {
  free(this);
}

void print_iscritto(struct iscritto *this) {
  printf("%10s %10s   (VR%i)\t\t", this->nome, this->cognome, this->matricola);

  if (this->voto >= 18)
    printf("%i", this->voto);
  else if (this->voto >= 0)
    printf("insufficiente");
  else
    printf("assente");
}
\end{verbatim}
}

Si scrivano i file \texttt{appello.h} e \texttt{appello.c} in modo da implementare un appello d'esame.
Un appello deve contenere una quantit\`a variabile di iscritti, non nota a priori, che pu\`o crescere
arbitrariamente. Si implementino le seguenti funzioni:
%
{\small
\begin{verbatim}
struct appello *construct_appello();
void destroy_appello(struct appello *this);
void iscrivi(struct appello *this, struct iscritto *studente);
void registra_voto(struct appello *this, int matricola, int voto);
void print_appello(struct appello *this);
\end{verbatim}
}

\noindent
La prima restituisce un nuovo appello senza ancora alcun iscritto. La seconda dealloca
l'appello e tutti gli iscritti contenuti al suo interno. Il terzo permette di iscrivere uno studente
a un appello. Il quarto registra un voto a uno studente, identificato tramite la sua matricola,
e l'ultimo stampa un appello, cio\`e tutti i suoi iscritti in sequenza.

Se tutto \`e corretto, l'esecuzione del programma:

{\small
\begin{verbatim}
#include "appello.h"
#include "iscritto.h"

int main(void) {
  struct appello *app = construct_appello();
  iscrivi(app, construct_iscritto("Fausto", "Spoto", 153535));
  iscrivi(app, construct_iscritto("Giovanni", "Senzaterra", 123456));
  iscrivi(app, construct_iscritto("Roberta", "Rossi", 310453));
  registra_voto(app, 123456, 28);
  registra_voto(app, 153535, 17);
  print_appello(app);
  destroy_appello(app);
  return 0;
}
\end{verbatim}
}
\noindent
dovr\`a stampare:

{\small
\begin{verbatim}
   Roberta      Rossi   (VR310453)		assente
  Giovanni Senzaterra   (VR123456)		28
    Fausto      Spoto   (VR153535)		insufficiente
\end{verbatim}
}
%
\end{esercizio}
%
\end{document}
