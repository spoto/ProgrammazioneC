\documentclass[12pt]{article}
\usepackage{amssymb}
\usepackage{amsmath}
\usepackage{color}
\usepackage{graphicx}

\definecolor{grey}{rgb}{0.3,0.3,0.3}

\usepackage{listings, framed}
\lstset{
  language=Java,
  showstringspaces=false,
  columns=flexible,
  basicstyle={\small\ttfamily},
  frame=none,
  numbers=none,
  keywordstyle=\bfseries\color{grey},
  commentstyle=\itshape\color{red},
  identifierstyle=\color{black},
  stringstyle=\color{blue},
  numberstyle={\ttfamily},
%  breaklines=true,
  breakatwhitespace=true,
  tabsize=3,
  escapechar=|
}

\thispagestyle{empty}
\setlength{\textwidth}{18.5cm}
\setlength{\topmargin}{-2.5cm}
\setlength{\textheight}{24.5cm}
\setlength{\oddsidemargin}{-1cm}
\setlength{\evensidemargin}{-1cm}
\begin{document}
\begin{center}{\LARGE Esame Completo di Programmazione I - Bioinformatica}\\
%\vspace*{-1ex}
\begin{center}
  \large 1 febbraio 2021 (tempo disponibile: 2 ore)
\end{center}
\end{center}

\vspace*{1ex}
\begin{center}{\Large Esercizio 1} ($18$ punti)\\
  \textbf{(si consegni \texttt{penta.c} e \texttt{penta.h})}
\end{center}
Si scriva un programma \texttt{penta.c} che implementa le seguenti funzioni:
\begin{verbatim}
// inizializza arr, lungo length, con numeri interi casuali tra 0 a 999,
// usando srand() e rand()
void init_random(int arr[], int length);

// stampa su un'unica riga il contenuto dell'array arr, lungo length, poi va a capo
void print(int arr[], int length);

// determina se il numero n non negativo e' pentafratto,
// cioe' se ha almeno 5 divisori interi positivi
int is_pentafract(int n);

// modifica l'array, lungo length, in modo da spostare al suo inizio i suoi elementi
// pentafratti e alla sua fine i suoi elementi non pentafratti
void pentafract_first(int arr[], int length);
\end{verbatim}
%
Si scriva quindi un file di header \texttt{penta.h} che dichiara le precedenti funzioni.

\mbox{}\\
\begin{center}{\Large Esercizio 2} ($14$ punti)\\
  \textbf{(si consegni \texttt{main.c})}\end{center}
%
Si scriva un programma \texttt{main.c} che include le funzioni dell'Esercizio~1 tramite
il file \texttt{penta.h}.
Il programma \texttt{main.c} deve definire una struttura

\begin{verbatim}
struct list {
  int x;
  struct list *tail;
};
\end{verbatim}

\noindent
e deve implementare due funzioni:

\begin{verbatim}
// alloca una struct list contenente x e tail e ne restituisce il puntatore in memoria
struct list *construct(int x, struct list *tail) { ... }

// determina se la lista l contiene almeno un elemento x pentafratto
// (si chiami la funzione opportuna dell'Esercizio 1)
int at_least_one(struct list *l) { ... }
\end{verbatim}

\noindent
Inoltre \texttt{main.c} deve contenere una funzione iniziale \texttt{main} che esegue
le seguenti operazioni:
\begin{enumerate}
\item crea una lista $5\rightarrow 12\rightarrow 6\rightarrow 15$;
\item chiama \texttt{at\_least\_one} passando tale lista come parametro;
\item stampa il valore ritornato dalla chiamata di funzione del punto precedente.
\end{enumerate}

\end{document}
